%
% [2020-10-02] Move this vocabulary file from SVN to github.
% [2018-08-24] 这个TeX文件最终可能只当作天天练的内容,里面的词汇将慢慢移动到ch_concret_eng.tex
%              文件当中去!
% [2015-10-19] Move all here, work well on TeX-Live2014, but Live-2015 seems
%              NOT GOOD for the ' symbole output.
%
% [2015-10-18] Try convert the concrete_eng.tex into XeLatex format

\documentclass[a4paper]{article}
\title{英语单词记忆卡片}
\author{Yang Songxiang}

%% MUST defined before fontspce
\usepackage{tipa}

\usepackage{fontspec,xunicode,xltxtra,makeidx,xecolor}
\usepackage{color,listings,tabularx,amsfonts}
\usepackage{amssymb}
\usepackage{wrapfig}
\usepackage{titlesec}
\usepackage{lettrine}
\usepackage{colortbl}
\usepackage{multicol}
\setlength{\columnseprule}{1pt}
\usepackage[many]{tcolorbox}
\usepackage[framemethod=tikz]{mdframed}
\usepackage{epigraph}
\usepackage{soul}
\usepackage[nonumberlist]{glossaries}
%
%%
\usepackage{multirow}
\usepackage{tikz}
\usepackage{tkz-base}
\usetikzlibrary{calc}
\usetikzlibrary{arrows,shapes,trees,calc,automata,positioning, fit}
\usetikzlibrary{calendar}
\usetikzlibrary{backgrounds}
\usetikzlibrary{angles}
\usetikzlibrary{graphs}

\tikzset{
  mybox/.style={
    rectangle,
    rounded corners,
    draw=black
  },
}

\usepackage[auto-label]{exsheets}
\usepackage{enumerate}
\usepackage{tasks}
\usepackage{indentfirst}

\usepackage[bookmarks=true,pdfborder={0 0 0}]{hyperref}
%%去除TexLive-2015开始fontspec对引号的影响
\defaultfontfeatures[\rmfamily,\sffamily,\ttfamily]{}
\setmainfont{WenQuanYi Micro Hei}
\setsansfont{WenQuanYi Micro Hei}
\setmonofont{WenQuanYi Micro Hei}

%required Windows fonts
\newcommand\fontnamekai{WenQuanYi Micro Hei}
% Tex-Live2016
%\newfontinstance\KAI {\fontnamekai}
\newfontfamily\KAI {\fontnamekai}
\newcommand{\kai}[1]{{\KAI#1}}
%\newfontfamily\mynamingfont {Liberation Serif}
%\newfontfamily\mynamingfontb {Century Schoolbook L}
%\newfontfamily\mynamingfontc {Droid Serif}
%\newfontfamily\mynamingfontd [Mapping=tex-text]{TeX Gyre Pagella}


\newcommand\vtextvisiblespace[1][.4em]{%
  \mbox{\kern.06em\vrule height.5ex}%
  \vbox{\hrule width#1}%
  \hbox{\vrule height.5ex}}

\setlength{\parindent}{2em}
\setlength{\parskip}{0.5\baselineskip}
\XeTeXlinebreaklocale "zh"
\XeTeXlinebreakskip=0pt plus 1pt minus 0.1pt

%listing global settings
\lstset{basicstyle=\scriptsize,frame=lines}

%some global color
\definecolor{light-gray}{rgb}{0.87,0.87,0.87}
\definecolor{light-yellow}{rgb}{0.88,0.92,0.48}
\definecolor{mygreen}{rgb}{0.63,1,0.35}

%complicated def
\lstnewenvironment{myjavacode}[1][]
      {\lstset{language=Java}\lstset{escapeinside={(*@}{@*)},
       basicstyle=\footnotesize\ttfamily,
       numbers=left,numberstyle=\scriptsize,stepnumber=1,numbersep=5pt,
       breaklines=true,
       %firstnumber=last,
           %frame=tblr,
           framesep=5pt,
           showstringspaces=false,
           keywordstyle=\itshape\color{blue},
          %identifierstyle=\ttfamily,
           stringstyle=\xecolor{maroon},
        commentstyle=\color{black},
        rulecolor=\color{black},
        xleftmargin=0pt,
        xrightmargin=0pt,
        aboveskip=\medskipamount,
        belowskip=\medskipamount,
               backgroundcolor=\color{white}, #1
}}
{}

%\newfontfamily\thekai{WenQuanYi Micro Hei}
\newfontfamily\thekai{STKaiti}
%%尝试把margainpar中的文字大小修改      
\setlength\marginparwidth{2cm}
\makeatletter
\long\def\@ympar#1{%
  \@savemarbox\@marbox{\thekai\footnotesize #1}%
  \global\setbox\@currbox\copy\@marbox
  \@xympar}
\makeatother

%% Try new a my own code 用于扩展词汇的统一处理
\newcounter{myexpandnum}
\newcounter{mydemonum}
%\newenvironment{myexpand2}[1][\unskip]{\colorbox{red}{Expand\refstepcounter{myexpandnum} \themyexpandnum #1}}{}

%\newenvironment{myexpand}[1][]{\refstepcounter{myexpandnum}  扩展词汇\themyexpandnum- \vspace{0.5cm} \tikz [baseline={([yshift=-.8ex]current bounding box.center)}] \node [signal, draw, text=white, fill=red!65!black, signal to=nowhere,signal from=east]{\large{ #1}};}{}

\newenvironment{myexpand}[1][]{\refstepcounter{myexpandnum} \tikz [baseline={([yshift=-.8ex]current bounding box.center)}] \shade [ball color=red] (0,0) circle (1ex); \colorbox{blue!30}{扩展词汇\themyexpandnum- \vspace{0.5cm}} \tikz [baseline={([yshift=-.8ex]current bounding box.center)}] \node [signal, draw, text=white, fill=red!65!black, signal to=nowhere,signal from=east]{\large{ #1}};}{}


\newenvironment{mydemosample}{\stepcounter{mydemonum} \tikz [baseline={([yshift=-.8ex]current bounding box.center)}] \shade [ball color=blue] (0,0) circle (1ex); \colorbox{blue!30}{例句 \themydemonum} 
\\
\rule{0.2\textwidth}{0.5mm} \begin{quotation}}{\end{quotation}}

%%这个用于适合打印的例句输出
\newcounter{printdemonum}
\newenvironment{printdemosample}{\stepcounter{printdemonum} \tikz [baseline={([yshift=-.8ex]current bounding box.center)}]  {\kai{例句} \theprintdemonum{}} \vspace{-0.41cm} \begin{quotation}} {\end{quotation}}

% a table collection
%\definecolor{Gray}{gray}{0.85}
%\definecolor{LightCyan}{rgb}{0.88,1,1}
%\newcolumntype{mycA}{>{\columncolor{Gray}}c}
%\newcolumntype{mycB}{>{\columncolor{LightCyan}}c}

%\newcommand\mywordnums[2]{First #1 , second #2}
\newcommand\mywordnums[2]{\bgroup \def\arraystretch{1.15} \scriptsize
  \begin{tabular}{|>{\columncolor[gray]{0.8}}c|>{\columncolor[rgb]{0.88,1,1}}c|}
   \hline
   {单数} & {复数} \\
   \hline
   {#1} & {#2} \\
   \hline
  \end{tabular}
\egroup}
  
%% 2016-12-22添加了一个类似tip类型的示例
%来源:http://tex.stackexchange.com/questions/171951/how-was-this-tip-box-produced/171954#171954
%%
%% 若需要修改TIP的标签位置,修改 \coordinate (aux) at ( $ (O)!0.5!(P) $ );的中间0.5这个数值
\newcounter{myTipStyleDemoCounter}

\newmdenv[
hidealllines=true,
innertopmargin=16pt,
innerbottommargin=10pt,
font=\sffamily\footnotesize,
leftmargin=-0.5cm,
rightmargin=-0.5cm,
skipabove=35pt,
skipbelow=15pt,
singleextra={
  \coordinate (aux) at ( $ (O)!0.5!(P) $ );
  \fill[rounded corners=8pt,line width=1pt,gray!30]
    (O|-P) -- 
    (aux|-P) --
    ([yshift=20pt]aux|-P) --
    ([yshift=20pt,xshift=4cm]aux|-P) --
    ([xshift=4cm]aux|-P) -- 
    (P) {[sharp corners] --
    ([yshift=-6pt]P) -- 
    ([yshift=-6pt]O|-P) } -- cycle;
  \draw[rounded corners=8pt,line width=1pt,gray]
    (O|-P) -- 
    (aux|-P) --
    ([yshift=20pt]aux|-P) --
    ([yshift=20pt,xshift=4cm]aux|-P) --
    ([xshift=4cm]aux|-P) -- 
    (P) --
    (P|-O) --
    (O) -- cycle;
  \node at ([xshift=2cm,yshift=6pt]aux|-P)
    {\refstepcounter{myTipStyleDemoCounter}\sffamily\large 例句~\themyTipStyleDemoCounter} ; 
  },
firstextra={
  \coordinate (aux) at ( $ (O)!0.5!(P|-O) $ );
  \fill[rounded corners=8pt,line width=1pt,gray!30,overlay]
    (O|-P) -- 
    (aux|-P) --
    ([yshift=20pt]aux|-P) --
    ([yshift=20pt,xshift=4cm]aux|-P) --
    ([xshift=4cm]aux|-P) -- 
    (P) {[sharp corners] --
    ([yshift=-6pt]P) -- 
    ([yshift=-6pt]O|-P) } -- cycle;
  \draw[rounded corners=8pt,line width=1pt,gray,overlay]
    (O) --
    (O|-P) -- 
    (aux|-P) --
    ([yshift=20pt]aux|-P) --
    ([yshift=20pt,xshift=4cm]aux|-P) --
    ([xshift=4cm]aux|-P) -- 
    (P) --
    (P|-O);
  \node[overlay] at ([xshift=2cm,yshift=6pt]aux|-P)
    {\refstepcounter{myTipStyleDemoCounter}\sffamily\large 例句~\themyTipStyleDemoCounter} ; 
  },
middleextra={
  \draw[rounded corners=8pt,line width=1pt,gray,overlay]
    (O|-P) -- 
    (O);
  \draw[rounded corners=8pt,line width=1pt,gray,overlay]
    (P) -- 
    (P|-O);
  },
secondextra={
  \coordinate (aux) at ( $ (O)!0.5!(P|-O) $ );
  \draw[rounded corners=8pt,line width=1pt,gray,overlay]
    (O|-P) -- 
    (O) --
    (P|-O) --
    (P);
  },
]{myTipStyleDemo}

%% 2016-12-22继续扩展上面的环境,打算做一个词汇积累类型的新环境,这个目前打算用来做英语的扩展词汇用,最终目标
%是替换掉上面的自己写的简单的myexpand环境
%
%来源:http://tex.stackexchange.com/questions/171951/how-was-this-tip-box-produced/171954#171954
\newcounter{myExpandWordCounter}

\newmdenv[
hidealllines=true,
innertopmargin=16pt,
innerbottommargin=10pt,
font=\sffamily\small,
leftmargin=0.0cm,
rightmargin=0.0cm,
skipabove=35pt,
skipbelow=15pt,
singleextra={
  \coordinate (aux) at ( $ (O)!0.1!(P) $ );
  \fill[rounded corners=8pt,line width=1pt,blue!30]
    (O|-P) -- 
    (aux|-P) --
    ([yshift=20pt]aux|-P) --
    ([yshift=20pt,xshift=4cm]aux|-P) --
    ([xshift=4cm]aux|-P) -- 
    (P) {[sharp corners] --
    ([yshift=-6pt]P) -- 
    ([yshift=-6pt]O|-P) } -- cycle;
  \draw[rounded corners=8pt,line width=1pt,gray]
    (O|-P) -- 
    (aux|-P) --
    ([yshift=20pt]aux|-P) --
    ([yshift=20pt,xshift=4cm]aux|-P) --
    ([xshift=4cm]aux|-P) -- 
    (P) --
    (P|-O) --
    (O) -- cycle;
  \node at ([xshift=2cm,yshift=6pt]aux|-P)
    {\refstepcounter{myExpandWordCounter}\sffamily\large 扩展词汇~\themyExpandWordCounter} ; 
  },
firstextra={
  \coordinate (aux) at ( $ (O)!0.5!(P|-O) $ );
  \fill[rounded corners=8pt,line width=1pt,blue!30,overlay]
    (O|-P) -- 
    (aux|-P) --
    ([yshift=20pt]aux|-P) --
    ([yshift=20pt,xshift=4cm]aux|-P) --
    ([xshift=4cm]aux|-P) -- 
    (P) {[sharp corners] --
    ([yshift=-6pt]P) -- 
    ([yshift=-6pt]O|-P) } -- cycle;
  \draw[rounded corners=8pt,line width=1pt,gray,overlay]
    (O) --
    (O|-P) -- 
    (aux|-P) --
    ([yshift=20pt]aux|-P) --
    ([yshift=20pt,xshift=4cm]aux|-P) --
    ([xshift=4cm]aux|-P) -- 
    (P) --
    (P|-O);
  \node[overlay] at ([xshift=2cm,yshift=6pt]aux|-P)
    {\refstepcounter{myExpandWordCounter}\sffamily\large 扩展词汇~\themyExpandWordCounter} ; 
  },
middleextra={
  \draw[rounded corners=8pt,line width=1pt,gray,overlay]
    (O|-P) -- 
    (O);
  \draw[rounded corners=8pt,line width=1pt,gray,overlay]
    (P) -- 
    (P|-O);
  },
secondextra={
  \coordinate (aux) at ( $ (O)!0.5!(P|-O) $ );
  \draw[rounded corners=8pt,line width=1pt,gray,overlay]
    (O|-P) -- 
    (O) --
    (P|-O) --
    (P);
  },
]{myExpandWord}



%%
%% 寒假学习计划的临时环境
%来源:http://tex.stackexchange.com/questions/171951/how-was-this-tip-box-produced/171954#171954
%%
%% 若需要修改TIP的标签位置,修改 \coordinate (aux) at ( $ (O)!0.5!(P) $ );的中间0.5这个数值
\newcounter{myPlanCounter}

\newmdenv[
hidealllines=true,
innertopmargin=16pt,
innerbottommargin=10pt,
font=\sffamily\footnotesize,
leftmargin=0.0cm,
rightmargin=1.8cm,
skipabove=35pt,
skipbelow=15pt,
singleextra={
  \coordinate (aux) at ( $ (O)!0.1!(P) $ );
  \fill[rounded corners=8pt,line width=1pt,green!30]
    (O|-P) -- 
    (aux|-P) --
    ([yshift=20pt]aux|-P) --
    ([yshift=20pt,xshift=4cm]aux|-P) --
    ([xshift=4cm]aux|-P) -- 
    (P) {[sharp corners] --
    ([yshift=-6pt]P) -- 
    ([yshift=-6pt]O|-P) } -- cycle;
  \draw[rounded corners=8pt,line width=1pt,green]
    (O|-P) -- 
    (aux|-P) --
    ([yshift=20pt]aux|-P) --
    ([yshift=20pt,xshift=4cm]aux|-P) --
    ([xshift=4cm]aux|-P) -- 
    (P) --\SetupExSheets{headings=margin}
    (P|-O) --
    (O) -- cycle;
  \node at ([xshift=2cm,yshift=6pt]aux|-P)
    {\refstepcounter{myPlanCounter}\sffamily\large 计划阶段~\themyPlanCounter} ; 
  },
firstextra={
  \coordinate (aux) at ( $ (O)!0.5!(P|-O) $ );
  \fill[rounded corners=8pt,line width=1pt,gray!30,overlay]
    (O|-P) -- 
    (aux|-P) --
    ([yshift=20pt]aux|-P) --
    ([yshift=20pt,xshift=4cm]aux|-P) --
    ([xshift=4cm]aux|-P) -- 
    (P) {[sharp corners] --
    ([yshift=-6pt]P) -- 
    ([yshift=-6pt]O|-P) } -- cycle;
  \draw[rounded corners=8pt,line width=1pt,gray,overlay]
    (O) --
    (O|-P) -- 
    (aux|-P) --
    ([yshift=20pt]aux|-P) --
    ([yshift=20pt,xshift=4cm]aux|-P) --
    ([xshift=4cm]aux|-P) -- 
    (P) --
    (P|-O);
  \node[overlay] at ([xshift=2cm,yshift=6pt]aux|-P)
    {\refstepcounter{myPlanCounter}\sffamily\large 计划阶段~\themyPlanCounter} ; 
  },
middleextra={
  \draw[rounded corners=8pt,line width=1pt,gray,overlay]
    (O|-P) -- 
    (O);
  \draw[rounded corners=8pt,line width=1pt,gray,overlay]
    (P) -- 
    (P|-O);
  },
secondextra={
  \coordinate (aux) at ( $ (O)!0.5!(P|-O) $ );
  \draw[rounded corners=8pt,line width=1pt,gray,overlay]
    (O|-P) -- 
    (O) --
    (P|-O) --
    (P);
  },
]{myPlan}


%%%%%%%%%%%%%%%%%%%%%%%%%%%%%%%%%%%%%%%%%%%%%%%%%%%%%%%%%%%%%
%% mySimpExpand
%%%%%%%%%%%%%%%%%%%%%%%%%%%%%%%%%%%%%%%%%%%%%%%%%%%%%%%%%%%%%

%% 2016-12-22继续扩展上面的环境,打算做一个词汇积累类型的新环境,这个目前打算用来做英语的扩展词汇用,最终目标
%也是替换掉上自己写的简单的myexpand环境
%% FROM - http://tex.stackexchange.com/questions/135871/what-are-the-relative-strong-and-weak-points-between-tcolorbox-and-mdframed
%\tcbset{colback=Salmon!50!white,colframe=FireBrick!75!black, width=(\linewidth-8mm)/2,before=,after=\hfill,equal height group=ske}
\newcounter{mySimpExpandCnt}
\newenvironment{mySimpExpand}[1][]
  {\refstepcounter{mySimpExpandCnt}\begin{tcolorbox}[
    colback=green!5,
    colframe=green!40!black,
    title=\large{扩展词汇-\themySimpExpandCnt : #1}]
  }
  {\end{tcolorbox}}

%% ReadIt(大家说英语节目的内容整理)
\newmdenv[
hidealllines=true,
innertopmargin=16pt,
innerbottommargin=10pt,
font=\sffamily\large,
leftmargin=0.0cm,
rightmargin=0.0cm,
skipabove=35pt,
skipbelow=15pt,
singleextra={
  \coordinate (aux) at ( $ (O)!0.1!(P) $ );
  \fill[rounded corners=8pt,line width=1pt,gray!10]
    (O|-P) -- 
    (aux|-P) --
    ([yshift=20pt]aux|-P) --
    ([yshift=20pt,xshift=4cm]aux|-P) --
    ([xshift=4cm]aux|-P) -- 
    (P) {[sharp corners] --
    ([yshift=-6pt]P) -- 
    ([yshift=-6pt]O|-P) } -- cycle;
  \draw[rounded corners=8pt,line width=1pt,gray]
    (O|-P) -- 
    (aux|-P) --
    ([yshift=20pt]aux|-P) --
    ([yshift=20pt,xshift=4cm]aux|-P) --
    ([xshift=4cm]aux|-P) -- 
    (P) --
    (P|-O) --
    (O) -- cycle;
  \node at ([xshift=2cm,yshift=6pt]aux|-P)
    {\sffamily Read It} ; 
  },
firstextra={
  \coordinate (aux) at ( $ (O)!0.5!(P|-O) $ );
  \fill[rounded corners=8pt,line width=1pt,gray!10,overlay]
    (O|-P) -- 
    (aux|-P) --
    ([yshift=20pt]aux|-P) --
    ([yshift=20pt,xshift=4cm]aux|-P) --
    ([xshift=4cm]aux|-P) -- 
    (P) {[sharp corners] --
    ([yshift=-6pt]P) -- 
    ([yshift=-6pt]O|-P) } -- cycle;
  \draw[rounded corners=8pt,line width=1pt,gray,overlay]
    (O) --
    (O|-P) -- 
    (aux|-P) --
    ([yshift=20pt]aux|-P) --
    ([yshift=20pt,xshift=4cm]aux|-P) --
    ([xshift=4cm]aux|-P) -- 
    (P) --
    (P|-O);
  \node[overlay] at ([xshift=2cm,yshift=6pt]aux|-P)
    {\sffamily Read It} ; 
  },
middleextra={
  \draw[rounded corners=8pt,line width=1pt,gray,overlay]
    (O|-P) -- 
    (O);
  \draw[rounded corners=8pt,line width=1pt,gray,overlay]
    (P) -- 
    (P|-O);
  },
secondextra={
  \coordinate (aux) at ( $ (O)!0.5!(P|-O) $ );
  \draw[rounded corners=8pt,line width=1pt,gray,overlay]
    (O|-P) -- 
    (O) --
    (P|-O) --
    (P);
  },
]{myReadList}

%% for keywords
\newmdenv[
hidealllines=true,
innertopmargin=14pt,
innerbottommargin=8pt,
font=\sffamily\footnotesize,
leftmargin=0.0cm,
rightmargin=0.0cm,
skipabove=34pt,
skipbelow=13pt,
singleextra={
  \coordinate (aux) at ( $ (O)!0.5!(P) $ );
  \fill[rounded corners=4pt,line width=1pt,gray!10]
    (O|-P) -- 
    (aux|-P) --
    ([yshift=15pt]aux|-P) --
    ([yshift=15pt,xshift=3cm]aux|-P) --
    ([xshift=3cm]aux|-P) -- 
    (P) {[sharp corners] --
    ([yshift=-6pt]P) -- 
    ([yshift=-6pt]O|-P) } -- cycle;
  \draw[rounded corners=8pt,line width=1pt,gray]
    (O|-P) -- 
    (aux|-P) --
    ([yshift=15pt]aux|-P) --
    ([yshift=15pt,xshift=3cm]aux|-P) --
     ([xshift=3cm]aux|-P) -- 
    (P) --
    (P|-O) --
    (O) -- cycle;
  \node at ([xshift=2cm,yshift=6pt]aux|-P)
    {\sffamily Keywords} ; 
  },
firstextra={
  \coordinate (aux) at ( $ (O)!0.5!(P|-O) $ );
  \fill[rounded corners=8pt,line width=1pt,gray!10,overlay]
    (O|-P) -- 
    (aux|-P) --
    ([yshift=15pt]aux|-P) --
    ([yshift=15pt,xshift=3cm]aux|-P) --
    ([xshift=3cm]aux|-P) -- 
    (P) {[sharp corners] --
    ([yshift=-6pt]P) -- 
    ([yshift=-6pt]O|-P) } -- cycle;
  \draw[rounded corners=8pt,line width=1pt,gray,overlay]
    (O) --
    (O|-P) -- 
    (aux|-P) --
    ([yshift=15pt]aux|-P) --
    ([yshift=15pt,xshift=3cm]aux|-P) --
    ([xshift=3cm]aux|-P) -- 
    (P) --
    (P|-O);
  \node[overlay] at ([xshift=2cm,yshift=6pt]aux|-P)
    {\sffamily Keywords} ; 
  },
middleextra={
  \draw[rounded corners=8pt,line width=1pt,gray,overlay]
    (O|-P) -- 
    (O);
  \draw[rounded corners=8pt,line width=1pt,gray,overlay]
    (P) -- 
    (P|-O);
  },
secondextra={
  \coordinate (aux) at ( $ (O)!0.5!(P|-O) $ );
  \draw[rounded corners=8pt,line width=1pt,gray,overlay]
    (O|-P) -- 
    (O) --
    (P|-O) --
    (P);
  },
]{myWordList}


%% 下面这个适合定理的风格

%% FROM - http://tex.stackexchange.com/questions/135871/what-are-the-relative-strong-and-weak-points-between-tcolorbox-and-mdframed
%\newcounter{mySimpExpandCounter}
%\tcbuselibrary{theorem}
%\newtcbtheorem[number within=section]{mySimpExpand22222}{My Theorem}%
%{colback=green!5,colframe=green!35!black,fonttitle=\bfseries}{th}



%\titleformat{\section}[block]{\kai}{\thesection}{10pt}{}

%\makeglossaries

\makeindex

\begin{document}

\maketitle

\tableofcontents

\begin{abstract}
由于使用\XeLaTeX{}格式已经存在一个乱序的单词词汇表格, 但现实中一个个地背诵这些孤立的单词, 对于提高词汇量的帮助并不太大, 而且也容易疲劳。

所以从2017年10月开始,单独使用这个文档,把整个词汇尽量按照分类来整理,用于日常单词记忆的主要手段。而总词汇表(vol.tex)则用于判断词汇量的参考文档使用。
\end{abstract}

%%% start new page from the first
\newpage

\section{身体部位}\label{sec.JuniorEng}

\vspace{1cm}
\begin{center}\index{hair} \index{ear} \index{leg} \index{mouth}  \index{nose} \index{face} \index{arm} \index{body} \index{toe} \index{wrong} \index{curly} \index{address} \index{Danmark} \index{ink} \index{attention} \index{pay attention} \index{comic} \index{comic book} \index{athletics} \index{athletic} \index{athlete} \index{wonder} \index{wonderful} \index{wrist} \index{middle} \index{central} \index{entrance} \index{ladder} \index{lonely} \index{alone} \index{set up} \index{agree with} \index{Paris} \index{winner} \index{nap} \index{Indian} \index{India} \index{race track} \index{blanket} \index{towel}
 \begin{tikzpicture}
 
\end{tikzpicture}

	\vspace{3cm}\index{hand} \index{eye} \index{foot} \index{head} \index{tail} \index{neck}  \index{blind} \index{internet} \index{large} \index{worker} \index{term} \index{straight} \index{street} \index{grey} \index{worse} \index{square} \index{expensive} \index{cheap} \index{purse} \index{bus station} \index{guitar} \index{granny} \index{fruit stall} \index{whisper} \index{stair} \index{stairs} \index{upstairs} \index{downstairs} \index{lemon} \index{site} \index{jam} \index{treasure} \index{cave} \index{repair} \index{kick} \index{tidy} \index{glamorous} \index{valley} \index{cliff} \index{wave} \index{burn} \index{bomb}

\begin{tikzpicture}
 
\end{tikzpicture}
\end{center}

\begin{printdemosample}
 - Look at the giraffe.
 
 - Wow! It has a long neck!
\end{printdemosample}


\vspace{1cm}


\section{颜色}

\begin{tikzpicture}
 
  \node [rectangle, anchor = west, text width=2cm] (ID1) {green};
  \node [rectangle, draw, right of = ID1, xshift=0.5cm, text width=1cm, minimum height=3ex]{};
  
  \node [rectangle, anchor = west, text width=2cm, below of = ID1] (ID2) {pink};
  \node [rectangle, draw, anchor = west, right of = ID2, xshift=0.5cm, text width=1cm, minimum height=3ex]{};
  
  \node [rectangle, anchor = west, right of = ID1, xshift=4cm, text width=2cm] (ID3) {yellow};
  \node [rectangle, draw, right of = ID3, xshift=0.5cm, text width=1cm, minimum height=3ex]{};
  
  \node [rectangle, anchor = west, text width=2cm, below of = ID3] (ID4) {red};
  \node [rectangle, draw, anchor = west, right of = ID4, xshift=0.5cm, text width=1cm, minimum height=3ex]{};
  
  \node [rectangle, anchor = west, text width=2cm, below of = ID2] (ID5) {blue};
  \node [rectangle, draw, right of = ID5, xshift=0.5cm, text width=1cm, minimum height=3ex]{};
  
  \node [rectangle, anchor = west, text width=2cm, below of = ID4] (ID6) {purple};
  \node [rectangle, draw, right of = ID6, xshift=0.5cm, text width=1cm, minimum height=3ex]{};
  
  \node [rectangle, anchor = west, text width=2cm, below of = ID5] (ID7) {white};
  \node [rectangle, draw, right of = ID7, xshift=0.5cm, text width=1cm, minimum height=3ex]{};
  
  \node [rectangle, anchor = west, text width=2cm, below of = ID6] (ID8) {brown};
  \node [rectangle, draw, right of = ID8, xshift=0.5cm, text width=1cm, minimum height=3ex]{};
  
  \node [rectangle, anchor = west, text width=2cm, below of = ID7] (ID9) {black};
  \node [rectangle, draw, right of = ID9, xshift=0.5cm, text width=1cm, minimum height=3ex]{};
  
  \node [rectangle, anchor = west, text width=2cm, below of = ID8] (ID10) {orange};
  \node [rectangle, draw, right of = ID10, xshift=0.5cm, text width=1cm, minimum height=3ex]{};
  
\end{tikzpicture}

\begin{printdemosample}  \index{with} \index{block} \index{build} \index{suitcase} \index{armchair} \index{hear} \index{noise} \index{both} \index{university} \index{universe} \index{in danger} \index{danger} \index{dangerous} \index{grasshopper} \index{grassland} \index{die} \index{enjoy} \index{carry} \index{restaurant} \index{Washington} \index{medicine} \index{grandparent} \index{lie} \index{exercise} \index{happen} \index{come round} \index{exactly} \index{fortnight} \index{appear} \index{appearance} \index{disappear} \index{sightsee}\index{sightseeing} \index{harbour} \index{crew} \index{cabin crew} \index{set off} \index{route} \index{expedition} \index{goods} \index{silk} \index{herb} \index{spice}
 First, they paint with red and orange. Then they paint with yellow and green.
\end{printdemosample}


\vspace{0.7cm}

由颜色可以衍生的词汇:\index{black} \index{pink}  \index{red}  \index{green} \index{blue} \index{colour} \index{color} \index{colourful} \index{white} \index{brown} \index{yellow} \index{purple} \index{paintbrush} \index{draw} \index{paint} \index{brush} \index{prepare} \index{ourselves} \index{heart} \index{race} \index{stamp} \index{shampoo} \index{team} \index{haircut} \index{secret} \index{dream}\index{kitten} \index{seaside} \index{group} \index{double} \index{portion} \index{alcohol} \index{alcoholic} \index{carbon} \index{carbohydrate} \index{unfortunately} \index{nervous} \index{miserable}

\begin{itemize}
 \item 五颜六色的   $\;$ $\;$ \quad colorful (colourful) \qquad colored
 \item 画画  $\;$ $\;$ $\;$ $\;$ \quad  $\;$ $\;$ $\;$ $\;$  draw
 \item 涂色 $\;$ $\;$ $\;$ $\;$  \quad $\;$ $\;$  $\;$ $\;$ paint
 \item 画刷 $\;$ $\;$ $\;$ $\;$  \quad $\;$ $\;$  $\;$ $\;$ paintbrush
\end{itemize}

\begin{printdemosample}
 She gives Lucy a paintbrush.
\end{printdemosample}

\begin{printdemosample} \index{cute}  \index{voice} \index{donkey}
 I have a rabbit, it's white. It's cute.
\end{printdemosample}


\newpage

\section{be}
be是一个概括性的词,它包括了英文中表达"是"这个意思的单词总类:  \index{am}  \index{are} \index{is} \index{there are}  \index{there is}  \index{isn't} \index{of course} \index{TV} \index{watch TV} \index{from} \index{exhibitor} \index{below} \index{part} \index{create} \index{direction} \index{toward} \index{manners} \index{disappointed} \index{disappointing} \index{bright} \index{brightness} \index{tulip} \index{rose} \index{towel} \index{wheat} \index{straight away} \index{grill} \index{grilled} \index{join} \index{realistic}

\begin{tikzpicture}
 \node [rectangle] (IDbase) {be};
 
 \node [rectangle, anchor = west, right of = IDbase, xshift=0.6cm, text width=0.7cm] (IDam) {am};
 \node [rectangle, anchor = west, above of = IDam, yshift=-0.3cm, text width=0.7cm ] (IDis) {is};
 \node [rectangle, anchor = west, below of = IDam, yshift=0.3cm, text width=0.7cm] (IDare) {are};
 
 \draw [->, >=stealth'] (IDbase) -- (IDam);
 \draw [->, >=stealth'] (IDbase) -- (IDis);
 \draw[->, >=stealth'] (IDbase) -- (IDare);
 
 \node [rectangle, anchor = west, text width=3.3cm, right of = IDbase, xshift=5cm, yshift=0.8cm, font=\scriptsize] (IDguize) {简单记得下面规则:};
 \node [rectangle, anchor = west, text width=3.3cm, below of = IDguize,xshift=0.5cm, yshift=0.5cm, font=\scriptsize] (ID1) {am只与我(I)用};
 \node [rectangle, anchor = west, text width=3.3cm, below of = ID1, yshift=0.5cm, font=\scriptsize] (ID2) {它他她this/that, 用is};
 \node [rectangle, anchor = west, text width=3.3cm, below of = ID2, yshift=0.5cm, font=\scriptsize] (ID3) {剩下的都用are};
 
 \draw [] ($(ID1.west)+ (0, 0.2)$) -- ($(ID1.west) + (0, -1.2)$);
\end{tikzpicture}

由于这个表达"是"的词使用太广泛了,所以我们在介绍用法的时候,用be来代表这些词的总体。

比如根据\cite{bibEngJuniorOneB}里面表达人高兴开心的说法,我们就可以使用: \index{I} \index{we} \index{plastic} \index{they} \index{though} \index{although} \index{congratulations} \index{dark} \index{rhyme} \index{above} \index{sky} \index{is gone} \index{Japan} \index{public} \index{public signs} \index{in public} \index{size} \index{ago} \index{themselves} \index{fossil} \index{gray} \index{ice} \index{change} \index{money}\index{forty} \index{forty-seven} \index{thirty} \index{thirty-two} \index{earth} \index{on earth} \index{able} \index{be able to} \index{divide} \index{division} \index{stick} \index{squirrel} \index{character} \index{narrator} \index{wrestler} \index{security} \index{invisible} \index{pretend to}

\vspace{0.4cm}

  \hspace{3cm} be happy
  
\vspace{0.3cm}

\marginpar{\vspace{0.8cm}你能够填对么?}
\begin{tasks}(2)
  \task They \blank[width=0.4cm]{} happy.  \task I \blank[width=0.4cm]{} happy.
  \task He \blank[width=0.4cm]{} happy.  \task She \blank[width=0.4cm]{} happy.
  \task We \blank[width=0.4cm]{} happy.  \task You \blank[width=0.4cm]{} happy.
\end{tasks}

\vspace{0.5cm}

再比如:

\vspace{0.4cm}

  \hspace{3cm} be made of
  
\vspace{0.3cm}

\begin{printdemosample}
 These toys are made of plastic.
 
 My table is made of wood.
\end{printdemosample}

\begin{tasks}(2)
  \task My chair \blank[width=0.4cm]{} made of wood.  \task My blocks \blank[width=0.4cm]{} made of wood.
\end{tasks}

\vspace{0.5cm}

某个地方有某个/某些东西:

\vspace{0.4cm}

  \hspace{3cm} there be
  
\vspace{0.3cm}

最近原则

\begin{printdemosample}
 There's a lamp next to the bed.
 
 There are three people in my family.
 
 There is a pen and two books on the desk.
 
 There are two books, a pen, and many pencils on the table.
\end{printdemosample}

\begin{printdemosample}
 Lucy finds a big box. She opens it. Inside, there is a dragon and some paints.
\end{printdemosample}

不是there be的情况下,看主语(谁)是几个人来决定用is/are:  \index{here} \index{there} \index{in order to} \index{order} \index{late} \index{loud} \index{later} \index{famous} \index{naughty} \index{second} \index{third} \index{fourth} \index{fifth} \index{country} \index{world} \index{worldwide} \index{sports centre} \index{sports center} \index{shoulder} \index{as soon as} \index{beard} \index{moustache} \index{sunglasses} \index{gather} \index{take care of} \index{have to} \index{has to} \index{had to}

\begin{printdemosample}
 Tom and Tim are brothers.
\end{printdemosample}


\vspace{5.1cm}
家庭成员的单词:\index{sister} \index{brother}  \index{aunt}  \index{uncle} \index{cousin} \index{niece} \index{nephew} \index{bride} \index{groom} \index{relative} \index{couple} \index{engagement} \index{engaged} \index{garden} \index{family} \index{daughter} \index{father} \index{mother} \index{friend} \index{grandfather} \index{grandmother}  \index{baby} \index{babysitter} \index{certainly} \index{tooth fairy} \index{fair} \index{fairly} \index{unfair} \index{moon} \index{wake up} \index{be sleepy} \index{asleep} \index{be asleep} \index{be awake} \index{magic} \index{pillow} \index{pill} \index{in a hurry} \index{husband}\index{band} \index{lose} \index{corner} \index{corn} \index{solve} \index{capital}\index{capital city} \index{since} \index{P.E.} \index{side} \index{ground} \index{accommodation}  \index{label} \index{biography} \index{blogger} \index{advert} \index{advertisement} \index{exist} \index{embarrassed} \index{prison} \index{prisoner} \index{jail} \index{punish} \index{chat show}

\begin{tasks}[counter-format=tsk[1].,label-offset=1em, label-align=right](4)
 \task father  \task mother \task grandfather \task grandmother
 \task brother \task sister  \task cousin    \task aunt
 \task uncle   \task family  \task dad       \task mum
 \task grandpa \task grandma \task friend    \task baby 
\end{tasks}

由家庭成员联想到人称的指代单词: \index{you}   \index{she}  \index{he} \index{Mr Chan} \index{Miss Thin} \index{Mrs Ship}  \index{it}  \index{shall} \index{comfortable} \index{percent} \index{leaf} \index{again} \index{stranger} \index{soccer} \index{mask} \index{circle} \index{hang} \index{hang up} \index{sour} \index{salt} \index{salty} \index{in total} \index{content}\index{tent} \index{contest} \index{totally} \index{England} \index{degree} \index{notice} \index{already} \index{neither} \index{neither nor} \index{sheet}

\begin{tasks}[counter-format=tsk[1].,label-offset=1em, label-align=right](4)
 \task I  \task you \task he \task she
 \task it \task we  \task they \task Mr
 \task Miss \task Mrs 
\end{tasks}

下一组:\index{your} \index{my} \index{its} \index{our} \index{their}  \index{his} \index{her} \index{on time} \index{sweep} \index{messy} \index{best} \index{the best} \index{finish line} \index{line} \index{reach} \index{reach up} \index{trunk} \index{agree} \index{disagree} \index{dislike} \index{spend} \index{spent} \index{special} \index{flat} \index{control} \index{replied} \index{said} \index{began} \index{brought} \index{bought}\index{taught} \index{none}

\begin{tasks}[counter-format=tsk[1].,label-offset=1em, label-align=right](4)
 \task my  \task your \task his \task her
 \task its \task our  \task their  \task 's
\end{tasks}

\begin{printdemosample}
 That's my book, and this is your pen.
\end{printdemosample}

\begin{printdemosample} \index{shop} \index{person} \index{or} \index{phone number} \index{throw away} \index{pleased} \index{handwriting} \index{television} \index{outside} \index{box office}

 This is Mr Green.
 
 This is Miss Li.
 
 This is Mrs Ship.
 
 This is Ron Rabbit's shop.
\end{printdemosample}

既然有家庭成员了,那么居住的房子称呼也要有:\index{bathroom} \index{house} \index{kitchen}  \index{bedroom} \index{room} \index{sitting room}  \index{lamp} \index{gate} \index{door} \index{wall} \index{dining room} \index{bed} \index{bad} \index{balcony} \index{basement} \index{connect}


\begin{tasks}[counter-format=tsk[1].,label-offset=1em, label-align=right](4)
 \task bathroom  \task house \task kitchen \task bedroom
 \task room \task sitting room  \task  lamp \task door
 \task wall \task dining room  \task bed
\end{tasks}

\begin{printdemosample}\marginpar{bed\\ bad \\ sad}
 - I got two tickets this morning.
 
 - Oh, that's too bad!
\end{printdemosample}


\vspace{0.4cm}

介绍家中东西位置,我们需要下面词汇:  \index{in} \index{behind} \index{in front of} \index{under}  \index{on} \index{next} \index{between} \index{inside} \index{holiday} \index{tie} \index{belt} \index{high heels} \index{bracelet} \index{earing} \index{handkerchief} \index{plain}\index{earring} \index{handsome} \index{shy} \index{drama} \index{fiction} \index{non-fiction} \index{thrill} \index{thriller} \index{journalist}

%% hand drawing

\vspace{5cm}  

\begin{tasks}[counter-format=tsk[1].,label-offset=1em, label-align=right](4)
 \task in    \task under     \task behind    \task in front of
 \task on    \task next to   \task between   \task inside
 \task outside
\end{tasks}

\newpage

\section{字段相同的词} \index{test} \index{rope} \index{skipping rope} \index{skip} \index{install} \index{installation} \index{webcam} \index{angrily} \index{frightened} \index{frightening} \index{relax} \index{relaxed} \index{download} \index{crowd} \index{crowed}
  
\begin{tikzpicture}
  \coordinate  (p1) at (0, 1);
  \coordinate  (p2) at (0, -1); \coordinate  (p3) at (5, 1); \coordinate  (p4) at (5, -1);
  \coordinate (pcenter) at (2, 0);
  
  \node at (p1) {w};
  \node at (p2) {b};
  \node at (0, 0) {p};
  \node at (pcenter) {ear};
  
  \node at (p3) {wear};
  \node at (5, 0) {pear};
  \node at (p4) {bear};
\end{tikzpicture}

\begin{printdemosample} \index{hat} \index{wear} \index{pear} \index{calendar}
 I can wear my hat.
\end{printdemosample}

\begin{printdemosample} \index{short}
 - Look at the bear.
 
 - Oh, it has a short tail.
\end{printdemosample}


\begin{tikzpicture}
  \coordinate  (p1) at (0, 1);
  \coordinate  (p2) at (0.1, -1); \coordinate  (p3) at (5, 1); \coordinate  (p4) at (5, -1);
  \coordinate (pcenter) at (2, 0);
  
  \node at (p1) {h};
  \node at (p2) {ch};
  \node at (pcenter) {air};
  
  \node at (p3) {hair};
  \node at (p4) {chair};
\end{tikzpicture}

\begin{printdemosample}
 There's a desk and a chair in my room.

 My chair is made of wood.
\end{printdemosample}

\begin{printdemosample} \marginpar{hair不可数,不可用a}
 - That is my cousin Maria, her hair is long.
 
 - Yes, she has long hair.
\end{printdemosample}

注意头发通常我们不说a long hair, 其它的则可以使用a, 而对称的部位则用复数即可:\index{see}  \index{last} \index{professor} \index{steal} \index{chain} \index{goalkeeper}

\begin{printdemosample}  \index{to} \index{will} \index{help} \index{border}
 This is a baby shark. It has eyes to help it see. 
 
 It has a tail to help it swim.
 
 It has a mouth to help it eat.
 
 It will grow into a big shark.
\end{printdemosample}

\newpage

\marginpar{\vspace{0.7cm}take off\\ take it easy}
\begin{tikzpicture} \index{take} \index{make} \index{cake} \index{cupcake} \index{borrow} \index{lend} \index{stone} \index{text} \index{textbook} \index{instruction}
  \coordinate  (p1) at (0, 1);
  \coordinate  (p2) at (0, 0);
  \coordinate  (p3) at (0, -1);
  \coordinate  (p4) at (5, 1);
  \coordinate  (p5) at (5, 0);
  \coordinate  (p6) at (5, -1);
  \coordinate (pcenter) at (2, 0);
  
  \node at (p1) {t};
  \node at (p2) {m};
  \node at (p3) {c};
  
  \node at (pcenter) {ake};
  
  \node at (p4) {take};
  \node at (p5) {make};
  \node at (p6) {cake};
\end{tikzpicture}

\begin{printdemosample}
 - This cake is cold.
 
 - This pizza is hot.
\end{printdemosample}

\marginpar{\vspace{0.7cm}sometimes}
\begin{tikzpicture}
  \coordinate  (p1) at (0, 1);
  \coordinate  (p2) at (0, -1); \coordinate  (p3) at (5, 1); \coordinate  (p4) at (5, -1);
  \coordinate (pcenter) at (2, 0);
  
  \node at (p1) {s};
  \node at (p2) {c};
  \node at (pcenter) {ome};
  
  \node at (p3) {some};
  \node at (p4) {come};
\end{tikzpicture}

\begin{printdemosample}
 Some people go by train, some people go by bicycle.
\end{printdemosample}


\begin{tikzpicture}
  \coordinate  (p1) at (0, 1);
  \coordinate  (p2) at (0, -1); \coordinate  (p3) at (5, 1); \coordinate  (p4) at (5, -1);
  \coordinate (pcenter) at (2, 0);
  
  \node at (p1) {c};
  \node at (p2) {p};
  \node at (pcenter) {arrot};
  
  \node at (p3) {carrot};
  \node at (p4) {parrot};
\end{tikzpicture}

\begin{printdemosample}
 He talks like a parrot and just repeat what he heard. 
\end{printdemosample}


\vspace{0.8cm} \index{walk}

\marginpar{\vspace{0.7cm}talk about}
\begin{tikzpicture}
  \coordinate  (p1) at (0, 1);
  \coordinate  (p2) at (0, -1); \coordinate  (p3) at (5, 1); \coordinate  (p4) at (5, -1);
  \coordinate (pcenter) at (2, 0);
  
  \node at (p1) {w};
  \node at (p2) {t};
  \node at (pcenter) {alk};
  
  \node at (p3) {walk};
  \node at (p4) {talk};
\end{tikzpicture}

\begin{printdemosample}
 Stop talking and listen!
\end{printdemosample}

\begin{printdemosample} \index{about}
 Let's talk about this room.
 
 Let's talk in English.
\end{printdemosample}


\newpage

\marginpar{\vspace{0.7cm}look at \\ look for \\look out \\look around}
\begin{tikzpicture} \index{look} \index{look at}  \index{look for} \index{look out} \index{take a look} \index{have a look}\index{have a swim} \index{cook} \index{mail} \index{beginner}
  \coordinate  (p1) at (0, 1);
  \coordinate  (p2) at (0, 0);
  \coordinate  (p3) at (0, -1);
  \coordinate  (p4) at (5, 1);
  \coordinate  (p5) at (5, 0);
  \coordinate  (p6) at (5, -1);
  \coordinate (pcenter) at (2, 0);
  
  \node at (p1) {c};
  \node at (p2) {b};
  \node at (p3) {l};
  
  \node at (pcenter) {ook};
  
  \node at (p4) {cook};
  \node at (p5) {book};
  \node at (p6) {look};
\end{tikzpicture}

\begin{printdemosample} \index{book} \index{dancing}
 I can cook. I can sing. I can dance.
\end{printdemosample}

\begin{printdemosample} \index{doctor} \index{dad} \index{supermarket} \index{floor} \index{missing}
 - My dad is a cook.
 
 - My dad is a doctor.
\end{printdemosample}

\begin{printdemosample}
 John is a very good cook. \qquad (He cooks well)
\end{printdemosample}

由cook联想出职位单词: \index{dancer} \index{singer} \index{each}  \index{clown} \index{teacher} \index{farmer} \index{nurse} \index{policeman} \index{police car} \index{police officer} \index{policewoman} \index{postman} \marginpar{\vspace{1.2cm}源自铜扣(copper\begin{IPA}[kA:p@r]\end{IPA}) }

\begin{tasks}[counter-format=tsk[1].,label-offset=1em, label-align=right](4)
 \task teacher    \task doctor        \task nurse  \task dancer
 \task policeman  \task policewoman   \task postman \task cop \begin{IPA}[kA:p]\end{IPA}
 \task singer     \task police officer \task farmer \task worker
\end{tasks}

\vspace{0.8cm}

\begin{tikzpicture} \index{wood} \index{wooden} \index{blood}\index{flood} \index{consider} \index{formal} \index{sugar} \index{thumb} \index{warn} \index{suggest} \index{metal} \index{attractive} \index{attraction} \index{gorgeous} \index{arrest} \index{attack} \index{crime} \index{criminal} \index{ruin} \index{ruinous} \index{anniversary} \index{tower} \index{sculpture}
  \coordinate  (p1) at (0, 1);
  \coordinate  (p2) at (0, 0);
  \coordinate  (p3) at (0, -1);
  \coordinate  (p4) at (5, 1);
  \coordinate  (p5) at (5, 0);
  \coordinate  (p6) at (5, -1);
  \coordinate (pcenter) at (2, 0);
  
  \node at (p1) {f};
  \node at (p2) {w};
  \node at (p3) {bl};
  
  \node at (pcenter) {ood};
  
  \node at (p4) {food};
  \node at (p5) {wood};
  \node at (p6) {blood};
\end{tikzpicture}
  
  \marginpar{\vspace{0.9cm}中国菜,中餐}
  
 \begin{printdemosample}  \index{like} \index{meal} \index{pick} \index{pick up} \index{simple} \index{calculator} \index{kilo-} \index{kilometer} \index{kilometre} \index{China} \index{Chinese} \index{America}\index{Italy} \index{Italian} \index{Poland} \index{Polish} \index{table tennis} \index{volleyball} \index{be made of} \index{fire} \index{firefly} \index{fire engine} \index{building} \index{twin} \index{lovely} \index{English} \index{close} \index{snowman} \index{try} \index{admit} \index{lesson} \index{hill} \index{headache} \index{go climbing} \index{fever} \index{be afraid of}
  I like Chinese food.
 \end{printdemosample}
 
 \begin{printdemosample}
  My chair is made of wood. My table is made of wood too.
 \end{printdemosample}


\newpage

联想词 \index{bottle} \index{fill up} \index{water}  \index{fine} \marginpar{\vspace{1.3cm}尽可能想到的所有单词}

\begin{tikzpicture}
\end{tikzpicture}
  
  
\vspace{2cm} \index{cup} \index{high} \index{check} \index{check in} \index{check out} \index{immigration} \index{waste}\index{a waste of time} \index{mean}\index{album} \index{slip} \index{through} \index{skin} \index{protect} \index{climb up} \index{sharp} \index{magazine} \index{future} \index{brave} \index{player} \index{football player} \index{painting} \index{diet} \index{on diet} \index{barbecue} \index{chess} \index{delicious} \index{Easter} \index{patient} \index{mature} \index{bridge}\index{tear up} \index{scary} \index{calm} \index{calm down} \index{anxious} \index{serious}\index{seriously} \index{charming} \index{literature} \index{shocking} \index{romantic} \index{soldier} \index{skill} \index{guilty} \index{stress} \index{stressed} \index{confused} \index{confuse} \index{freeze} \index{froze} \index{frozen} \index{identity card} \index{check-in desk}

\begin{tasks}[counter-format=tsk[1].,label-offset=1em, label-align=right](4)
 \task bottle  \task empty  \task full     \task fill up
 \task water    \task wine   \task low     \task high
 \task cup     \task cupcake \task drink   \task eat
\end{tasks}

\begin{printdemosample}
 a bottle of milk.
 
 seven bottles of milk
\end{printdemosample}

\vspace{0.9cm}

由于满、空等,联想出一些反义词汇: \index{long} \index{tall} \index{short} \index{happy}  \index{unhappy} \index{small} \index{fat} \index{big} \index{thin} \index{little} \index{quick} \index{slow} \index{slowly} \index{slow down} \index{speed up} \index{speed} \index{keep} \index{move} \index{young} \index{spirit} \index{finger} \index{hobby} \index{telescope} \index{Canada} \index{finally} \index{final} \index{storybook} \index{at weekends} \index{Christmas} \index{Christmas tree} \index{message} \index{be good at} \index{knowledge} \index{ambition} \index{soft} \index{softly} \index{software} \index{understand} \index{traffic light} \index{traffic} \index{smell} \index{password} \index{hero} \index{heroine} \index{fairy} \index{fight} \index{firefighter} \index{unbelievable} \index{uncomfortable} \index{unforgettable} \index{unpleasant} \index{terrific} \index{magnificent}

\begin{tikzpicture}
\end{tikzpicture}

\begin{tasks}[counter-format=tsk[1].,label-offset=1em, label-align=right](4)
 \task fat  \task thin   \task tall   \task short
 \task black \task white \task big    \task small
 \task little \task long  \task fast  \task slow
 \task old    \task young  \task happy  \task sad
 \task up     \task down   \task in  \task out
 \task on     \task off   \task high   \task low
\end{tasks}


高频率词汇on/off: \index{off} \index{turn off}  \index{turn on}  \index{take off} \index{back off} \index{get on} \index{get off} \index{old} \index{new} \index{put on} \index{back} \index{Sunday} \index{Friday} \index{Monday} \index{sound} \index{festival} \index{listen to} \index{fast} \index{fasten} \index{knit}

\begin{printdemosample}
 Buzz off, Bill!
\end{printdemosample}

\marginpar{\vspace{1cm}put on \\ take off} \index{yes} \index{just} \index{top} \index{top ten} \index{a little} \index{January} \index{February} \index{March} \index{April} \index{May} \index{June} \index{July} \index{August} \index{September} \index{October} \index{November}  \index{power} \index{power on} \index{power off} \index{powerful} \index{architect} \index{cottage} \index{location}

\begin{printdemosample}
 - Put on your coat, Linling.
 
 - Yes, Mum.
\end{printdemosample}



\begin{tasks}[counter-format=tsk[1].,label-offset=1em, label-align=right](4)
 \task turn on  \task turn off  \task power on  \task power off
 \task get on     \task get off \task back off  \task take off
 \task put on
\end{tasks}

\begin{tikzpicture}[scale=0.7]
 \draw (1.2856,1.532) arc 
     [start angle = 50,
      end angle = 130,
      radius=2.0];
      
  \draw [ultra thick] (0,0) -- (130:2);
  \draw [] (0, 0) -- (50:2);
  
  \draw [->, >=stealth'] (0.7, 1.2) to [bend right=35] (-0.7, 1.2);
  
   \draw [->, >=stealth'] (-0.4, 0.6) to [bend left=35] (0.4, 0.6);
   
   \node at (0, 1.6) [font=\scriptsize]{turn on};
   \node at (0, 1.0) [font=\scriptsize]{turn off};
  
\end{tikzpicture}

单词turn具有转圈的意思:  \index{turn} \index{introduce} \index{important} \index{life} \index{news report} \index{shape} \index{different} \index{difference} \index{health} \index{healthy} \index{shine} \index{airport} \index{sign} \index{dollar} \index{pound} \index{difficult}\index{difficulty} \index{gram} \index{kilogram} \index{heavy} \index{shiny} \index{turkey}

\marginpar{\vspace{1cm}turn around}
\begin{printdemosample}
 - Your turn.
 
 - Yes, it's my turn!
\end{printdemosample}

\vspace{1cm}

另外两个高频词是up/down: \index{cut up} \index{cut} \index{shut} \index{shut up} \index{listen up} \index{get up} \index{sit down} \index{stand} \index{stand up} \index{sit} \index{seat} \index{so} \index{mustn't} \index{won't} \index{saw} \index{found} \index{got} \index{dessert} \index{desert} \index{past} \index{pasta} \index{lift} \index{half} \index{perfect} \index{early} \index{excellent} \index{put up} \index{take down} \index{shadow} \index{antique}

\begin{tasks}[counter-format=tsk[1].,label-offset=1em, label-align=right](4)
 \task get up  \task cut up  \task fill up  \task listen up
 \task stand up  \task sit down  \task take down  \task write down
 \task speed up  \task slow down  \task be fed up with
\end{tasks}

\begin{printdemosample}
 Cut up this fish!
 
 Fill up this cup!
\end{printdemosample}

\begin{printdemosample}  \index{job}  \index{clean up} \index{cleaner} \index{mind} \index{kind} \index{center} \index{centre} \index{than} \index{paw} \index{claw} \index{useful} \index{bite} \index{listener} \index{weak} \index{pity} \index{match} \index{sleep} \index{have a sleep} \index{sleepy} \index{camp} \index{waterproof} \index{bulletproof}
 I'm fed up with this job!
\end{printdemosample}

\begin{printdemosample}
 Let's clean up this room.
 
 Let's clean it up.
\end{printdemosample}


\newpage

\begin{tikzpicture}
\end{tikzpicture}
   
\vspace{2cm} \index{van} \index{boat} \index{car} \index{bike} \index{bicycle}  \index{boy} \index{girl}  \index{wheel} \index{river} \index{lake}   \index{man} \index{woman}  \index{after school} \index{classmate} \index{hundred} \index{thousand} \index{thousands of} \index{hundreds of} \index{marathon} \index{space} \index{spaceship} \index{pianist} \index{traveller} \index{U.K.} \index{chat}

\marginpar{\vspace{1cm}交通工具}
\marginpar{\vspace{1.5cm}使用工具的人} \index{taxi} \index{tax} \index{loan} \index{once} \index{at once} \index{number} \index{tea} \index{show} \index{over there} \index{report} \index{reporter} \index{shopper} \index{dial}
\begin{tasks}[counter-format=tsk[1].,label-offset=1em, label-align=right](4)
 \task car    \task boat   \task ship     \task train
 \task van    \task bike   \task bicycle  \task taxi 
 \task ferry  \task wheel  \task boy      \task girl
 \task man     \task woman  \task driver   \task lake
 \task river 
\end{tasks}

乘坐交通工具,一般用by:  \index{by} \index{by car} \index{by bus} \index{drive} \index{driver} \index{same} \index{similar} \index{light}\index{lightning} \index{hold} \index{hold on} \index{beside} \index{crayon} \index{problem} \index{each other} \index{present} \index{flavour} \index{sweet} \index{busy} \index{pepper} \index{caf\'{e}} \index{penguin} \index{jungle} \index{mountain} \index{come bottom in} \index{bottom} \index{come top in} \index{had} \index{dinosaur} \index{safe} \index{project} \index{poster} \index{Hong Kong} \index{firecracker} \index{hooray} \index{prince} \index{metro} \index{habit} \index{deep} \index{cheer} \index{cheerful} \index{Sydney} \index{astronaut} \index{gale}

\begin{itemize}
 \item by car
 \item by bus
 \item by train
 \item by bicycle
\end{itemize}

和by发音类似的buy, 以及反义词sell: \index{sell} \index{buy} \index{slippers} \index{sneakers} \index{boots} \index{sandals} \index{cupboard} \index{a cup of} \index{a glass of} \index{twice} \index{better} \index{quite} \index{ski} \index{skiing} \index{meter} \index{metre} \index{feel} \index{fee} \index{most} \index{list} \index{score} \index{crab} \index{octopus} \index{roar} \index{tap} \index{us} \index{art} \index{artist} \index{museum} \index{turn into} \index{honey} \index{sentence} \index{National Day} \index{fashion} \index{fashionable} \index{fashion model} \index{fashion show} \index{bald} \index{advanced}


\begin{printdemosample}
 Sam and his dad go to a bookstore, they buy a book.
\end{printdemosample}

\begin{printdemosample}
 This is Miss Thin's shop. Miss Thin sells eggs, nuts and carrots.
\end{printdemosample}

\vspace{0.2cm}

称赞的说法:

\begin{printdemosample} \index{amazing}  \index{great}  \index{nice} \index{awesome} \index{uniform} \index{plane} \index{airplane} \index{air} \index{weather} \index{drawer} \index{rubbish}\index{incredible} \index{marvellous} \index{increase} \index{decrease} \index{lady} \index{dentist} \index{snack} \index{library} \index{peach} \index{glove} \index{gloves} \index{idea} \index{fact} \index{ocean} \index{land} \index{shorts} \index{waterfall}\index{colored} \index{page} \index{nearby} \index{lion} \index{nut} \index{encourage} \index{expect} \index{avoid} \index{remind} \index{pain} \index{painful} \index{describe} \index{episode} \index{series} \index{fancy} \index{sore} \index{dull} \index{subject} \index{distance} \index{temple} \index{monument} \index{palace}
 That's great/nice.
 
 That's amazing.
 
 It's awesome.
\end{printdemosample}

\begin{printdemosample}
  I can cook.
  
  That's great.
\end{printdemosample}


  
  \newpage
  

\section{天气和四季} \index{rain} \index{rainy} \index{cloud} \index{cloudy} \index{sun} \index{sunny} \index{name}  \index{brain} \index{offer} \index{tractor} \index{truck} \index{shopping} \index{else} \index{scooter} \index{for} \index{snow} \index{snowy} \index{spill} \index{value} \index{valuable} \index{usable} \index{adjust} \index{adjustable} \index{reason} \index{reasonable} \index{sweat} \index{businessman} \index{sometime} \index{some times} \index{some time} \index{signal}

\begin{tikzpicture}
\end{tikzpicture}

\begin{tasks}(4)
 \task windy  \task rainy  \task sunny  \task cloudy
 \task wind   \task rain   \task sun    \task cloud
\end{tasks}


\begin{printdemosample}  \index{but} \index{dry} \index{wet} \index{season} \index{clever} \index{smart} \index{silly} \index{in fact} \index{tonight} \index{email} \index{surf} \index{way} \index{this way} \index{no way} \index{only} \index{believe} \index{motorbike} \index{motorway} \index{lorry} \index{helicopter} \index{fireman} \index{pilot} \index{sail} \index{sailor} \index{teapot} \index{pot} \index{spot} \index{fantastic} \index{volunteer} \index{medal} \index{shake} \index{raise} \index{passage} \index{politician} \index{poem} \index{poetry}

 When the rain comes, ants look for shelter. But ducks stay in the rain.
\end{printdemosample}

\vspace{0.5cm}
联想词:\index{train}  \index{by train}   \index{rainbow} \index{question} \index{answer} \index{finish} \index{December} \index{Saturday} \index{Wednesday} \index{Thursday} \index{parents} \index{parent} \index{either} \index{no problem} \index{foil jacket} \index{curtain} \index{webpage} \index{website} \index{sixth} \index{seventh} \index{look up} \index{joke} \index{also} \index{curiously} \index{traditional} \index{wedding}\index{fetch} \index{deliver} \index{culture} \index{departure} \index{department} \index{custom} \index{pour}

rainy \qquad rain  \qquad train  \qquad brain  \qquad rainbow

\vspace{1.8cm}

一年四季:\index{spring} \index{summer} \index{autumn}  \index{fall} \index{winter} \index{cold} \index{hot} \index{warm} \index{cool} \index{upset} \index{French} \index{rest} \index{raincoat} \index{costume} \index{arrive at} \index{necklace} \index{alarm} \index{worried} \index{friendly} \index{friendship} \index{lazy} \index{grandson} \index{granddaughter} \index{marry} \index{married} \index{marriage} \index{teenager} \index{comb} \index{hometown} \index{among} \index{writer} \index{keyboard} \index{mile} \index{positive} \index{elderly}

\begin{tasks}[counter-format=tsk[1].,label-offset=1em, label-align=right](4)
 \task spring  \task summer  \task autumn     \task winter
 \task warm    \task hot    \task cool  \task cold
\end{tasks}

秋天的两种说法:

autumn  \qquad  fall \qquad \qquad 

\begin{printdemosample}
 - It's cool.
 
 - Yes, it's autumn.
\end{printdemosample}

\index{these} \index{those} \index{forget} \index{forget to} \index{follow}  \index{chicken} \index{nature} \index{cabbage} \index{cucumber} \index{mushroom} \index{newspaper} \index{news} \index{learn} \index{student} \index{mug} \index{rucksack} \index{key} \index{wallet} \index{swan} \index{rainforest} \index{camel} \index{tight} \index{tights} \index{diary} \index{online} \index{right now}
\begin{printdemosample}\marginpar{these \\ those}
 These are apples, those are pumpkins.
\end{printdemosample}

 \newpage
 
一年四季展示了时间的变化,所以下面是时间相关单词: \index{morning} \index{evening} \index{night} \index{time}  \index{home}  \index{go home} \index{hour} \index{minute} \index{year} \index{day}  \index{then} \index{now}  \index{afternoon} \index{noon} \index{week} \index{weekend} \index{month} \index{breathe} \index{survive} \index{survivor} \index{movie} \index{stretch} \index{blonde} \index{nothing} \index{soft drink} \index{injure} \index{elbow} \index{ankle} \index{knee}

\marginpar{\vspace{1cm}no\\now\\then}
\begin{tasks}[counter-format=tsk[1].,label-offset=1em, label-align=right](4)
 \task morning  \task afternoon  \task evening \task noon 
 \task night    \task time     \task hour       \task minute 
 \task day      \task year     \task week       \task now
\end{tasks}

\marginpar{\vspace{0.7cm}after}
\begin{printdemosample}\index{after} \index{discuss}
 Good monring. Good afternoon. Good evening. Good night.
\end{printdemosample}


对于每一个时间单位,我们都可使用every来搭配。\index{every} \index{everyone} \index{could} \index{would} \index{should} \index{cross} \index{true} \index{false} \index{basket} \index{as usual} \index{bank} \index{banker} \index{suddenly} \index{celebration} \index{celebrate} \index{celebrity} \index{calculate} \index{absent} \index{fog} \index{foggy} \index{success}  \index{successful} \index{unsuccessful} \index{plan} \index{enough} \index{bored} \index{concert} \index{photograph} \index{bowl} \index{reply} \index{keeper} \index{fresh} \index{flag} \index{Thanksgiving} \index{while} \index{cage} \index{regular} \index{regularly} \index{promise} \index{lecture} \index{lecturer} \index{behave} \index{behaviour} \index{interview} \index{headline} \index{scene} \index{responsible} \index{rescue} \index{qualification} \index{announcement} \index{destination} \index{extinct} \index{worth} \index{take up} \index{cruel} \index{prescription} \index{salary} \index{cash} \index{coin}

\marginpar{\vspace{1cm}形容词}
\begin{printdemosample}
This is a suit for everyday wear.

I speak English every day.

Hello, everyone.
\end{printdemosample}

同时,需要注意区分 very  \qquad every  \qquad very much  \index{very much} \index{very} \index{visit} \index{visitor} \index{expo} \index{a pair of} \index{baker} \index{decide} \index{decide to}\index{decision} \index{leave} \index{trouble} \index{choose} \index{bat} \index{be careful} \index{carefully} \index{result} \index{mistake} \index{weigh} \index{weight} \index{smoke} \index{quickly} \index{boxer} \index{row} \index{a number of} \index{clear} \index{clearly} \index{unclear} \index{storm} \index{thunder} \index{thunderstorm} \index{flour} \index{dear} \index{private} \index{crash} \index{laptop} \index{bargain} \index{attic} \index{for ages} \index{funfair}

\begin{printdemosample}
 Thank you very much.
 
 It's very high.
\end{printdemosample}

而day也可以组合成多个词汇: \index{today} \index{birthday} \index{yesterday} \index{tomorrow} \index{dragon} \index{away} \index{at} \index{at home} \index{at school} \index{let} \index{kite} \index{together} \index{globe} \index{global} \index{zone} \index{obvious} \index{gold} \index{golden} \index{improve} \index{theatre} \index{champion}\index{championship} \index{at all} \index{luggage} \index{main} \index{low} \index{unless} \index{however} \index{whoever} \index{whenever} \index{wherever} \index{whatever} \index{look after} \index{hate} \index{recommend} \index{especially} \index{underground} \index{roundabout} \index{journey} \index{astronaut} \index{explore} \index{explorer} \index{prefer} \index{queue} \index{army} \index{equipment} \index{flee}

\marginpar{\vspace{1cm}Happy birthday!}
\begin{tasks}[counter-format=tsk[1].,label-offset=1em, label-align=right](4)
 \task today  \task yesterday  \task tomorrow  \task birthday
\end{tasks}

\marginpar{\vspace{0.9cm}all day long}
\begin{printdemosample}
 Dad is a singer. He sings all day long.
\end{printdemosample}


\vspace{2cm}

询问时间的方式: \index{o'clock} \index{and} \index{clock} \index{because} \index{toilet} \index{hotel} \index{hostel} \index{stadium} \index{castle} \index{station} \index{railway station} \index{bake} \index{baked} \index{oil} \index{cereal} \index{prize} \index{price} \index{flight} \index{passport} \index{challenge} \index{support} \index{advise} \index{allow} \index{advice} \index{towel} \index{boil} \index{boiled} \index{coach} \index{gym} \index{practise} \index{practice} \index{field} \index{maybe} \index{traffic jam} \index{normal} \index{program} \index{programme} \index{programmer} \index{complete} \index{bath} \index{instead} \index{usual} \index{by usual} \index{litter} \index{littering} \index{much} \index{activity} \index{information} \index{tyre} \index{flat tyres} \index{fix} \index{tourist} \index{tourism} \index{tour} \index{tour guide} \index{chef} \index{vehicle} \index{pavement} \index{sidewalk} \index{disease}

\marginpar{\vspace{0.8cm}可以回答该到做什么事情的时间了。}
\begin{printdemosample}
 - What's the time?
 
 - It's time to drink and eat.
\end{printdemosample}

\marginpar{\vspace{0.8cm}必需回答几点}
\begin{printdemosample}
 - What time is it?
 
 - It's seven o'clock. It's time for breakfast.
\end{printdemosample}

接下来注意It's time for sth. 和It's time to do sth.两种句式

Peppa Pig中猪妈妈的"Home time."相当于\blank[width=3cm]{}? \marginpar{It's time to go home.}

 \newpage
 
\section{动词}

  \subsection{基础} \index{jump} \index{jumper}  \index{run} \index{giant} \index{Santa Claus}\index{go} \index{come} \index{hop} \index{fly} \index{jog}  \index{reindeer} \index{indeed} \index{rather} \index{envelope} \index{receive} \index{reception} \index{sold} \index{Korea} \index{force} \index{air force}
\index{want} \index{want to} \index{sport} \index{more} \index{hungry} 
\index{thirsty} \index{hide} \index{catch} \index{throw} \index{picnic} 
\index{break} \index{break into} \index{take a break} \index{have a break} \index{tear} \index{suit} \index{suitable} \index{act} 
\index{action} \index{action film} \index{actor} \index{actress} \index{speech} \index{as well} 
\index{fail} \index{ice-cream} \index{iceberg} \index{scream} \index{opinion} \index{advice}  \index{reliable} \index{stage} \index{Belgium} \index{well-known} \index{phrase} \index{take place}
\index{coal}
  

\begin{tikzpicture}
\end{tikzpicture}

\begin{printdemosample} \index{go away} \index{south} \index{south east} \index{south west} \index{north} \index{east} \index{west} \index{western} \index{club} \index{model} \index{logo} \index{ever} \index{business} \index{certain} \index{fur} \index{furry} \index{gave} \index{probably} \index{detect} \index{detective} \index{hankie} \index{sometimes} \index{brake} \index{detail} \index{feather} \index{silently} \index{silent} \index{a can of} \index{fry} \index{fried} \index{grain} \index{french fries}
 Don't go away, we'll right be back.
\end{printdemosample}


 关于说话: \index{say} \index{call} \index{tell} \index{touch} \index{shout} \index{pie} \index{him} \index{them} \index{bleed} \index{fruit} \index{a piece of} \index{piece} \index{adventure} \index{teenager} \index{assist}\index{assistant} \index{college} \index{paragraph} \index{population} \index{environmental} \index{issue} \index{reduce} \index{shower} \index{triangle} \index{rectangle} \index{angle}
 
\begin{tasks}[counter-format=tsk[1].,label-offset=1em, label-align=right](4)
 \task say      \task talk      \task speak     \task tell
 \task shout    \task call
\end{tasks}
 
  其它杂项:\index{grow} \index{grow up} \index{has}  \index{have} \index{care} \index{sing} \index{stop} \index{find} \index{mix} \index{stay} \index{count} \index{play} \index{read} \index{listen} \index{write} \index{wrote} \index{think} \index{thought} \index{open}  \index{get} \index{know} \index{give} \index{give up} \index{ride} \index{ought to} \index{retire}
  
\begin{tasks}[counter-format=tsk[1].,label-offset=1em, label-align=right](4)
 \task grow   \task has  \task have   \task care
 \task sing   \task find \task stop   \task mix
 \task stay   \task count   \task play \task read
 \task write  \task think  \task listen \task open
 \task get    \task give  \task know  \task ride
\end{tasks} 


  
表达人的表情: smile \; angry \; laugh \; sad  \; surprised \; tired \; excited \index{smile}  \index{angry}   \index{laugh}  \index{sad} \index{surprised} \index{surprising} \index{tired} \index{excited} \index{exciting} \index{clothes} \index{lock} \index{unlock} \index{nightmare} \index{blow} \index{circus} \index{unicycle} \index{trolley} \index{Ferris wheel} \index{witch} \index{broom} \index{ugly} \index{lobster} \index{seashell} \index{knock} \index{chest} \index{wide} \index{guard} \index{possible} \index{impossible} \index{cycle} \index{into} \index{cube} \index{jet lag} \index{menu} \index{main course}

  \begin{tikzpicture}
 
\end{tikzpicture}

表明频率的词: \index{never} \index{always} \index{hardly} \index{often} \index{usually} \index{unusually} \index{worry} \index{forest} \index{vase} \index{broccoli} \index{roof} \index{opposite} \index{scared} \index{save} \index{accident} \index{poor} \index{rich} \index{law} \index{lawyer} \index{by the way} \index{date} \index{gun} \index{thief} \index{foreign} \index{foreigner} \index{stupid} \index{sunset} \index{during} \index{plenty of} \index{neighbour} \index{president} \index{polite} \index{several} \index{all the time} \index{level} \index{sea level} \index{record} \index{recording} \index{plateau} \index{homeland} \index{boarding pass} \index{duty-free shops} \index{compose} \index{composer} \index{conductor} \index{nightclub}

\begin{tasks}[counter-format=tsk[1].,label-offset=1em, label-align=right](5)
 \task always \task often  \task never  \task sometimes \task usually
\end{tasks}

  \subsection{现在时和过去时}
现在时: 经常发生的情况。 过去时\marginpar{故事中最常用的两种描述方式。}: 强调过去发生的,现在不发生的情况。\index{welcome} \index{spell} \index{point} \index{star} \index{start} \index{party} \index{must} \index{such as} \index{vegetable} \index{something} \index{someone} \index{climb} \index{type} \index{pet} \index{language} \index{Australia} \index{Australian} \index{hug} \index{greet} \index{greeting} \index{trap} \index{caught} \index{own} \index{on one's own} \index{owe} \index{moment} \index{fool} \index{foolish} \index{midnight} \index{owner} \index{dive} \index{diver} \index{homesick}

现在时注意事项: {\large{他、她、它的动词加s}{}}

\begin{printdemosample} \index{use}  \index{discover}\index{continent} \index{anyway} \index{honestly} \index{earlier} \index{stuff}
 I use scissors.

 We use scissors.
  
 She uses scissors
\end{printdemosample}

\begin{printdemosample}
 Mum never sings, but she hums all day long!
\end{printdemosample}


\begin{printdemosample}
 The sun rises in the east and sets in the west every day.
\end{printdemosample}

动词加s,作为第三人称单数:

{
\centering
\bgroup
\def\arraystretch{1.15} 
\begin{tabular}{|l|l||l|l|}
\hline
{动词原型} & {第三人称单数} &  {动词原型} & {第三人称单数} \\
\hline
{sing} & {sing} & {like} & {likes}  \\
{jump} & {jumps} & {give} & {gives} \\
\hline
\end{tabular}
\egroup{}
}

\begin{printdemosample}
 The duck jumps into the river.
 
 Three ducks jump into the river.
\end{printdemosample}


\vspace{0.5cm}

除了普通加s外,还有较多的一部分是和名词复数的情况类似的:

\begin{itemize}
 \item o, s, x, ch, sh结尾: + es
 \item 辅音+y结尾: y 变 ies
 \item 元音+y结尾: + s
\end{itemize}

实际中,好多动词都是特殊情况,这个需要平时多接触、多阅读才能熟练掌握,而不是孤立地背诵单个单词。  \index{do} \index{does} \index{wash} \index{wash up} \index{pass} \index{watch} \index{yell} \index{quarter} \index{a.m.} \index{p.m.} \index{pupil} \index{drop} \index{worst} \index{American} \index{Japanese} \index{until} \index{history} \index{historical fiction} \index{science fiction} \index{comedy} \index{ask} \index{remember} \index{island} \index{chick} \index{blond} \index{full} \index{kill} \index{alligator} \index{novel} \index{novelist} \index{mystery} \index{liquid} \index{recipe} \index{oven} \index{slice} \index{stir} \index{saucepan} \index{ingredient}


{
\centering
\bgroup
\def\arraystretch{1.15} 
\begin{tabular}{|l|l||l|l|}
\hline
{动词原型} & {第三人称单数} &  {动词原型} & {第三人称单数} \\
\hline
{go} & {goes} & {do} & {does}  \\
{fly} & {flies} & {study} & {studies} \\
{play} & {plays} & {teach} & {teaches} \\
{watch} & {watches} & {fix} & {fixes} \\
{wash} & {washes} & {pass} & {passes} \\
\hline
\end{tabular}
\egroup{}
}

\begin{printdemosample}
 She often goes to school by bus.
\end{printdemosample}


\newpage
\rule{\textwidth}{0.05cm}

\begin{question}
 Qucy \blank[width=1.5cm]{} (study) English, Chinese and Math at school.
\end{question}

\begin{question}
 We often \blank[width=1.5cm]{} (play) in the playground.
\end{question}

\begin{question}
 Miss Hill \blank[width=3.0cm]{} (get on) the bus. 
 
 Miss Hill \blank[width=3.0cm]{} (get off) the bus.
\end{question}

\begin{question}
 I always \blank[width=3.0cm]{} (walk) along the river. 
 
 He always \blank[width=3.0cm]{} (walk) along the river.
\end{question}

\begin{question}
 A raven \blank[width=1.0cm]{} (see) some cheese. He \blank[width=1cm]{}(take) the cheese, and \blank[width=1cm]{}(fly) to a tall tree.
\end{question}

\begin{question}
I \blank[width=1.5cm]{} (give) Lucy a paintbrush.

She \blank[width=1.5cm]{} (give) Lucy a paintbrush. 

They \blank[width=1.5cm]{} (give) Lucy a paintbrush.

Dad \blank[width=1.5cm]{} (give) Lucy a paintbrush.

The dragon \blank[width=1.5cm]{} (give) Lucy a paintbrush.
\end{question}

\rule{\textwidth}{0.05cm}

\vspace{0.3cm}
现在时的询问: Do, Does

\begin{printdemosample}
 Do you like carrots?
 
 Does he like apples?
\end{printdemosample}

一种学习方法,就是尝试把上面的示例和练习改成疑问以及否定方式。不断巩固。


否定方式的使用: do not, does not \index{not} \index{no}  \index{aren't}  \index{didn't} \index{ceiling} \index{intelligent} \index{collection} \index{collect} \index{compare}\index{actually} \index{apartment} \index{height} \index{pocket money} \index{personal} \index{length} \index{abroad} \index{south west} \index{file} \index{crazy} \index{hard-working} \index{definitely}
 

\begin{printdemosample}
 I don't like carrots.
 
 He doesn't like carrots.
\end{printdemosample}

\begin{printdemosample}
 "Does your dad like rainbows?" says the dragon. 
 
 "Yes," syas Lucy. "But he doesn't like a mess."
\end{printdemosample}


\newpage
\rule{\textwidth}{0.05cm}
  
过去时,强调以前发生的,现在不发生的事情。 \index{clean}

特点: {\large{动词加ed, is/am变was, are变were}{}}


{
\centering
\bgroup
\def\arraystretch{1.15} 
\begin{tabular}{|l|l||l|l|}
\hline
{动词原型} & {过去式} &  {动词原型} & {过去式} \\
\hline
{look} & {looked} & {watch} & {watched}  \\
{clean} & {cleaned} & {jump} & {jumped} \\
\hline
\end{tabular}
\egroup{}
}

\begin{printdemosample}
 Bob Bug was in his cot.
 
 "Get up, Bob." said Dad.
\end{printdemosample}

\begin{printdemosample}
 I was there yesterday.
\end{printdemosample}

\begin{printdemosample}
These rooms were very clean in the morning.
\end{printdemosample}

\framebox{练习}学着把前面现在时改成过去时

\vspace{0.8cm}

\begin{printdemosample}
  When I was in the countryside, I often walked by the riverside. 
\end{printdemosample}


动词使用过去式,格式变化较多,通常我们分成下面几种:

\begin{itemize}
 \item 直接 + ed
 \item 单词已经是e结尾, + d (dance, danced)
 \item 辅音字母+y结尾, y变i, 再 + ed (study)
 \item 辅音字母结尾,双写该字母, 再 + ed (shop)
 \item 以上规则都没有的(come, go)
\end{itemize}


不规则的过去式需要平时多读书来积累: \index{was}  \index{were} \index{sorry} \index{quietly}\index{politely} \index{engineer} \index{nest}\index{patiently}\index{enormous}\index{hatch}\index{chance} \index{the whole} \index{centimetre} \index{virus} \index{aspirin} \index{cure} \index{flu}

{
\centering
\bgroup
\def\arraystretch{1.15} 
\begin{tabular}{|l|l||l|l|}
\hline
{动词原型} & {过去式} &  {动词原型} & {过去式} \\
\hline
{go} & {went} & {do} & {did}  \\
{say} & {said} & {come} & {came} \\
{have/has} & {had} & {eat} & {ate} \\
{is/am} & {was} & {are} & {were} \\
{begin} & {began} & {give} & {gave} \\
\hline
\end{tabular}
\egroup{}
}

\newpage

\begin{printdemosample}
 She often came to help us in those days.
\end{printdemosample}


询问: Did(Was, Were), 否定: did not (was not, were not)

\begin{printdemosample}
 - Did you go to Beijing last week?
 
 - Yes, we did.
\end{printdemosample}

\begin{printdemosample}
 - Did you go to Beijing yesterday?
 
 - No, I didn't.
\end{printdemosample}

\begin{printdemosample}
 What did you do yesterday?
\end{printdemosample}

  \subsection{wh开头的询问}  \index{what} \index{who} \index{when} \index{where}  \index{which} \index{whose}  \index{why}   \index{how}  \index{game} \index{pretty} \index{bigger} \index{biggest} \index{happier} \index{dolphin} \index{starfish} \index{good at}

\begin{tasks}[counter-format=tsk[1].](4)
 \task when \task which \task what \task where
 \task who  \task whose \task why  \task how
\end{tasks}

\begin{printdemosample}
What are you doing?

What's this?
\end{printdemosample}

\begin{printdemosample}
 - Where are they going?
 
 - They're going to the game.
\end{printdemosample}
\begin{printdemosample}
When will my fish come?

Which is my fish?
\end{printdemosample}
  
how属于特殊的不是以wh打头的询问词:

\begin{printdemosample}
 How are you, Sucy?
\end{printdemosample}
  
\begin{printdemosample}
 How old are you?
 
 I'm ten.
\end{printdemosample}

\begin{printdemosample}
 How to make a hat?
\end{printdemosample}

\begin{printdemosample}
 - How many ducks can you see?
 
 - Three.
\end{printdemosample}

\begin{printdemosample}
 - How many marbles do you have?
 
 - I have ten marbles.
\end{printdemosample}

问天气如何:

\begin{printdemosample}
 - How's the weather?  \qquad \qquad What is the weather like?
 
 - It's sunny.
\end{printdemosample}

  
  \subsection{进行时的语法}
表明正在做某件事情:

% be doing

主要的特点: 动词加ing, 当然这个进行也包括下面两种: \index{hospital} \index{clinic} \index{sofa} \index{homework} \index{toothache} \index{cough} \index{temperature} \index{all right} \index{stomach} \index{stomachache} \index{fit} \index{fitness} \index{unfit} \index{keep fit} \index{so easy} \index{outfit} \index{elegant} \index{opera} \index{damage} \index{damaged} \index{middle-aged} \index{refund} \index{carpet} \index{pollution} \index{government}

\begin{itemize}
 \item was/were doing sth.
 \item is/am/are doing sth.
\end{itemize}

\begin{printdemosample}
 - What are you doing? 
 
 -I am doing my homework.
\end{printdemosample}

\begin{printdemosample}
 They aren't doing their homework.
\end{printdemosample}

\begin{printdemosample}
 We were watching TV from seven to nine last night.
\end{printdemosample}

\begin{question}
把下面句子改成进行时:

  I watch TV every day. \quad \blank[width=4cm]{}.
  
  She works in a hospital.   \quad \blank[width=4cm]{}.
  
  Do you read this book?  \quad  \blank[width=4cm]{}?
  
  They don't play computer games. \quad  \blank[width=4cm]{}.
  
\end{question}


\newpage

\section{名词}

  \subsection{名词指代词}
\lstinline{a,e,i,o,u}开头的,需要使用an, 其它用a. \index{egg} \index{umbrella} \index{a} \index{an} \index{amazing} \index{London} \index{Berlin} \index{well} \index{get well} \index{may} \index{speak} \index{talk} \index{phone} \index{cell phone} \index{mobile phone} \index{grass} \index{over} \index{music} \index{radio} \index{radio station} \index{hairdryer} \index{hairdresser} \index{most of} \index{make friends} \index{hand in} \index{proud} \index{forever}

\begin{tasks}[counter-format=tsk[1].,label-offset=1em, label-align=right](4)
 \task apple  \task egg \task insect     \task orange
 \task onion \task umbrella
 
\end{tasks}

这里需要注意的是, \lstinline{a,e,i,o,u}开头指的是读音,而不是简单的字母。

下面是几个示例:

\begin{tasks}(4)
 \task an hour
 \task an L
 \task an M
 \task an SOS
\end{tasks}

\begin{printdemosample}
  \begin{tikzpicture}
     \node [font=\large] (ID1) {Fish};
     \node [right of = ID1, xshift=4cm, yshift=-0.3cm] (ID2) {This word begins with an F.};
     
     \draw [->, >=stealth']($(ID2.west) + (0, -0.11)$) to [bend left=35] ($(ID1.south) + (0.0, -0.1)$);
  \end{tikzpicture}

\end{printdemosample}

单词the和a/an一样,是英文中出现频率最高的几个,意思多和中文里面的"这个"相近,用于指示某个物品。\index{the} \index{anywhere} \index{rule} \index{hurt} \index{whole} \index{area}

\begin{printdemosample}
 Look at the tree.
\end{printdemosample}

\begin{printdemosample}
Oh, the sun.
\end{printdemosample}

the在元音开头的字母前,需要发{ \begin{IPA}[Di:]\end{IPA}

\begin{tasks}
 \task the apple  \task the egg  \task the orange
\end{tasks}

“最喜欢的”说法是favorite,favourite, 极其有用:  \index{favorite} \index{boring} \index{cartoon} \index{drum} \index{road} \index{countryside}

\begin{printdemosample}
 - Hey, I was watching my favorite TV show!
 
 - Your show is boring, we want to watch cartoons.
\end{printdemosample}

\newpage

  \subsection{单数与复数} \index{ball}

\begin{printdemosample}
 There's a cat on that table.
 
 There're three ducks in the river.
 
 I have a lot of balls.
\end{printdemosample}

普通的复数情况: \index{animal} \index{animation}  \index{spider} \index{plant} \index{bookstore} \index{bookshop} \index{right} \index{left} \index{entertain} \index{entertainment}

{
\centering
\bgroup
\def\arraystretch{1.15} 
\begin{tabular}{|l|l|l|l|l|}
\hline
{单词复数} & {单词复数} &  {单词复数} & {单词复数} & {单词复数}\\
\hline
{birds} & {ducks} & {horses} & {books} & {spiders}  \\
{lambs} & {eggs} & {carrots} & {animals} & {plants}\\
\hline
\end{tabular}
\egroup{}
}

\vspace{0.3cm}

特殊情况 - s, o, x, sh, ch结尾的,多数是加es: \index{box} \index{bus}  \index{bus stop}  \index{scissors}

\begin{tasks}[counter-format=tsk[1].,label-offset=1em, label-align=right](4)
 \task bus  \task buses
 \task box  \task boxes
 \task glass    \task glasses
 \task potato  \task potatoes
 \task tomato  \task tomatoes
 \task  peach    \task peaches
 \task sandwich  \task sandwiches
\end{tasks}

注意有些o结尾的单词又不需要es: \index{photo} \index{piano} \index{rise} \index{rise up} \index{wise} \index{wealthy} \index{gradually} \index{forecast}

\begin{tasks}[counter-format=tsk[1].,label-offset=1em, label-align=right](2)
 \task photo  \task photos
 \task piano  \task pianos
\end{tasks}

\begin{printdemosample}
 I like tomatoes.
\end{printdemosample}



特殊情况 - 辅音加y结尾的, y变成i, 再加es

\begin{tasks}[counter-format=tsk[1].,label-offset=1em, label-align=right](4)
 \task cherry  \task cherries
 \task strawberry  \task strawberries
\end{tasks}

特殊情况 - 以f, fe结尾的,变v, 再加es
\begin{tasks}[counter-format=tsk[1].,label-offset=1em, label-align=right](4)
 \task life
\end{tasks}

特殊情况 - 不加s/es的: \index{tooth} \index{teeth} \index{child} \index{children} \index{mice} \index{rat} \index{sure} \index{energy}

\begin{tasks}[counter-format=tsk[1].,label-offset=1em, label-align=right](4)
 \task foot  \task feet \task tooth    \task teeth
 \task man  \task men   \task child  \task children
 \task mouse \task mice
\end{tasks}

\vspace{0.5cm}

不可数名词,没有s的情况: \index{people}\index{paper} \marginpar{\vspace{1.2cm}Fish and Chips}

\begin{tasks}[counter-format=tsk[1].,label-offset=1em, label-align=right](4)
 \task fish \task fun \task hair \task juice \task milk \task paper \task people \task rain  \task sheep \task water
 \task cheese
\end{tasks}

\begin{printdemosample}  \index{please} \index{cinema} \index{refer to} \index{fridge}\index{refrigerator} \index{belong to} \index{bigger} \index{faster} \index{helpful} \index{make sure} \index{make up}  \index{lost} \index{city}\index{town} \index{downtown} \index{Antarctica}
 - Have some juice, please!
 
 - Have some snacks, please!
\end{printdemosample}

\begin{printdemosample} \index{funny}  \index{fun} \index{strange} \index{note} \index{pocket} \index{seek} \index{before} \index{Tuesday} \index{sand} \index{look forward to}
 Hello, friends. Let's have some fun!
\end{printdemosample}


因为单数、复数,所以联想出数字单词: \index{one}  \index{two} \index{three}  \index{four} \index{five} \index{six} \index{seven} \index{eight} \index{nine} \index{ten} \index{eleven} \index{twelve}

\begin{tasks}[counter-format=tsk[1].,label-offset=1em, label-align=right](4)
 \task one     \task two  \task three  \task four  \task five
 \task six     \task seven  \task eight  \task nine  \task ten
 \task eleven  \task twelve
\end{tasks}

有了数目,联想"很多的"词汇: a lot of  \quad lots of  \quad some  \quad many                   \index{a lot of} \index{lots of} \index{few} \index{a few}

询问有多少? \index{how many} \index{many} \index{any} \index{arrive} \index{arrival} \index{rival} \index{round} \index{around} \index{hurry} \index{hurry up}

\begin{printdemosample}
 How many marbles?
 
 How many eggs?
\end{printdemosample}


\newpage

  \subsection{动物(animals)} \index{goat} \index{cow} \index{tiger} \index{parrot} \index{hamster} \index{horse} \index{sheep} \index{shark} \index{giraffe} \index{fox} \index{hard}\index{work hard} \index{hardware} \index{war} \index{delete} \index{work} \index{housework} \index{bring} \index{story} \index{float} \index{sink} \index{rock} \index{hedgehog} \index{selfish}

  \index{rabbit} \index{hare} \index{insect} \index{zebra} \index{monkey} \index{cat} \index{zoo} \index{panda} \index{mouse} \index{tortoise} \index{bear} \index{puppy} \index{bee}
  \index{duck} \index{lamb} \index{frog} \index{pig} \index{fish}  \index{ant} \index{dragonfly} \index{ladybird} \index{butterfly} \index{cicada} \index{bird} \index{dog}  \index{farm} \index{elephant} \index{wildlife} \index{snake} \index{whale} \index{hippo} \index{kangaroo} \index{butter}

\marginpar{\vspace{1cm}zoo \\ farm \\ puppy}

\begin{tasks}[counter-format=tsk[1].,label-offset=1em, label-align=right](4)
 \task rabbit  \task insect \task dog     \task zebra
 \task cat     \task tiger  \task panda   \task monkey
 \task frog    \task parrot \task mouse   \task tortoise
 \task giraffe \task bear   \task shark   \task ant
 \task duck    \task sheep  \task bird  
 \task butterfly \task dragonfly \task ladybird \task cicada
 \task goat     \task cow  \task hamster  \task horse
 \task fox     \task lamb  \task pig      \task fish
 \task bee     \task elephant \task snake  \task wild life
\end{tasks}

由狗联想出单词: \qquad bark  \qquad woof  \qquad bone                                           \index{bone} \index{bark} 

  \subsection{植物(plants)} \index{bamboo} \index{flower} \index{tree}  \index{police}\index{police station} \index{officer} \index{easy} \index{wish} \index{dish} \index{quack} \index{for example} \index{exam} \index{hope} \index{hopelessly}
  
\begin{tasks}[counter-format=tsk[1].](4)
 \task flower \task tree \task trunk \task bark
 \task leaf  \task leaves  \task branch  \task rose
 \task bamboo
\end{tasks}

  \subsection{场景化的对话}
初次见面的打招呼:  \index{glad} \index{done} \index{be made of}  \index{word} \index{fig}  \index{buzz}   \index{example} \index{me}  \index{made}

\begin{printdemosample}
 - Nice to meet you. \quad/ \qquad Glad to meet you
 
 - Me too. \quad/ \qquad Nice to meet you too.
\end{printdemosample}

熟悉人的打招呼则使用Hello, 或者Hi即可。


称赞:\index{Well done} \index{I'm sorry} \index{beautiful} \index{how much} \index{how often} \index{jump into} \index{went to} \index{went} \index{cost} \index{came} \index{pay} \index{pay back} \index{bill} \index{high jump} \index{long jump}

\begin{printdemosample}
 - Well done! \quad/  Perfect! \quad/ Excellent! \quad /  They're beautiful!
 
 - Thank you.
\end{printdemosample}

询问陌生人: \index{thank you} \index{thank} \index{show off} \index{thanks} \index{roll} \index{spin} \index{glass} \index{glasses} \index{forward} \index{is broken} \index{be broken}

\begin{printdemosample}
 - Execuse me, xxxxx
\end{printdemosample}

对不起:  \index{Excuse me}

\begin{printdemosample}
 - I'm sorry.
 
 - It's ok.
\end{printdemosample}


\newpage

  \subsection{食物(food)} \index{food}
 \index{cherry} \index{salad} \index{banana} \index{jelly} \index{juice} \index{onion} \index{cheese} \index{apple} \index{bread} \index{melon} \index{pineapple} \index{pumpkin}\index{rice} \index{soup}\index{soap} \index{soap opera} \index{noodle}

基本学习的单词包括: \index{chip} \index{carrot} \index{grape} \index{eat} \index{drink} \index{tomato} \index{hamburger} \index{sausage} \index{milk} \index{potato} \index{sandwich} \index{tart} \index{pizza}   \index{orange} \index{watermelon} \index{coconut} \index{biscuit} \index{yogurt}

\begin{tasks}[counter-format=tsk[1].,label-offset=1em, label-align=right](4)
 \task apple     \task onion   \task cupcake  \task cherry
 \task carrot    \task banana  \task grape    \task peach
 \task pizza     \task cake    \task salad    \task chip
 \task hamburger \task tomato  \task donut    \task sausage
 \task milk      \task jelly   \task potato   \task cheese
 \task bread     \task coffee   \task juice   \task pear
 \task sandwich  \task tart    \task melon \task watermelon
 \task pumpkin \task pineapple \task coconut \task biscuit
\end{tasks}

\vspace{0.5cm}

联想: \index{yummy} \index{yuk} \index{coffee} \index{wait} \index{wait for} \index{near} \index{far} \index{become} \index{market}\index{marketing} \index{bump into} \index{exit} \index{love} \index{fall in love} \index{in particular} \index{nearly} \index{almost} \index{travel} \index{travel agent} \index{trip} \index{everything} \index{thing} \index{without} \index{really} \index{chemist} \index{performer} \index{perform} \index{performance} \index{proper} \index{admiring} \index{admiration} \index{get married} \index{smooth}

\begin{tasks}[counter-format=tsk[1].](2)
 \task yummy  \task yuk
 \task eat    \task drink
\end{tasks}

扩展: \index{breakfast} \index{lunch} \index{dinner} \index{supper} \index{begin}\index{begin to} \index{beginning} \index{ticket} \index{wife} \index{empty} \index{cry} \index{return} \index{was born} \index{German} \index{Germany} \index{competitor} \index{course} \index{musician}

\begin{tasks}[counter-format=tsk[1].](2)
 \task breakfast  \task lunch
 \task supper    \task dinner
\end{tasks}

\vspace{0.8cm}

{\large{\~{}berry}{}} \hspace{0.2cm}  这是一个和甜水果有关的后缀, 通常小小,圆圆的小果子都带有这样的后缀: \index{camera}  \index{strawberry} \index{still} \index{blueberry}  \index{yours} \index{mine} \index{ours} \index{theirs} \index{hers} \index{cell} \index{real} \index{photographer} \index{image} \index{sat}\index{told} \index{raw}

  比如: \marginpar{\vspace{0.6cm}ferry}
  
\begin{tasks}[counter-format=tsk[1].](2)
 \task berry         \task blueberry
 \task strawberry    \task lingonberry
\end{tasks}


\newpage

  \subsection{文具用品} \index{film} \index{computer} \index{chair}\index{school bag} \index{ruler}\index{rubber} \index{eraser} \index{pencil case} \index{pen} \index{bag}  \index{card} \index{class} \index{playground} \index{desk}  \index{grade}  \index{blackboard} \index{school}  \index{school bus} \index{balloon} \index{table}   \index{window} \index{bell} \index{toy}  \index{doll} \index{robot} \index{pencil} \index{science} \index{scientist} \index{math}\index{maths} \index{classroom} \index{step}


\marginpar{\vspace{0.4cm}eraser为美国人对橡皮的说法}

\begin{tasks}[counter-format=tsk[1].,label-offset=1em, label-align=right](4)
 \task pencil case \task school bag \task rubber \task eraser
 \task pen         \task pencil     \task ruler  \task book
 \task desk        \task grade      \task class  \task chair
 \task playground  \task film       \task computer \task blackboard
 \task school bus  \task balloon    \task window  \task bell
 \task card        \task toy        \task doll   \task robot
\end{tasks}

  \subsection{衣服服装}

这里包括了衣、帽、鞋等人穿着相关的所有词汇说法。\index{cap} \index{skirt} \index{shirt} \index{T-shirt} \index{sweater}\index{anorak} \index{coat} \index{scarf} \index{beanie} \index{shoe} \index{cloth} \index{vest} \index{sock} \index{jacket} \index{trousers} \index{pants}\index{whereas}       \marginpar{pants} 

\begin{tasks}[counter-format=tsk[1].,label-offset=1em, label-align=right](4)
 \task sweater  \task cap   \task T-shirt  \task coat
 \task scarf    \task shirt  \task anorak \task beanie
 \task dress    \task shoe   \task hat   \task cloth
 \task vest     \task sock   \task jacket \task trousers
\end{tasks}

\begin{printdemosample} \index{dress} \index{dress up} \index{need} \index{needn't} \index{invite} \index{myself} \index{letter} \index{soon} \index{village} \index{male} \index{female} \index{more than}
 You do not need to dress up for dinner 
\end{printdemosample}

帽子由于不同类型,说法不同。cap \qquad beanie  \qquad hat \index{hat} \index{send} \index{place} \index{listen to music} \index{New Zealand}



  \subsection{其它杂项单词}
这里是将之前ch\_concret\_eng文档中剩下的词汇都汇总到这里: \index{dirty}  \index{first}  \index{mess} \index{Let's}   \index{age} \index{another} \index{along} \index{lucky} \index{luck} \index{song}  \index{shell}   \index{sick} \index{sickness}  \index{ill} \index{illness}\index{quiet} \index{swimming}  \index{map}  \index{not yet} \index{all}  \index{some} \index{interesting} \index{interest} \index{sea} \index{beach} \index{recover} \index{cover}
\index{ready}  \index{ring} \index{king} \index{queen} \index{chop}   \index{swimming pool} \index{the other} \index{good luck}
\index{pull} \index{push} \index{strong} \index{end} \index{integer} \index{minus} \index{hole} \index{manager}

\begin{tasks}[counter-format=tsk[1].,label-offset=1em, label-align=right](4)
 \task dirty  \task first  \task mess  \task Let's
 \task age  \task another  \task along  \task lucky
 \task song   \task shell  \task sick \task quiet
 \task sing   \task swimming \task map \task swimming pool
 \task ready    \task ring   \task king \task not yet
 \task queen  \task chop  \task all  \task interesting
\end{tasks}


  
\newpage

\section{拾遗}

  \subsection{数学英语-初级}
由于英语和数学的结合将是一个趋势,所以这里开始需要让笑笑掌握简单的数学英语知识点。这也有助于后面阅读科技文章。\index{plus} \index{equal} \index{sum} \index{Me too} \index{fifty} \index{sixty} \index{eyeslash} \index{beak} \index{reuse} \index{recycle} \index{recycling bin} \index{tip} \index{protection} \index{technology}

本小节主要以\cite{bibCaliforniaMath}中内容安排为基础,逐步介绍数学中,英语的描述方式。\index{add} \index{addition}

我们首先从加法(addition)开始,第一个学习的就是加法符号(+),在英语中称为plus: \marginpar{add, addition} \index{thirteen} \index{fourteen} \index{sixteen} \index{fifteen} \index{seventeen} \index{eighteen} \index{nineteen} \index{twenty} \index{remain} \index{taste} \index{Sweden} \index{produce} \index{butcher} \index{design} \index{designer} \index{cotton} \index{horrible} \index{go with} \index{disgusting}

\begin{tikzpicture}
  \node [] (IDmath) {$2 + 3 = 5$};
  
  \node [right of = IDmath, xshift=3cm] {2 plus 3 equals 5};
  
  \node at (-2, -0.7) [] (IDplus){plus}; \node at (2, -0.7) [] (IDsum){sum};
  \draw [->, >=stealth'] (IDplus.east) to [bend right=35] ($(IDmath.south) + (-0.4, 0)$);
  \draw [->, >=stealth'] (IDsum.west) to [bend left=30] ($(IDmath.south) + (0.6, 0)$);
  %\node at () {};
\end{tikzpicture}

"等于"的单词为equal, 所以上述过程为: $\;$ \underline{2 plus 3 equals 5}

同时,我们对于上面的加法,可以说: $\;$ \underline{5 is the sum of $3 + 2$}

下面可以学着使用这些说法来完成习题,这些练习均来自\cite{bibCaliforniaMath}中的内容:



减法(substraction)的符号$-$英语中称为minus, 下面是减法的运算示例: \marginpar{substraction} \index{substraction} \index{all night}

\begin{tikzpicture}
  \node [] (IDmath) {$5 - 2 = 3$};
  
  \node [right of = IDmath, xshift=3cm] {five minus two equals three};
  
  \node at (-2, -0.7) [] (IDplus){minus}; \node at (2, -0.7) [] (IDsum){difference};
  \draw [->, >=stealth'] (IDplus.east) to [bend right=35] ($(IDmath.south) + (-0.4, 0)$);
  \draw [->, >=stealth'] (IDsum.west) to [bend left=30] ($(IDmath.south) + (0.6, 0)$);
  %\node at () {};
\end{tikzpicture}

尝试完成下面的试题练习:




  \subsection{易混淆单词} \index{too}  \index{fan} \index{mirror} \index{bin} \index{bean} \index{even} \index{popular} \index{review}\index{view} \index{saying} \index{on foot} \index{shopkeeper} \index{hunt} \index{hunter} \index{peanut} \index{national} \index{mark} \index{marker} \index{cries} \index{handle} \index{dead} \index{document} \index{documentary} \index{audience} \index{screen}
  
\begin{tasks}[counter-format=tsk[1].,label-offset=1em, label-align=right](2)
 \task lamb \quad lamp   \task bad \quad bed
 \task for \quad four  \task  nose \quad nurse
 \task house \quad horse  \task to \quad two \quad too
 \task coat \quad goat \quad boat  \task every  \quad very
 \task buy  \quad by  \task fun   \quad fan
 \task bin  \quad bean  \task body  \task baby
\end{tasks}

极容易混的词; \index{live} \index{swamp} \index{bug} \index{lives} \index{style} \index{mood} \index{Asia} \index{Asian} \index{Africa} \index{African} \index{Europe} \index{European} \index{post} \index{satisfy} \index{satisfied}\index{satisfactory} \index{material} \index{section} \index{environment} \index{across} \index{whether} \index{dump} \index{alphabet} \index{Geography} \index{beat} \index{fold}

\vspace{0.7cm}

{\large{lives}{}}

\vspace{1cm}

{\large{read}{}}


  \subsection{相加词汇} \index{skate} \index{basketball} \index{baseball} \index{football} \index{badminton} \index{toothbrush}\index{toothpaste} \index{bathtub}\index{jellybean}\index{skateboard} \index{perhaps} \index{rush} \index{rush hour} \index{tiny} \index{dumpling} \index{pack} \index{unpack} \index{backpack} \index{backpacking} \index{backpacker} \index{zip} \index{unzip} \index{less} \index{least} \index{at least} \index{guide} \index{guidebook} \index{provide} \index{exchange} \index{species} \index{currency} \index{delay} \index{reservation} \index{escape} \index{resort} \index{sleeve} \index{collar} \index{visa} \index{local}

\begin{tasks}[counter-format=tsk[1].,label-offset=1em, label-align=right](4)
 \task toothbrush  \task paintbrush \task blackboard \task skateboard 
 \task bathtub 
 \task football    \task jellybean
\end{tasks}

skateboard/scooter


  
%%%%%%%%%%%%%%%%%%%%%%%%%%%
%% 2017-11-18 英语天天练
%%%%%%%%%%%%%%%%%%%%%%%%%%%
\newpage


%%%%%%%%%%%%%%%%%%%%%%%%%%%%%%%%%%%%%%%%%%%%%%%%%%%%%%%%%%%%%

  
 

\begin{thebibliography}{99}
\bibitem{bibNewConceptEngYoungVer} 外语教学与研究出版社:{\em 新概念英语青少版入门级}, 2010年.
\bibitem{bibEngJuniorOneA}译林出版社:{\em 小学英语(一上)}, 2012年.
\bibitem{bibEngJuniorOneB}译林出版社:{\em 小学英语(一下)}, 2012年.
\bibitem{bibEngJuniorTwoA}译林出版社:{\em 小学英语(二上)}, 2016年.
\bibitem{bibPhonicsStageTwo} 外语教学与研究出版社:{\em 丽声拼读故事会 第二级}, 2011年.
\bibitem{bibNationalGeoBasic} 外语教学与研究出版社:{\em 国家地理儿童百科入门级}, 2010年.
\bibitem{bibNewDongFangReadBasic} 西安交通大学出版社:{\em 新东方泡泡剑桥儿童英语故事阅读}, 2012年.
\bibitem{bibCaliforniaMath}McGraw-Hill Press:{\em 加州小学数学}, 2012年.
\end{thebibliography}

\printindex

\end{document}
