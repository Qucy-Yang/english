%%小学错题集合 2021-01-23


\settasks{counter-format=tsk[A]) }

\begin{question}[tags={xiaoxuect}]
The old men under the tree \blank[width=1.5cm]{}(talk) loudly now.
\end{question}
\begin{solution}
are talking

注意句子中的now和men这个代表复数的线索
\end{solution}

\begin{question}[tags={xiaoxuect}]
The woman \blank[width=1cm]{}(can't) find her bag. She was very \blank[width=1cm]{}(sad).
\end{question}
\begin{solution}
couldn't, sad

注意句子中的was的线索
\end{solution}

\begin{question}[tags={xiaoxuect}]
He \blank[width=1cm]{}(bring) some water quickly and \blank[width=1cm]{}(pour) it into the hole just now.
\end{question}
\begin{solution}
brought, poured

注意句子中just now的线索,表示刚刚发生过的事情
\end{solution}


\begin{question}[tags={xiaoxuect}]
Helen is good \blank[width=1cm]{} basketball, but her brother \blank[width=1cm]{}.

  \begin{tasks}(4)
    \task at; isn't
    \task in; doesn't
    \task at; doesn't
    \task in; isn't
  \end{tasks}
\end{question}
\begin{solution}
A

这个臭东西错选了C
\end{solution}

\begin{question}[tags={xiaoxuect}]
The children were very \blank[width=1cm]{}. They played \blank[width=1cm]{} in the park.

  \begin{tasks}(3)
    \task excited; exciting
    \task excited; excitedly
    \task exciting; excitedly
  \end{tasks}
\end{question}
\begin{solution}
B

这个臭东西错选了A
\end{solution}

\begin{question}[tags={xiaoxuect}]
My uncle caught a thief and \blank[width=1cm]{} him \blank[width=1cm]{} the police station.

  \begin{tasks}(4)
    \task take; for
    \task take; to
    \task took; to
    \task took; for
  \end{tasks}
\end{question}
\begin{solution}
C

这个臭东西错选了B
\end{solution}

\begin{question}[tags={xiaoxuect}]
Tim is on the ground. Can you help \blank[width=1cm]{} ?

  \begin{tasks}(4)
    \task he up
    \task him down
    \task he down
    \task him up
  \end{tasks}
\end{question}
\begin{solution}
D

注意理解句子的意思
\end{solution}


\begin{question}[tags={xiaoxuect}]
Let the worker \blank[width=1cm]{} a hole in the ground.

  \begin{tasks}(3)
    \task make
    \task made
    \task makes
  \end{tasks}
\end{question}
\begin{solution}
A

还要注意Let him/her/me等用法
\end{solution}

\begin{question}[tags={xiaoxuect}]
The mouse \blank[width=1cm]{} the lion \blank[width=1cm]{} out from the net at last.

  \begin{tasks}(3)
    \task help; got
    \task helped; get
    \task helped; got
  \end{tasks}
\end{question}
\begin{solution}
B

help sb. do sth.
\end{solution}

\begin{question}[tags={xiaoxuect}]
Tim usually \blank[width=1cm]{} his homework before supper, but yesterday he \blank[width=1cm]{}.

  \begin{tasks}(2)
    \task finishes; doesn't
    \task finish; didn't
    \task finishes; didn't
    \task finish; doesn't
  \end{tasks}
\end{question}
\begin{solution}
C

这个臭东西错选了A
\end{solution}

\begin{question}[tags={xiaoxuect}]
Who teaches \blank[width=1cm]{} English?

  \begin{tasks}(4)
    \task them   \task their
    \task they   \task theirs
  \end{tasks}
\end{question}
\begin{solution}
A

这个容易按照中文思维习惯误选B
\end{solution}

\begin{question}[tags={xiaoxuect}]
My uncle went to Shenzhen \blank[width=1cm]{}.

  \begin{tasks}(2)
    \task in 1990s       \task in the 1990s
    \task on the 1990s   \task on the 1990
  \end{tasks}
\end{question}
\begin{solution}
B

这是'在90年代'的习惯说法
\end{solution}

\begin{question}[tags={xiaoxuect}]
Summer is coming. It's getting \blank[width=1cm]{}.

  \begin{tasks}(1)
    \task hot and hot         \task hotter and hotter
    \task more and more hot   \task more hotter and more hotter
  \end{tasks}
\end{question}
\begin{solution}
B

不要误选C
\end{solution}

\begin{question}[tags={xiaoxuect}]
\blank*[width=1cm]{} of them has been to Australia serveral times, so they know the Opera House very well.

  \begin{tasks}(4)
    \task Both         \task All
    \task None         \task Each
  \end{tasks}
\end{question}
\begin{solution}
D

该题极其容易让人错选B,句子当中的has是重要线索
\end{solution}

\begin{question}[tags={xiaoxuect}]
Tom's father was \blank[width=1cm]{} at Tom's \blank[width=1cm]{} performance.

  \begin{tasks}(2)
    \task disappointed; disappointed
    \task disappointed; disappointing
    \task disappointing; disappointed
    \task disappointing; disappointing
  \end{tasks}
\end{question}
\begin{solution}
B

该题目理解意思,不要出错
\end{solution}

\begin{question}[tags={xiaoxuect}]
Linda said that the moon \blank[width=1cm]{} around the earth.

  \begin{tasks}(3)
    \task traveled
    \task travels
    \task had traveled
  \end{tasks}
\end{question}
\begin{solution}
B

注意不能教条地想着动词的词性保持一致
\end{solution}

\begin{question}[tags={xiaoxuect}]
- Must I wait here all morning? I have a lot of work to do.

\noindent - No, you \blank[width=1cm]{}. You may be back in the afternoo.

  \begin{tasks}(4)
    \task mustn't
    \task can't
    \task don't have to
    \task couldn't
  \end{tasks}
\end{question}
\begin{solution}
C
\end{solution}

\begin{question}[tags={xiaoxuect}]
May I \blank[width=1cm]{} with you?

  \begin{tasks}(4)
    \task to go
    \task going
    \task goes
    \task go
  \end{tasks}
\end{question}
\begin{solution}
D

对于这类,可以先尝试将问句还原回去
\end{solution}


\begin{question}[tags={xiaoxuect}]
- The boy looks \blank[width=1cm]{}.

\noindent - Yes, tomorrow is his birthday.

  \begin{tasks}(4)
    \task happily
    \task sadly
    \task happy
    \task sad
  \end{tasks}
\end{question}
\begin{solution}
C

类似的句子还有It sounds good.
\end{solution}

\begin{question}[tags={xiaoxuect}]
\blank*[width=1cm]{} a good job, we must have the right ways.

  \begin{tasks}(4)
    \task Doing
    \task To do
    \task Done
    \task Do
  \end{tasks}
\end{question}
\begin{solution}
B

如果想使用动名词,那么主语就不应该是后面的we, 所以这里是要表达原因,使用不定式来表示原因.
\end{solution}

\begin{question}[tags={xiaoxuect}]
I usually \blank[width=1cm]{} my bedroom \blank[width=1cm]{} before going to shcool.

  \begin{tasks}(4)
    \task get; cleaned
    \task get; clean
    \task get; cleaning
    \task get; cleans
  \end{tasks}
\end{question}
\begin{solution}
A

get sth done, have sth done,习惯用法
\end{solution}

\begin{question}[tags={xiaoxuect}]
I always go to school early in the morning. (用never改写句子)

\noindent I \blank[width=1cm]{} go to school \blank[width=1cm]{} in the morning.

\end{question}
\begin{solution}
never; late
\end{solution}

\begin{question}[tags={xiaoxuect}]
I gave the little boy a sweet, then he \blank[width=1cm]{}.

  \begin{tasks}(2)
    \task become happily
    \task become happy
    \task became happy
    \task became happily
  \end{tasks}

\end{question}
\begin{solution}
C
\end{solution}

\begin{question}[tags={xiaoxuect}]
My friend and I \blank[width=1cm]{} like running.

  \begin{tasks}(4)
    \task all
    \task are
    \task both
    \task very
  \end{tasks}

\end{question}
\begin{solution}
C

错选B了,注意喜欢和像这两个意思表达
\end{solution}

\begin{question}[tags={xiaoxuect}]
His hobby is \blank[width=1cm]{} books.

  \begin{tasks}(3)
    \task reads
    \task read
    \task reading
  \end{tasks}

\end{question}
\begin{solution}
C

错选了A
\end{solution}

\begin{question}[tags={xiaoxuect}]
- Excuse me, is this the right way to the Palace Museum?

\noindent - Sorry, I am not sure. But it \blank[width=1cm]{} be.

  \begin{tasks}(4)
    \task needn't
    \task can
    \task can't
    \task must
  \end{tasks}

\end{question}
\begin{solution}
B

若选D,那么需要明白must表示根据某个现象而推断出可能性很高的结果,

而can/could则一般用来表述可能性较低的结果
\end{solution}



\begin{question}[tags={xiaoxuect}]
The teacher asked us \blank[width=1cm]{} so much noise.

  \begin{tasks}(4)
    \task don't make
    \task not make
    \task not making
    \task not to make
  \end{tasks}

\end{question}
\begin{solution}
D

错选A, ask sb not to do sth, as sb to do sth
\end{solution}


\begin{question}[tags={xiaoxuect}]
There \blank[width=1cm]{} a mouse in my house. Look, our food is destroyed.

  \begin{tasks}(4)
    \task must have
    \task could have
    \task must be
    \task could be
  \end{tasks}

\end{question}
\begin{solution}
C/D均可

must表明的可能性更高一些
\end{solution}


\begin{question}[tags={xiaoxuect}]
My sisters \blank[width=1cm]{} like meat.

  \begin{tasks}(4)
    \task aren't
    \task doesn't
    \task don't
    \task isn't
  \end{tasks}

\end{question}
\begin{solution}
C

错选了B
\end{solution}


\begin{question}[tags={xiaoxuect}]
Here \blank[width=1cm]{} a lot of meat for the party.

  \begin{tasks}(4)
    \task are
    \task is
    \task that
    \task be
  \end{tasks}

\end{question}
\begin{solution}
B

错选了A
\end{solution}


\begin{question}[tags={xiaoxuect}]
The students are cheering for the football match \blank[width=1cm]{}(excite).

\end{question}
\begin{solution}
excitedly

错写为excited,不符合语法规则
\end{solution}


\begin{question}[tags={xiaoxuect}]
- Please go \blank[width=1cm]{} the bedroom, Mary.

\noindent - OK.

  \begin{tasks}(3)
    \task out
    \task of
    \task into
  \end{tasks}

\end{question}
\begin{solution}
C

若选A,那么应该为go out of xxxx
\end{solution}


\begin{question}[tags={xiaoxuect}]
The mouse helped the lion \blank[width=1cm]{} out from the net ten minutes ago.

  \begin{tasks}(3)
    \task get
    \task got
    \task getting
  \end{tasks}

\end{question}
\begin{solution}
B

句型 help sb do sth.
\end{solution}

\begin{question}[tags={xiaoxuect}]
There is not \blank[width=1cm]{}(many) food in the fridge.
\end{question}
\begin{solution}
much

错写成了any,注意不可数名词多少是用much
\end{solution}

\begin{question}[tags={xiaoxuect}]
There \blank[width=1cm]{}(be not) any meat on my plate just now.
\end{question}
\begin{solution}
wasn't

两个注意点,一个是不能写arn't,因为meat是不可数,另一个是just now的提示表明它是一个过去式
\end{solution}

\begin{question}[tags={xiaoxuect}]
- How about \blank[width=1cm]{}(drink) some tea?

\noindent - No, thank you.
\end{question}
\begin{solution}
drinkig

注意how about doing sth的句型用法,错写成了drink
\end{solution}

\begin{question}[tags={xiaoxuect}]
\blank*[width=1cm]{} there any milk in the glass just now?

  \begin{tasks}(4)
    \task Is
    \task Was
    \task Are
    \task Were
  \end{tasks}

\end{question}
\begin{solution}
B

注意不可数,另外just now的暗示
\end{solution}

\begin{question}[tags={xiaoxuect}]
- Can I have \blank[width=1cm]{} orange juice?

\noindent - Sure.

  \begin{tasks}(3)
    \task any
    \task many
    \task some
  \end{tasks}

\end{question}
\begin{solution}
C

错选了A,希望得到肯定回答场景要用some
\end{solution}

\begin{question}[tags={xiaoxuect}]
Mary has \underline{a little} meat every day. (对画线部分提问)

\noindent \blank*[width=2cm]{} \blank*[width=2cm]{} meat does Mary have every day?

\end{question}
\begin{solution}
How much

错写成How many,注意不可数
\end{solution}

\begin{question}[tags={xiaoxuect}]
I am going to watch TV after I finish \blank[width=1.5cm]{}(do) my homework this evening.
\end{question}
\begin{solution}
doing

finish doing sth.完成某件事,错误写为done,
\end{solution}

\begin{question}[tags={xiaoxuect}]
The students are singing \blank[width=1.0cm]{} in the classroom.

  \begin{tasks}(3)
    \task excitedly
    \task excited
    \task exciting
  \end{tasks}
\end{question}
\begin{solution}
A

错选B
\end{solution}

\begin{question}[tags={xiaoxuect}]
The flowers are very beautiful. (改为感叹句)

\noindent \blank*[width=1.9cm]{} \blank*[width=1.9cm]{} flowers they are!
\end{question}
\begin{solution}
What beautiful

错写成How beautiful
\end{solution}

\begin{question}[tags={xiaoxuect}]
The boy can jump over the chair \blank[width=1.4cm]{}(easy).
\end{question}
\begin{solution}
easily

错写成easyly
\end{solution}

\begin{question}[tags={xiaoxuect}]
Look! The ducks \blank[width=1.4cm]{}(cross) the road \blank[width=1cm]{}(slow).
\end{question}
\begin{solution}
are crossing, slowly

第一个由于有Look的先行,表明正在发生,而不能写cross

其他类似的有Listen!等等
\end{solution}


\begin{question}[tags={xiaoxuect}]
Yang Ling is going to \blank[width=1.4cm]{}(带来) some toys for the party.
\end{question}
\begin{solution}
bring

错写成take, take表示带到某处,而bring是从别的地方带来,描述的方向不一样
\end{solution}


\begin{question}[tags={xiaoxuect}]
How about \blank[width=1.4cm]{}(go) on a picnic tomorrow afternoon?
\end{question}
\begin{solution}
going

建议去做某个事情是how about doing sth, 或者what about doing sth
\end{solution}

\begin{question}[tags={xiaoxuect}]
There \blank[width=1.4cm]{}(be) a football game on TV tomorrow evening.
\end{question}
\begin{solution}
will be, 或者is going to be

注意时间的单词线索
\end{solution}

\begin{question}[tags={xiaoxuect}]
\blank*[width=1.4cm]{} anybody \blank[width=1.4cm]{} why it snows in winter?

  \begin{tasks}(2)
    \task Does; knows
    \task Does; know
    \task Do; knows
    \task Do; know
  \end{tasks}

\end{question}
\begin{solution}
B

错选了D
\end{solution}

\begin{question}[tags={xiaoxuect}]
Mike \blank[width=1.4cm]{}(stay) in London for two weeks last year.
\end{question}
\begin{solution}
stayed

错填写成staied, 错误写法! 同理还有play
\end{solution}

\begin{question}[tags={xiaoxuect}]
She is talking \blank[width=1.4cm]{}(happy) with her friends.
\end{question}
\begin{solution}
happily

不会写happy的副词形式
\end{solution}

\begin{question}[tags={xiaoxuect}]
I want to be a \blank[width=1.4cm]{}. I like \blank[width=1.4cm]{} around the world. (travel)
\end{question}
\begin{solution}
traveller, travelling
\end{solution}

\begin{question}[tags={xiaoxuect}]
Does he go to the UK \blank[width=1.4cm]{} ship?

  \begin{tasks}(3)
    \task take
    \task by a
    \task by
  \end{tasks}

\end{question}
\begin{solution}
c

错选了B
\end{solution}

\begin{question}[tags={xiaoxuect}]
There are some \blank[width=1.4cm]{} (visit) on the bus.
\end{question}
\begin{solution}
visitors

不要写单数,同时注意不能写成visiters, 是or, 不是er

类似的有doctor, actor
\end{solution}

\begin{question}[tags={xiaoxuect}]
- \blank*[width=1.4cm]{} is the weather in spring?

\noindent- It's sometimes \blank[width=1.4cm]{}.

  \begin{tasks}(3)
    \task What; rains
    \task How; rainy
    \task What; rainy
    \task How; rains
  \end{tasks}

\end{question}
\begin{solution}
B

错选了C, 注意问天气的方法
\end{solution}


\begin{question}[tags={xiaoxuect}]
There \blank*[width=1.4cm]{} a football match this afternoon.

  \begin{tasks}(3)
    \task will have
    \task will be
    \task is going to have
  \end{tasks}

\end{question}
\begin{solution}
B

错选了C, 注意总是犯这个错! 一般主语是人的时候,have才代表有
\end{solution}



\begin{question}[tags={xiaoxuect}]

\noindent 我爸爸下周一将给我寄一些钱

\noindent My father \blank*[width=1.4cm]{} \blank[width=1.4cm]{} some money \blank[width=1.4cm]{} me next Monday.

\end{question}
\begin{solution}
will send, to

错写成will sends

will后面动词是原型

另外,usually, sometimes这类代表频率的词,才不影响三单
\end{solution}

%% 作文小学部分
\section{小学作文的积累}

  \subsection{讲述事情}
学会使用顺序描述词:

\begin{itemize}
    \item First
    \item Then
    \item Next
    \item After that
    \item Finally
\end{itemize}

如果描述好,那么需要会写下面几个:

\begin{itemize}
    \item good
    \item perfect
    \item terrific
    \item nice
    \item excellent
\end{itemize}

注意surprise这个词,Qucy在练习中没有正确书写出来
