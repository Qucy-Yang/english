% An Annoated version of Grammer
%
% [2020-12-05] Create this new grammar one
%
% [2015-10-19] Move all here, work well on TeX-Live2014, but Live-2015 seems
%              NOT GOOD for the ' symbole output.
%
% [2015-10-18] Try convert the concrete_eng.tex into XeLatex format

\documentclass[a4paper]{article}
\title{English Grammar}
\author{Yang Songxiang}

%% MUST defined before fontspce
\usepackage{tipa}

\usepackage{fontspec,xunicode,xltxtra,makeidx,xecolor}
\usepackage{color,listings,tabularx,amsfonts}
\usepackage{amssymb}
\usepackage[table]{xcolor}
\usepackage{wrapfig}
\usepackage{titlesec}
\usepackage{lettrine}
\usepackage{framed}
\usepackage{colortbl}
\usepackage{multicol}
\setlength{\columnseprule}{1pt}
\usepackage[many]{tcolorbox}
\usepackage[framemethod=tikz]{mdframed}
\usepackage{epigraph}
\usepackage{soul}
\usepackage[nonumberlist]{glossaries}
%
%%
\usepackage{multirow}
\usepackage{tikz}
\usepackage{tkz-base}
\usetikzlibrary{calc}
\usetikzlibrary{arrows,shapes,trees,calc,automata,positioning, fit}
\usetikzlibrary{calendar}
\usetikzlibrary{backgrounds}
\usetikzlibrary{angles}
\usetikzlibrary{graphs}

\tikzset{
  mybox/.style={
    rectangle,
    rounded corners,
    draw=black
  },
}

\usepackage[auto-label]{exsheets}
\usepackage{enumerate}
\usepackage{tasks}
\usepackage{indentfirst}

\usepackage[bookmarks=true,pdfborder={0 0 0}]{hyperref}
%%去除TexLive-2015开始fontspec对引号的影响
\defaultfontfeatures[\rmfamily,\sffamily,\ttfamily]{}
\setmainfont{WenQuanYi Micro Hei}
\setsansfont{WenQuanYi Micro Hei}
\setmonofont{WenQuanYi Micro Hei}

%required Windows fonts
\newcommand\fontnamekai{STKaiti}
% Tex-Live2016
%\newfontinstance\KAI {\fontnamekai}
\newfontfamily\KAI {\fontnamekai}
\newcommand{\kai}[1]{{\KAI#1}}
%\newfontfamily\mynamingfont {Liberation Serif}
%\newfontfamily\mynamingfontb {Century Schoolbook L}
%\newfontfamily\mynamingfontc {Droid Serif}
%\newfontfamily\mynamingfontd [Mapping=tex-text]{TeX Gyre Pagella}


\newcommand\vtextvisiblespace[1][.4em]{%
  \mbox{\kern.06em\vrule height.5ex}%
  \vbox{\hrule width#1}%
  \hbox{\vrule height.5ex}}

\setlength{\parindent}{2em}
\setlength{\parskip}{0.5\baselineskip}
\XeTeXlinebreaklocale "zh"
\XeTeXlinebreakskip=0pt plus 1pt minus 0.1pt

%listing global settings
\lstset{basicstyle=\scriptsize,frame=lines}

%some global color
\definecolor{light-gray}{rgb}{0.87,0.87,0.87}
\definecolor{light-yellow}{rgb}{0.88,0.92,0.48}
\definecolor{mygreen}{rgb}{0.63,1,0.35}

%complicated def
\lstnewenvironment{myjavacode}[1][]
      {\lstset{language=Java}\lstset{escapeinside={(*@}{@*)},
       basicstyle=\footnotesize\ttfamily,
       numbers=left,numberstyle=\scriptsize,stepnumber=1,numbersep=5pt,
       breaklines=true,
       %firstnumber=last,
           %frame=tblr,
           framesep=5pt,
           showstringspaces=false,
           keywordstyle=\itshape\color{blue},
          %identifierstyle=\ttfamily,
           stringstyle=\xecolor{maroon},
        commentstyle=\color{black},
        rulecolor=\color{black},
        xleftmargin=0pt,
        xrightmargin=0pt,
        aboveskip=\medskipamount,
        belowskip=\medskipamount,
               backgroundcolor=\color{white}, #1
}}
{}

\newfontfamily\thekai{STKaiti}
%%尝试把margainpar中的文字大小修改
\setlength\marginparwidth{2cm}
\makeatletter
\long\def\@ympar#1{%
  \@savemarbox\@marbox{\thekai\footnotesize #1}%
  \global\setbox\@currbox\copy\@marbox
  \@xympar}
\makeatother

%% Try new a my own code 用于扩展词汇的统一处理
\newcounter{myexpandnum}
\newcounter{mydemonum}
%\newenvironment{myexpand2}[1][\unskip]{\colorbox{red}{Expand\refstepcounter{myexpandnum} \themyexpandnum #1}}{}

%\newenvironment{myexpand}[1][]{\refstepcounter{myexpandnum}  扩展词汇\themyexpandnum- \vspace{0.5cm} \tikz [baseline={([yshift=-.8ex]current bounding box.center)}] \node [signal, draw, text=white, fill=red!65!black, signal to=nowhere,signal from=east]{\large{ #1}};}{}

\newenvironment{myexpand}[1][]{\refstepcounter{myexpandnum} \tikz [baseline={([yshift=-.8ex]current bounding box.center)}] \shade [ball color=red] (0,0) circle (1ex); \colorbox{blue!30}{扩展词汇\themyexpandnum- \vspace{0.5cm}} \tikz [baseline={([yshift=-.8ex]current bounding box.center)}] \node [signal, draw, text=white, fill=red!65!black, signal to=nowhere,signal from=east]{\large{ #1}};}{}


\newenvironment{mydemosample}{\stepcounter{mydemonum} \tikz [baseline={([yshift=-.8ex]current bounding box.center)}] \shade [ball color=blue] (0,0) circle (1ex); \colorbox{blue!30}{例句 \themydemonum}
\\
\rule{0.2\textwidth}{0.5mm} \begin{quotation}}{\end{quotation}}

%%这个用于适合打印的例句输出
\newcounter{printdemonum}
\newenvironment{printdemosample}{\stepcounter{printdemonum} \tikz [baseline={([yshift=-.8ex]current bounding box.center)}] {\node[rectangle]{\includegraphics[scale=0.10]{../img/huawen.eps}}} {\kai{例句} \theprintdemonum{}} \vspace{-0.41cm} \begin{quotation}} {\end{quotation}}

% a table collection
%\definecolor{Gray}{gray}{0.85}
%\definecolor{LightCyan}{rgb}{0.88,1,1}
%\newcolumntype{mycA}{>{\columncolor{Gray}}c}
%\newcolumntype{mycB}{>{\columncolor{LightCyan}}c}

%\newcommand\mywordnums[2]{First #1 , second #2}
\newcommand\mywordnums[2]{\bgroup \def\arraystretch{1.15} \scriptsize
  \begin{tabular}{|>{\columncolor[gray]{0.8}}c|>{\columncolor[rgb]{0.88,1,1}}c|}
   \hline
   {单数} & {复数} \\
   \hline
   {#1} & {#2} \\
   \hline
  \end{tabular}
\egroup}

%% 2016-12-22添加了一个类似tip类型的示例
%来源:http://tex.stackexchange.com/questions/171951/how-was-this-tip-box-produced/171954#171954
%%
%% 若需要修改TIP的标签位置,修改 \coordinate (aux) at ( $ (O)!0.5!(P) $ );的中间0.5这个数值
\newcounter{myTipStyleDemoCounter}

\newmdenv[
hidealllines=true,
innertopmargin=16pt,
innerbottommargin=10pt,
font=\sffamily\footnotesize,
leftmargin=-0.5cm,
rightmargin=-0.5cm,
skipabove=35pt,
skipbelow=15pt,
singleextra={
  \coordinate (aux) at ( $ (O)!0.5!(P) $ );
  \fill[rounded corners=8pt,line width=1pt,gray!30]
    (O|-P) --
    (aux|-P) --
    ([yshift=20pt]aux|-P) --
    ([yshift=20pt,xshift=4cm]aux|-P) --
    ([xshift=4cm]aux|-P) --
    (P) {[sharp corners] --
    ([yshift=-6pt]P) --
    ([yshift=-6pt]O|-P) } -- cycle;
  \draw[rounded corners=8pt,line width=1pt,gray]
    (O|-P) --
    (aux|-P) --
    ([yshift=20pt]aux|-P) --
    ([yshift=20pt,xshift=4cm]aux|-P) --
    ([xshift=4cm]aux|-P) --
    (P) --
    (P|-O) --
    (O) -- cycle;
  \node at ([xshift=2cm,yshift=6pt]aux|-P)
    {\refstepcounter{myTipStyleDemoCounter}\sffamily\large 例句~\themyTipStyleDemoCounter} ;
  },
firstextra={
  \coordinate (aux) at ( $ (O)!0.5!(P|-O) $ );
  \fill[rounded corners=8pt,line width=1pt,gray!30,overlay]
    (O|-P) --
    (aux|-P) --
    ([yshift=20pt]aux|-P) --
    ([yshift=20pt,xshift=4cm]aux|-P) --
    ([xshift=4cm]aux|-P) --
    (P) {[sharp corners] --
    ([yshift=-6pt]P) --
    ([yshift=-6pt]O|-P) } -- cycle;
  \draw[rounded corners=8pt,line width=1pt,gray,overlay]
    (O) --
    (O|-P) --
    (aux|-P) --
    ([yshift=20pt]aux|-P) --
    ([yshift=20pt,xshift=4cm]aux|-P) --
    ([xshift=4cm]aux|-P) --
    (P) --
    (P|-O);
  \node[overlay] at ([xshift=2cm,yshift=6pt]aux|-P)
    {\refstepcounter{myTipStyleDemoCounter}\sffamily\large 例句~\themyTipStyleDemoCounter} ;
  },
middleextra={
  \draw[rounded corners=8pt,line width=1pt,gray,overlay]
    (O|-P) --
    (O);
  \draw[rounded corners=8pt,line width=1pt,gray,overlay]
    (P) --
    (P|-O);
  },
secondextra={
  \coordinate (aux) at ( $ (O)!0.5!(P|-O) $ );
  \draw[rounded corners=8pt,line width=1pt,gray,overlay]
    (O|-P) --
    (O) --
    (P|-O) --
    (P);
  },
]{myTipStyleDemo}

%% 2016-12-22继续扩展上面的环境,打算做一个词汇积累类型的新环境,这个目前打算用来做英语的扩展词汇用,最终目标
%是替换掉上面的自己写的简单的myexpand环境
%
%来源:http://tex.stackexchange.com/questions/171951/how-was-this-tip-box-produced/171954#171954
\newcounter{myExpandWordCounter}

\newmdenv[
hidealllines=true,
innertopmargin=16pt,
innerbottommargin=10pt,
font=\sffamily\small,
leftmargin=0.0cm,
rightmargin=0.0cm,
skipabove=35pt,
skipbelow=15pt,
singleextra={
  \coordinate (aux) at ( $ (O)!0.1!(P) $ );
  \fill[rounded corners=8pt,line width=1pt,blue!30]
    (O|-P) --
    (aux|-P) --
    ([yshift=20pt]aux|-P) --
    ([yshift=20pt,xshift=4cm]aux|-P) --
    ([xshift=4cm]aux|-P) --
    (P) {[sharp corners] --
    ([yshift=-6pt]P) --
    ([yshift=-6pt]O|-P) } -- cycle;
  \draw[rounded corners=8pt,line width=1pt,gray]
    (O|-P) --
    (aux|-P) --
    ([yshift=20pt]aux|-P) --
    ([yshift=20pt,xshift=4cm]aux|-P) --
    ([xshift=4cm]aux|-P) --
    (P) --
    (P|-O) --
    (O) -- cycle;
  \node at ([xshift=2cm,yshift=6pt]aux|-P)
    {\refstepcounter{myExpandWordCounter}\sffamily\large 扩展词汇~\themyExpandWordCounter} ;
  },
firstextra={
  \coordinate (aux) at ( $ (O)!0.5!(P|-O) $ );
  \fill[rounded corners=8pt,line width=1pt,blue!30,overlay]
    (O|-P) --
    (aux|-P) --
    ([yshift=20pt]aux|-P) --
    ([yshift=20pt,xshift=4cm]aux|-P) --
    ([xshift=4cm]aux|-P) --
    (P) {[sharp corners] --
    ([yshift=-6pt]P) --
    ([yshift=-6pt]O|-P) } -- cycle;
  \draw[rounded corners=8pt,line width=1pt,gray,overlay]
    (O) --
    (O|-P) --
    (aux|-P) --
    ([yshift=20pt]aux|-P) --
    ([yshift=20pt,xshift=4cm]aux|-P) --
    ([xshift=4cm]aux|-P) --
    (P) --
    (P|-O);
  \node[overlay] at ([xshift=2cm,yshift=6pt]aux|-P)
    {\refstepcounter{myExpandWordCounter}\sffamily\large 扩展词汇~\themyExpandWordCounter} ;
  },
middleextra={
  \draw[rounded corners=8pt,line width=1pt,gray,overlay]
    (O|-P) --
    (O);
  \draw[rounded corners=8pt,line width=1pt,gray,overlay]
    (P) --
    (P|-O);
  },
secondextra={
  \coordinate (aux) at ( $ (O)!0.5!(P|-O) $ );
  \draw[rounded corners=8pt,line width=1pt,gray,overlay]
    (O|-P) --
    (O) --
    (P|-O) --
    (P);
  },
]{myExpandWord}


%% marginpar for exercise
\SetupExSheets{headings=margin-nr}

%%
%% 寒假学习计划的临时环境
%来源:http://tex.stackexchange.com/questions/171951/how-was-this-tip-box-produced/171954#171954
%%
%% 若需要修改TIP的标签位置,修改 \coordinate (aux) at ( $ (O)!0.5!(P) $ );的中间0.5这个数值
\newcounter{myPlanCounter}

\newmdenv[
hidealllines=true,
innertopmargin=16pt,
innerbottommargin=10pt,
font=\sffamily\footnotesize,
leftmargin=0.0cm,
rightmargin=1.8cm,
skipabove=35pt,
skipbelow=15pt,
singleextra={
  \coordinate (aux) at ( $ (O)!0.1!(P) $ );
  \fill[rounded corners=8pt,line width=1pt,green!30]
    (O|-P) --
    (aux|-P) --
    ([yshift=20pt]aux|-P) --
    ([yshift=20pt,xshift=4cm]aux|-P) --
    ([xshift=4cm]aux|-P) --
    (P) {[sharp corners] --
    ([yshift=-6pt]P) --
    ([yshift=-6pt]O|-P) } -- cycle;
  \draw[rounded corners=8pt,line width=1pt,green]
    (O|-P) --
    (aux|-P) --
    ([yshift=20pt]aux|-P) --
    ([yshift=20pt,xshift=4cm]aux|-P) --
    ([xshift=4cm]aux|-P) --
    (P) --
    (P|-O) --
    (O) -- cycle;
  \node at ([xshift=2cm,yshift=6pt]aux|-P)
    {\refstepcounter{myPlanCounter}\sffamily\large 计划阶段~\themyPlanCounter} ;
  },
firstextra={
  \coordinate (aux) at ( $ (O)!0.5!(P|-O) $ );
  \fill[rounded corners=8pt,line width=1pt,gray!30,overlay]
    (O|-P) --
    (aux|-P) --
    ([yshift=20pt]aux|-P) --
    ([yshift=20pt,xshift=4cm]aux|-P) --
    ([xshift=4cm]aux|-P) --
    (P) {[sharp corners] --
    ([yshift=-6pt]P) --
    ([yshift=-6pt]O|-P) } -- cycle;
  \draw[rounded corners=8pt,line width=1pt,gray,overlay]
    (O) --
    (O|-P) --
    (aux|-P) --
    ([yshift=20pt]aux|-P) --
    ([yshift=20pt,xshift=4cm]aux|-P) --
    ([xshift=4cm]aux|-P) --
    (P) --
    (P|-O);
  \node[overlay] at ([xshift=2cm,yshift=6pt]aux|-P)
    {\refstepcounter{myPlanCounter}\sffamily\large 计划阶段~\themyPlanCounter} ;
  },
middleextra={
  \draw[rounded corners=8pt,line width=1pt,gray,overlay]
    (O|-P) --
    (O);
  \draw[rounded corners=8pt,line width=1pt,gray,overlay]
    (P) --
    (P|-O);
  },
secondextra={
  \coordinate (aux) at ( $ (O)!0.5!(P|-O) $ );
  \draw[rounded corners=8pt,line width=1pt,gray,overlay]
    (O|-P) --
    (O) --
    (P|-O) --
    (P);
  },
]{myPlan}


%%%%%%%%%%%%%%%%%%%%%%%%%%%%%%%%%%%%%%%%%%%%%%%%%%%%%%%%%%%%%
%% mySimpExpand
%%%%%%%%%%%%%%%%%%%%%%%%%%%%%%%%%%%%%%%%%%%%%%%%%%%%%%%%%%%%%

%% 2016-12-22继续扩展上面的环境,打算做一个词汇积累类型的新环境,这个目前打算用来做英语的扩展词汇用,最终目标
%也是替换掉上自己写的简单的myexpand环境
%% FROM - http://tex.stackexchange.com/questions/135871/what-are-the-relative-strong-and-weak-points-between-tcolorbox-and-mdframed
%\tcbset{colback=Salmon!50!white,colframe=FireBrick!75!black, width=(\linewidth-8mm)/2,before=,after=\hfill,equal height group=ske}
\newcounter{mySimpExpandCnt}
\newenvironment{mySimpExpand}[1][]
  {\refstepcounter{mySimpExpandCnt}\begin{tcolorbox}[
    colback=green!5,
    colframe=green!40!black,
    title=\large{扩展词汇-\themySimpExpandCnt : #1}]
  }
  {\end{tcolorbox}}

%% ReadIt(大家说英语节目的内容整理)
\newmdenv[
hidealllines=true,
innertopmargin=16pt,
innerbottommargin=10pt,
font=\sffamily\large,
leftmargin=0.0cm,
rightmargin=0.0cm,
skipabove=35pt,
skipbelow=15pt,
singleextra={
  \coordinate (aux) at ( $ (O)!0.1!(P) $ );
  \fill[rounded corners=8pt,line width=1pt,gray!10]
    (O|-P) --
    (aux|-P) --
    ([yshift=20pt]aux|-P) --
    ([yshift=20pt,xshift=4cm]aux|-P) --
    ([xshift=4cm]aux|-P) --
    (P) {[sharp corners] --
    ([yshift=-6pt]P) --
    ([yshift=-6pt]O|-P) } -- cycle;
  \draw[rounded corners=8pt,line width=1pt,gray]
    (O|-P) --
    (aux|-P) --
    ([yshift=20pt]aux|-P) --
    ([yshift=20pt,xshift=4cm]aux|-P) --
    ([xshift=4cm]aux|-P) --
    (P) --
    (P|-O) --
    (O) -- cycle;
  \node at ([xshift=2cm,yshift=6pt]aux|-P)
    {\sffamily Read It} ;
  },
firstextra={
  \coordinate (aux) at ( $ (O)!0.5!(P|-O) $ );
  \fill[rounded corners=8pt,line width=1pt,gray!10,overlay]
    (O|-P) --
    (aux|-P) --
    ([yshift=20pt]aux|-P) --
    ([yshift=20pt,xshift=4cm]aux|-P) --
    ([xshift=4cm]aux|-P) --
    (P) {[sharp corners] --
    ([yshift=-6pt]P) --
    ([yshift=-6pt]O|-P) } -- cycle;
  \draw[rounded corners=8pt,line width=1pt,gray,overlay]
    (O) --
    (O|-P) --
    (aux|-P) --
    ([yshift=20pt]aux|-P) --
    ([yshift=20pt,xshift=4cm]aux|-P) --
    ([xshift=4cm]aux|-P) --
    (P) --
    (P|-O);
  \node[overlay] at ([xshift=2cm,yshift=6pt]aux|-P)
    {\sffamily Read It} ;
  },
middleextra={
  \draw[rounded corners=8pt,line width=1pt,gray,overlay]
    (O|-P) --
    (O);
  \draw[rounded corners=8pt,line width=1pt,gray,overlay]
    (P) --
    (P|-O);
  },
secondextra={
  \coordinate (aux) at ( $ (O)!0.5!(P|-O) $ );
  \draw[rounded corners=8pt,line width=1pt,gray,overlay]
    (O|-P) --
    (O) --
    (P|-O) --
    (P);
  },
]{myReadList}

%% for keywords
\newmdenv[
hidealllines=true,
innertopmargin=14pt,
innerbottommargin=8pt,
font=\sffamily\normalsize,
leftmargin=0.0cm,
rightmargin=0.0cm,
skipabove=34pt,
skipbelow=13pt,
singleextra={
  \coordinate (aux) at ( $ (O)!0.5!(P) $ );
  \fill[rounded corners=4pt,line width=1pt,gray!10]
    (O|-P) --
    (aux|-P) --
    ([yshift=15pt]aux|-P) --
    ([yshift=15pt,xshift=3cm]aux|-P) --
    ([xshift=3cm]aux|-P) --
    (P) {[sharp corners] --
    ([yshift=-6pt]P) --
    ([yshift=-6pt]O|-P) } -- cycle;
  \draw[rounded corners=8pt,line width=1pt,gray]
    (O|-P) --
    (aux|-P) --
    ([yshift=15pt]aux|-P) --
    ([yshift=15pt,xshift=3cm]aux|-P) --
     ([xshift=3cm]aux|-P) --
    (P) --
    (P|-O) --
    (O) -- cycle;
  \node at ([xshift=2cm,yshift=6pt]aux|-P)
    {\sffamily read it} ;
  },
firstextra={
  \coordinate (aux) at ( $ (O)!0.5!(P|-O) $ );
  \fill[rounded corners=8pt,line width=1pt,gray!10,overlay]
    (O|-P) --
    (aux|-P) --
    ([yshift=15pt]aux|-P) --
    ([yshift=15pt,xshift=3cm]aux|-P) --
    ([xshift=3cm]aux|-P) --
    (P) {[sharp corners] --
    ([yshift=-6pt]P) --
    ([yshift=-6pt]O|-P) } -- cycle;
  \draw[rounded corners=8pt,line width=1pt,gray,overlay]
    (O) --
    (O|-P) --
    (aux|-P) --
    ([yshift=15pt]aux|-P) --
    ([yshift=15pt,xshift=3cm]aux|-P) --
    ([xshift=3cm]aux|-P) --
    (P) --
    (P|-O);
  \node[overlay] at ([xshift=2cm,yshift=6pt]aux|-P)
    {\sffamily read it} ;
  },
middleextra={
  \draw[rounded corners=8pt,line width=1pt,gray,overlay]
    (O|-P) --
    (O);
  \draw[rounded corners=8pt,line width=1pt,gray,overlay]
    (P) --
    (P|-O);
  },
secondextra={
  \coordinate (aux) at ( $ (O)!0.5!(P|-O) $ );
  \draw[rounded corners=8pt,line width=1pt,gray,overlay]
    (O|-P) --
    (O) --
    (P|-O) --
    (P);
  },
]{myWordList2}

%% for keywords
\newmdenv[
hidealllines=true,
innertopmargin=14pt,
innerbottommargin=8pt,
font=\sffamily\footnotesize,
leftmargin=0.0cm,
rightmargin=0.0cm,
skipabove=34pt,
skipbelow=13pt,
singleextra={
  \coordinate (aux) at ( $ (O)!0.5!(P) $ );
  \fill[rounded corners=4pt,line width=1pt,gray!10]
    (O|-P) --
    (aux|-P) --
    ([yshift=15pt]aux|-P) --
    ([yshift=15pt,xshift=3cm]aux|-P) --
    ([xshift=3cm]aux|-P) --
    (P) {[sharp corners] --
    ([yshift=-6pt]P) --
    ([yshift=-6pt]O|-P) } -- cycle;
  \draw[rounded corners=8pt,line width=1pt,gray]
    (O|-P) --
    (aux|-P) --
    ([yshift=15pt]aux|-P) --
    ([yshift=15pt,xshift=3cm]aux|-P) --
     ([xshift=3cm]aux|-P) --
    (P) --
    (P|-O) --
    (O) -- cycle;
  \node at ([xshift=2cm,yshift=6pt]aux|-P)
    {\sffamily Keywords} ;
  },
firstextra={
  \coordinate (aux) at ( $ (O)!0.5!(P|-O) $ );
  \fill[rounded corners=8pt,line width=1pt,gray!10,overlay]
    (O|-P) --
    (aux|-P) --
    ([yshift=15pt]aux|-P) --
    ([yshift=15pt,xshift=3cm]aux|-P) --
    ([xshift=3cm]aux|-P) --
    (P) {[sharp corners] --
    ([yshift=-6pt]P) --
    ([yshift=-6pt]O|-P) } -- cycle;
  \draw[rounded corners=8pt,line width=1pt,gray,overlay]
    (O) --
    (O|-P) --
    (aux|-P) --
    ([yshift=15pt]aux|-P) --
    ([yshift=15pt,xshift=3cm]aux|-P) --
    ([xshift=3cm]aux|-P) --
    (P) --
    (P|-O);
  \node[overlay] at ([xshift=2cm,yshift=6pt]aux|-P)
    {\sffamily Keywords} ;
  },
middleextra={
  \draw[rounded corners=8pt,line width=1pt,gray,overlay]
    (O|-P) --
    (O);
  \draw[rounded corners=8pt,line width=1pt,gray,overlay]
    (P) --
    (P|-O);
  },
secondextra={
  \coordinate (aux) at ( $ (O)!0.5!(P|-O) $ );
  \draw[rounded corners=8pt,line width=1pt,gray,overlay]
    (O|-P) --
    (O) --
    (P|-O) --
    (P);
  },
]{myWordList}


%% 下面这个适合定理的风格

%% FROM - http://tex.stackexchange.com/questions/135871/what-are-the-relative-strong-and-weak-points-between-tcolorbox-and-mdframed
%\newcounter{mySimpExpandCounter}
%\tcbuselibrary{theorem}
%\newtcbtheorem[number within=section]{mySimpExpand22222}{My Theorem}%
%{colback=green!5,colframe=green!35!black,fonttitle=\bfseries}{th}



%\titleformat{\section}[block]{\kai}{\thesection}{10pt}{}

%% 2015-10-18 Try use glossaries packages for English Words Annotation
%\input words.tex

%\makeglossaries

\makeindex

\begin{document}

\maketitle

\tableofcontents

\begin{abstract}
这是语法文档的\XeLaTeX{}格式输出。
\end{abstract}

%%% start new page from the first
\newpage

\section{小学英语语法整理}\label{sec.JuniorEng}
%%%%%%%%%%%%%%%%%%%%%%%%%%%%%%%%%%%%%%%%%%%%%%%%%%%%%%%%%%%%%%%%%%%%%%%
%这里打算开始重构整个文档, 将学习过程中的琐碎知识点都一个个地整理起来,不追求完整性
%%%%%%%%%%%%%%%%%%%%%%%%%%%%%%%%%%%%%%%%%%%%%%%%%%%%%%%%%%%%%%%%%%%%%%%
  \subsection{excited和exciting}
这类词汇是小学经常考察的知识点:

\begin{itemize}
 \item $\sim$ ed多指"感到$\cdots$的", 所以多用来表示人。
 \item $\sim$ ing多指"让人感到$\cdots$的", 所以多用来表示物。
\end{itemize}

\begin{printdemosample}
 She was frightened of flying.
\end{printdemosample}

\begin{printdemosample} \marginpar{be interested in}
 This new book is very interesting, I'm interested in it.
\end{printdemosample}

\begin{printdemosample}
  I felt relaxed when I heard the relaxing music.
\end{printdemosample}

\begin{tasks}[style=enumerate](3)
 \task interesting  \task exciting  \task surprising
 \task interested   \task excited   \task suprised
 \task frightening  \task disappointing \task relaxing
 \task frightened   \task disappointed  \task relaxed
\end{tasks}

试着完成下面的试题:

\begin{question}
 My father is very \blank[width=1cm]{} to watch the \blank[width=1cm]{} basketball match.

  \begin{tasks}[counter-format=tsk[A].](3)
   \task excited, excited
   \task exciting, excited
   \task excited, exciting
   \end{tasks}
\end{question}
\begin{solution}
C
\end{solution}

\begin{question}
  It is \blank[width=1cm]{} that he didn't pass the examination.

   \begin{tasks}[counter-format=tsk[A].](2)
    \task disappointed
    \task disappointing
    \end{tasks}
 \end{question}
 \begin{solution}
 B
 \end{solution}

%%%%%%%%%%%%%%%%%%%%%%%%%%%%%%%%%%%%%%%%%%%%%%%%%%%%%

  \subsection{关于帮助(help)}

首先,请人帮忙的说法有好多种,需要至少熟悉下面几种:

\begin{printdemosample}
 - Would you please do me a favour?

 - Sure, what can I do for you?
\end{printdemosample}

或者:

\begin{printdemosample}
 - Could you please do a favor for me?

 - Sure, how may I help you?
\end{printdemosample}

give $\cdots$ a hand也表示帮忙的意思:\marginpar{此为口语化表达}

\begin{printdemosample}
 Give me a hand, please.
\end{printdemosample}

\begin{printdemosample}
 Come and give me a hand in the garden.
\end{printdemosample}

help单独使用,则表示中文中的"救命"的求救意思:

\begin{printdemosample} \marginpar{cry $\,$ cries}
 An ant falls into a river, "Help!", he cries.
\end{printdemosample}

当“有助于$\cdots$”讲的时候,可以这么说:

\begin{printdemosample}
 Some dinosaurs left fossils. Fossils help us learn about them.
\end{printdemosample}

\begin{printdemosample}  \index{to} \index{will} \index{help}
 This is a baby shark. It has eyes to help it see.

 It has a tail to help it swim.

 It has a mouth to help it eat.

 It will grow into a big shark.
\end{printdemosample}

询问别人需要帮助(比如商店购物等),需要知道下面的两种问法:

\begin{printdemosample}
 What can I do for you?
\end{printdemosample}

\begin{printdemosample}
 How may I help you?
\end{printdemosample}

  \subsection{less和ness后缀}

\begin{itemize}
  \item \~less : 一般代表否定的意思
  \item \~ness : 一般是对应单词的名词形式
\end{itemize}

homeless

careless

%%%%%%%%%%%%%%%%END OF REFACTORING ...%%%%%%%%%%%%%%%%%%%%%%%%%%%%%%%%%%%%%


  \subsection{天气情况的询问} \label{sec.Weather}\index{weather}

最常见的是下面两种问法:

\begin{itemize}
 \item How's the weather?
 \item What's the weather like?
\end{itemize}

\begin{printdemosample}
 - How's the weather?

 - It's cloudy.
\end{printdemosample}


\begin{tasks}(5)
 \task windy  \task rainy  \task sunny  \task cloudy  \task snowy
 \task wind   \task rain   \task sun    \task cloud   \task snow
\end{tasks}

\begin{printdemosample} \marginpar{look for}
 When the rain comes, ants look for shelter. But ducks stay in the rain.
\end{printdemosample}

\begin{printdemosample}
 - What is the weather like in your hometown?

 - It is always sunny.
\end{printdemosample}


  \subsection{aw发音} \index{draw} \index{drawer}

这个多数情况下发\begin{IPA}[O:]\end{IPA}音

\begin{tasks}(4)
 \task draw  \task saw  \task paw  \task claw  \task hawk
 \task awful \task awesome  \task drawer \task strawberry
\end{tasks}

\begin{printdemosample}
 That's awful!
\end{printdemosample}

\begin{printdemosample}
 Wow! That's totally awesome!
\end{printdemosample}

\begin{printdemosample}\marginpar{可以尝试一下不同时态的说法!}
 He opened a drawer in his writing-table.
\end{printdemosample}

\begin{printdemosample}
 He doesn't know how to paint a lion's paw.
\end{printdemosample}


 \subsection{单词like}\label{sec.like} \index{like}
喜欢做某事的两种说法:

\begin{tasks}[style=enumerate](1)
 \task sb like to do sth.
 \task sb like doing sth.
\end{tasks}

需要注意这两种说法的细微区别

\begin{printdemosample}
 She likes watching TV everyday.
\end{printdemosample}

\begin{printdemosample}
 She likes to swim this afternoon.
\end{printdemosample}

而直接用like sth,通常表示喜欢某个事物、或者像某个事物: (这个有两种含义)

\begin{printdemosample}
 He looks like a tiger.
\end{printdemosample}

\begin{printdemosample}
 I like pandas because they're cute.
\end{printdemosample}

\begin{printdemosample}
 He likes pandas because they're so lovely.
\end{printdemosample}

\begin{printdemosample}
 She liked pandas when she was a child. Now, she doesn't like them any more.
\end{printdemosample}

另外一个比较客气的说法是:

\begin{printdemosample}
 I'd like to watch T.V. with my mother.
\end{printdemosample}

\begin{printdemosample}
 I would like to watch T.V. with my mother.
\end{printdemosample}

\begin{printdemosample}
 I'd like some juice.
\end{printdemosample}

和like to do相比, would like to更加强调自己的想要去做某件事情

下面留几个小练习:

\begin{question}
- Do you like playing football?

\noindent - No, I don't. I like \blank[width=1cm]{}.

\begin{tasks}[counter-format=tsk[A].](3)
 \task run
 \task skipping
 \task swims
\end{tasks}
\end{question}
\begin{solution}
B
\end{solution}

\begin{question}
 - What do you like doing?

\noindent - I like \blank[width=1cm]{}.

\begin{tasks}[counter-format=tsk[A].](3)
 \task swim
 \task swimming
 \task swims
\end{tasks}
\end{question}
\begin{solution}
B
\end{solution}

 \subsection{has got/have got}  \marginpar{学而思三年级暑假内容}
这个一般表示"已经得到、有"的意思。

当表示"有"的意思时,基本和单独的have/has表达的意思一样。

\begin{printdemosample}
 Miss, Miss, Bill has got my hat!
\end{printdemosample}

\begin{printdemosample}  \marginpar{知道什么时候是have,什么时候是has}
 I have got some books.

 I haven't got any books.
\end{printdemosample}

\begin{printdemosample}
 - What's the matter?

 - I've got a bad cold.
\end{printdemosample}

\begin{printdemosample}  \marginpar{疑问句和否定句的说法}
 - Have you got a cold?

 - No, I haven't.
\end{printdemosample}


  \subsection{str的发音}

\begin{tasks}[style=enumerate](3)
 \task street    \task strong        \task strange
 \task string    \task strawberries   \task stranger
\end{tasks}

在街道上的说法: \index{on the street}  \index{in the street}

in the street和on the street一般都可以\marginpar{in适合于英国,on适用于美国}

\begin{printdemosample}
We mustn't play on the street
\end{printdemosample}

\begin{printdemosample} \marginpar{by chance}
 I met her by chance on the street yesterday.
\end{printdemosample}


  \subsection{want和want to} \index{want} \index{want to}
这里需要知道下面三种用法:

\begin{itemize}
 \item want to do sth
 \item want sth
 \item want sb to do sth
\end{itemize}

\begin{printdemosample}
I want to have a talk with him alone.
\end{printdemosample}

\begin{printdemosample}
 Do you want some tea?

 Do you want some more tea?
\end{printdemosample}

\begin{printdemosample}
I want you to buy a cup of coffee for me right now.
\end{printdemosample}

  \subsection{be good at}
这个单词表示"擅长做$\cdots$"

\begin{itemize}
 \item be good at sth
 \item be good at doing sth.
\end{itemize}


\begin{printdemosample}
 Lucy is good at swimming.
\end{printdemosample}

\begin{printdemosample}
 James was good at running when he was a child. Now, he is good at jumping.
\end{printdemosample}

\begin{printdemosample}
 I am good at chess, but not checkers.
\end{printdemosample}

 \subsection{be going to}
将要做什么事情: sb be going to do sth.

相当于sb will do sth.

\begin{printdemosample}
 Don't worry, this is going to be fine.
\end{printdemosample}

\begin{printdemosample}
 They're going to buy some books tomorrow.
\end{printdemosample}

\begin{printdemosample}
 The weather is getting windy and cloudy. It's going to rain.
\end{printdemosample}

  \subsection{一周7天} \marginpar{周一到周日的书写要贯穿到平时的天天练当中!}

\begin{printdemosample}
 There are seven days in a week.
\end{printdemosample}

各个周的称呼:

\begin{tasks}[counter-format=tsk[1])](4)
 \task Monday
 \task Tuesday
 \task Wednesday
 \task Thursday
 \task Friday
 \task Saturday
 \task Sunday
 \end{tasks}

习惯地记住"在星期$\cdots$”是用on $\cdots$:

\begin{printdemosample}  \marginpar{注意当中的细微区别!}
 Lucy goes to school on Thursday.

 We go to school on Wednesdays.
\end{printdemosample}

上面的on Wednesdays就是"每个星期三",我们可以使用两种方式:

\begin{printdemosample}
 We go to school on Wednesdays.

 We go to school every Wednesday.
\end{printdemosample}


周末 - 为week和end的结合: weekend

在周末的说法,通常是可以用两种方式:\marginpar{at为英国用法,on为美国用法}

\begin{tasks}[counter-format=tsk[1])](2)
 \task at the weekend
 \task at weekends
 \task on the weekend
 \task on weekends
 \end{tasks}

  \subsection{some和any} \marginpar{学而思培训内容}
否定、疑问下,some要变any

\begin{printdemosample}
 Lele has some tomatoes.
\end{printdemosample}

\begin{printdemosample}
 Does Lele have any tomatoes?
\end{printdemosample}

\begin{printdemosample}
 There are some potatoes under the table.
\end{printdemosample}

\begin{printdemosample}
 Are there any potatoes under the table?

 There aren't any potatoes under the table.
\end{printdemosample}

\begin{printdemosample}   \marginpar{剑桥二知识点}
 Can you find any differences between these two pictures?
\end{printdemosample}

\begin{printdemosample}
 - Let's find some buses, can you see number 35?

 - No, I can't see any buses!
\end{printdemosample}

注意一个非常有用的not $\cdots$ any more, 或者not $\cdots$ any longer的用法: \marginpar{及其有用的表达方式!}

\begin{printdemosample}
 I went to kindergarten When I was a young child. But now, I don't go there any more.
\end{printdemosample}

\begin{printdemosample}
 He doesn’t live here any longer.
\end{printdemosample}

  \subsection{问年龄/数量/价格}
这里列举日常中经常使用的询问句型:

\begin{printdemosample} \marginpar{问年龄、岁数}
 - How old are you?

 - I am eight years old.
\end{printdemosample}

\begin{printdemosample}
 - How old is your mother?

 - She is twenty-seven.
\end{printdemosample}

\begin{printdemosample}  \marginpar{问数量}
 - How many birds can you see?

 - I can see forty-nine birds.
\end{printdemosample}

\begin{printdemosample}
 - How many books do you have?

 - Thirty-six.
\end{printdemosample}

\begin{printdemosample} \marginpar{in all}
 Tom has two cats.

 Peter has five cats.

 How many cats do they have in all?
\end{printdemosample}


\begin{printdemosample} \marginpar{问价格}
 - How much is it?

 - Five yuan.
\end{printdemosample}

\begin{printdemosample}  \marginpar{dollar}
 - How much is it?

 - \$20.
\end{printdemosample}

  \subsection{there be句型}
基础部分:记得$> 1$的情况下用are

\begin{printdemosample}
 There is a lamp next to the bed.

 There are some eggs in the basket.

 There are a lot of pencils on the desk.

 There is a piece of cake in the refrigerator.
\end{printdemosample}

对于有多个事物一起的情况下,有最近的事物决定使用is还是are: \marginpar{近视眼!}

\begin{printdemosample}
 There is a pen and two books on the desk.

 There are two books, a pen and many pencils on the table.

 Lucy finds a big box. She opens it. Inside, there's a dragon and some paints.
\end{printdemosample}

难点: 对于不可数的液体(milk, juice, water, $\cdots$),由于其没有s的形式,故还用is:

\begin{printdemosample}
 There is some milk in the cup.

 Is there any milk in the cup?
\end{printdemosample}

\begin{printdemosample}
 There is some meat in the fridge.
\end{printdemosample}

people是一个非常特殊的单词,它代表的是一个人的集体,虽然没有s的复数形式,但我们通常还用are:

\begin{printdemosample}
 There are some people in this bookstore.
\end{printdemosample}


尝试下面的小练习:

\begin{question}  \marginpar{2018博亚杯}
 \blank*[width=1cm]{} a lot of rain in New York in summer.
 \begin{tasks}[counter-format=tsk[A].](3)
 \task There was
 \task There were
 \task There is
 \end{tasks}
\end{question}
\begin{solution}
%placeholder
\end{solution}

  \subsection{difference和different}  \marginpar{剑二题型}
这两个单词一个是名词,一个是形容词

\begin{printdemosample}
 - Can you find any differences?

 - Yes, I can see something different.
\end{printdemosample}

\begin{printdemosample}
 There's no difference in the results.
\end{printdemosample}

和”某某不同“,我们多使用be different to/from的方式:

\begin{printdemosample}
 Paul's very different from his brother.

 Paul is very different to his brother.
\end{printdemosample}

注意除了表示不同外,difference还是用来表示数序中减法的结果,也就是我们说的"差"

\begin{printdemosample}
 \ul{Difference} is the result of a subtraction.
\end{printdemosample}

\begin{printdemosample}
 The difference of five and two is three, that is, \[5 - 2 = 3\]
\end{printdemosample}

 练习一题: %%放在这里,而不是放在some/any话题下,是因为打印纸的范围问题,只能放在这里

 \begin{question} \marginpar{这里有陷阱!}
 I don't have \blank[width=1cm]{} ice cream.

 \begin{tasks}[counter-format=tsk[A].](3)
   \task a
   \task any
   \task some
   \end{tasks}
\end{question}
\begin{solution}
%placeholder
\end{solution}

除了上面的陷阱例题外,我们还要注意,当是客气地询问别人需要吃、喝什么东西的时候,我们是希望对方肯定的回答的,所以也是用some.

所以是Would you like some xxxxxx?

\begin{question}
 - Would you like \blank[width=1cm]{} water?

 - No, I don't need \blank[width=1cm]{} water, thank you.
 \begin{tasks}[counter-format=tsk[A].](3)
   \task a
   \task any
   \task some
   \end{tasks}
\end{question}


  \subsection{borrow和lend}
借的方向不一样,所以我们都是用:

\begin{itemize}
 \item borrow $\cdots$ from $\cdots$
 \item lend $\cdots$ to $\cdots$
\end{itemize}

\begin{printdemosample}
 Members can borrow up to ten books from the library at any one time.
\end{printdemosample}

\begin{printdemosample} \marginpar{顺便学习下几个钱币的表示方式}
 She borrowed \pounds100 from her parents.

 She borrows \$10 from her parents every Monday.
\end{printdemosample}

\begin{printdemosample}
 Can you lend me your car this evening?
\end{printdemosample}

  \subsection{序数词}
英语中对于表达"第$\cdots$个"的意思,不是简单的数字,而是下面的序数词: \marginpar{second还可表示时间秒}

\begin{tasks}[style=enumerate, label-offset=0.5em, label-align=right](5)
 \task first  \task second   \task third  \task fourth \task fifth
 \task sixth  \task seventh  \task eighth  \task ninth  \task tenth
 \task eleventh \task twelfth \task thirteenth  \task fourteenth \task fifteenth
\end{tasks}

表达第几个,经常在上面的序数词前面加上一个the:

\begin{printdemosample}
 Alice finds two cats. The first one is white, and the second one is black.
\end{printdemosample}

短语a second opinion, 意思指"别的(其他人的)建议/诊断"。\marginpar{a second opinion}

该短语多用于多用于医疗场景下面,比如某个医生得出某个诊断结果后,一般人若不愿意接收,那么总是想再找另外的医生(second)来确认、查看。这个时候就可以说:

\begin{printdemosample}
 Let's get a second opinion.
\end{printdemosample}

实际场景中,还有序数词的缩写方式,要了解:

\begin{tasks}[style=enumerate](4)
 \task 1st        \task 2nd     \task 3rd   \task 4th
 \task 1$\displaystyle ^{st}$  \task  2$^{nd}$  \task 3$^{rd}$  \task 4$^{th}$
\end{tasks}

  \subsection{tw发音}
这个用的最多的是几个数字(包含2的):

\begin{tasks}[style=enumerate](4)
 \task twelve  \task twenty  \task twice  \task twentieth \task twelfth
\end{tasks}

别的单词:

\begin{tasks}[style=enumerate](4)
 \task twin  \task tweet
\end{tasks}

  \subsection{gram和kilogram}
这个与三年级上的数学内容有关,所以需要学习以下。 \marginpar{kilo-的前缀表示千}

\begin{tasks}[style=enumerate](3)
 \task gram        \task meter      \task metre
 \task kilogram  \task kilometer  \task kilometre
\end{tasks}

理解了这个单词,那么数学中的单位缩写(g, kg, m, km)就都能够理解,而不是死记硬背! \marginpar{看到k头脑中立刻想到thousand(1000)}

Wikipedia对千克的定义解释:
\begin{quote}
 The {\large{kilogram}{}} or {\large{kilogramme}{}} (symbol: kg) is the base unit of mass in the International System of Units (SI)
\end{quote}

下面是美国数学测试题目:

\begin{question}
 How many grams are in 95 kg?

 \begin{tasks}[counter-format=tsk[A].](4)
 \task $9.5$ g
 \task $9500$ g
 \task $0.95$ g
 \task $95000$ g
 \end{tasks}

\end{question}
\begin{solution}
%placeholder
\end{solution}
\begin{question}
 How many grams are in forty-eight kg? \qquad \blank[width=2cm]{} g
\end{question}
\begin{solution}
%placeholder
\end{solution}
  \subsection{打电话场景}

\begin{printdemosample}
 - May I speak to Sucy?

 - Speaking
\end{printdemosample}

\begin{printdemosample}\marginpar{和上句意思一样}
 - May I speak to Sucy?

 - This is Sucy.
\end{printdemosample}

\begin{printdemosample}
 - May I speak to Mr Gates?

 - He's not here right now. He's out.
\end{printdemosample}

\begin{printdemosample} \marginpar{Don't hang up}
 - May I speak to Mrs Sucy?

 - Hold on, please.
\end{printdemosample}


\begin{printdemosample}
 - What's your phone number?

 - My phone number is 18913381721.
\end{printdemosample}

make a phone call

\begin{printdemosample} \marginpar{phone call}
 If you make a phone call, you dial someone's phone number and speak to them by phone.
\end{printdemosample}

在电话里讲$\cdots$, 注意这个和中文的说法是不同的, 我们要用on the phone:

\begin{printdemosample}
 They talked about something on the phone last night.
\end{printdemosample}

\begin{printdemosample}
 William is talking on his phone.
\end{printdemosample}


  \subsection{问频率how often} \label{sec.HowOften}
首先,要熟悉下面几个表示频率的词:

\begin{tasks}[style=enumerate, label-offset=1em, label-align=right](4)
 \task always        \task often    \task usually   \task never
 \task ever          \task seldom   \task sometimes
\end{tasks}

\begin{printdemosample}
 Mum never sings, but she hums all day long.
\end{printdemosample}

\begin{printdemosample}
 We usually go to school by bicycle.
\end{printdemosample}

\begin{printdemosample}
 They seldom speak.
\end{printdemosample}

\begin{printdemosample} \marginpar{有时候我觉得自己快要疯掉了}
 Sometimes I feel I'm losing my mind.
\end{printdemosample}

表示次数的单词:

\begin{tasks}[style=enumerate, label-offset=1em, label-align=right](3)
 \task one time        \task two times    \task three times
 \task once            \task twice   \task four times
\end{tasks}

\begin{printdemosample}
 - How often does he play tennis?

 - He plays tennis every day.
\end{printdemosample}

\begin{printdemosample}
 - How often do you drink milk?

 - I drink milk twice a day.
\end{printdemosample}

如果只是问一共有多少次,那么可以简单地使用how many times

\begin{printdemosample}
 - How many times did you read this book last night?

 - Five times.
\end{printdemosample}


\begin{question}
 How \blank[width=1cm]{} people are there in the room?
 \begin{tasks}[counter-format=tsk[A].](3)
   \task many
   \task much
   \task long
   \end{tasks}

\end{question}
\begin{solution}
%placeholder
\end{solution}
people是集体名词,不是一个不可数名词,所以不用how much.

我们也通常用many people bla bla bla.的说法

  \subsection{比较级和最高级}
$A$比$B$更加$\cdots$的说法需要知道:

$A$ be $\cdots$er than $B$

\begin{printdemosample}
 Tigers are bigger than cats.

 The duck is louder than the cat.
\end{printdemosample}

比较级别(更$\cdots$)的加$\sim$er有几种不同的添加方式:

$\bigtriangleup$ 直接加er:

\begin{tasks}[style=enumerate](4)
 \task loud        \task strong      \task tall     \task quiet
 \task louder      \task stronger    \task taller   \task quieter
\end{tasks}

\begin{printdemosample}
 The hedgehog is smaller than the lion.

 The monkey is weaker than the elephant.
\end{printdemosample}

$\bigtriangleup$单词已经是e结尾,所以只要加r:

\begin{tasks}[style=enumerate](2)
 \task nice      \task large
 \task nicer     \task larger
\end{tasks}

$\bigtriangleup$单词y结尾,y$\rightarrow$i, 再加er:

\begin{tasks}[style=enumerate](4)
 \task happy        \task pretty      \task ugly     \task easy
 \task happier      \task prettier    \task uglier   \task easier
\end{tasks}

\begin{printdemosample}
 The girl is happier than her brother.

 Parrots are prettier than mice.

 The monster is uglier than the parrot.
\end{printdemosample}

$\bigtriangleup$末尾字母双拼后, 再加er: \marginpar{辅元辅的结构}

\begin{tasks}[style=enumerate](4)
 \task big         \task fat       \task hot     \task thin
 \task bigger      \task fatter    \task hotter  \task thinner
\end{tasks}

除了加$\sim$er外,比较级还可以使用more $\cdots$,这个主要用于字母比较长的单词,当然一些短单词也可以:

\begin{tasks}[style=enumerate](3)
 \task fun          \task famous         \task important
 \task more fun     \task more famous    \task more important
\end{tasks}

\begin{tasks}[style=enumerate](3)
 \task boring          \task clever         \task difficult
 \task more boring     \task more clever    \task more difficult
\end{tasks}

\begin{printdemosample}
  He wants to be a pilot. It would be more exciting than being a footballer.
\end{printdemosample}

\begin{printdemosample}
 I think Chinese is more difficult than Maths.
\end{printdemosample}


最高级,表示“最$\cdots$的”,用于大于2个以上的比较,注意两个事物比较,只能产生比较级,不能为最高级!

$\cdots$ be the xxxxx--est in $\cdots$

\begin{printdemosample}
 Nobody is taller than me in our class, so, I am the tallest in our class.
\end{printdemosample}

\begin{printdemosample}
 Jim is stronger than Carry. Carry is stronger than William. So, Jim is the strongest among them.
\end{printdemosample}

\begin{printdemosample}
  He is the youngest and tallest boy in his class.
\end{printdemosample}

这里需要稍微注意多数的最高级,前面经常加上the。

加$\sim$est构成最$\cdots$的规则和加$\sim$er是类似的:

$\bigtriangleup$ 直接加est:

\begin{tasks}[style=enumerate](4)
 \task loud        \task strong      \task tall     \task quiet
 \task louder      \task stronger    \task taller   \task quieter
 \task loudest     \task strongest   \task tallest   \task quietest
\end{tasks}

$\bigtriangleup$末尾字母双拼后, 再加est:

\begin{tasks}[style=enumerate](4)
 \task big         \task fat        \task hot     \task thin
 \task bigger      \task fatter    \task hotter   \task thinner
 \task biggest     \task fattest   \task hottest  \task thinnest
\end{tasks}

对于需要使用more来表示比较级的情况,则可用most来表示最高级:

\begin{tasks}[style=enumerate](3)
 \task fun          \task famous         \task important
 \task most fun     \task most famous    \task most important
\end{tasks}

\begin{printdemosample}
  He is the youngest and tallest boy in his class.
\end{printdemosample}


最后要记住几个特殊的比较级和最高级:

\begin{tasks}[style=enumerate, label-offset=1em, label-align=right](4)
 \task good         \task bad/ill    \task many/much   \task little
 \task better      \task worse   \task more             \task less
 \task best        \task worst   \task most	   \task least
\end{tasks}

\begin{printdemosample}
 - How do you feel, Susie?

 - Much better
\end{printdemosample}

\begin{printdemosample}
 I am the best!
\end{printdemosample}

\begin{printdemosample}
 Most of us like tigers very much.
\end{printdemosample}

\begin{printdemosample}
 The weather got worse during the day.
\end{printdemosample}


  \subsection{as ... as}
表示和$\cdots$一样$\cdots$

\begin{printdemosample}
 "Run, run, as fast as you can. You can't catch me. I am the gingerbread man."
\end{printdemosample}

\begin{printdemosample}
 This book is as interesting as that one.
\end{printdemosample}

\begin{printdemosample}
 I think I am as clever as you!
\end{printdemosample}

\begin{printdemosample} \marginpar{ASAP}
 "Bob, finish your homework as soon as possible!"
\end{printdemosample}

这里面的as soon as 在中文里面就是一$\cdots$就$\cdots$

\begin{printdemosample}
 As soon as I came into the classroom, I gave my homework to Mrs. Chen.
\end{printdemosample}

\begin{printdemosample}
 I'll telephone you as soon as I get home.
\end{printdemosample}

一个有名的缩写ASAP的意思:

\begin{tikzpicture}
 \node [] {as soon as possible};
 \node at (4, 0) [] {ASAP};
 \draw [->, >=stealth'] (2, 0) -- (3, 0);
\end{tikzpicture}

\begin{printdemosample}
 The colonel ordered, "I want two good engines down here asap."
\end{printdemosample}


  \subsection{情态动词}
这种单词要和动词的原型一起使用,多表示“可不可以,能不能,应不应该”等意思,常见的情态动词如下:

\begin{tasks}[style=enumerate, label-offset=1em, label-align=right](4)
 \task can         \task may    \task must   \task should    \task shall
 \task will      \task need   \task would
 \task could     \task might
\end{tasks}

\begin{printdemosample}
 She must stay here, and you mustn't stay here.
\end{printdemosample}

\begin{printdemosample}
 We can't carry this heavy box.
\end{printdemosample}

\begin{printdemosample}
 You should finish your homework as soon as possible.
\end{printdemosample}

\begin{printdemosample}
 I'm sorry I can't help you.
\end{printdemosample}

\begin{printdemosample}
 May I take your order now?
\end{printdemosample}


另外,have to, has to也是用来表示“必须”的意思,和must比较起来,have to/has to强调的是客观原因,而must多表示人的主观因素!

\begin{printdemosample}
 We have to go to school before 8 o'clock.
\end{printdemosample}

不允许做什么事情,我们多用mustn't来表达:

\begin{printdemosample}
 We mustn't play on the street.
\end{printdemosample}


  \subsection{健康-health}
当然,首先要知道health和healthy这两个词的区别:

\begin{printdemosample}
 His health is good.

 He is healthy.
\end{printdemosample}

\begin{printdemosample}
 Your health is not good, please keep healthy.
\end{printdemosample}

当某人看起来不舒服,关心地询问,可以用下面说法:

\begin{printdemosample} \marginpar{或者What's the matter with you?}
 - What the matter?

 - My feet hurt.
\end{printdemosample}

\begin{printdemosample}
 - What's wrong with you?

 - My foot hurts.
\end{printdemosample}

当然,平常见面的普通问候则是:

\begin{printdemosample}
 - How are you Sucie?

 - I'm tired, I didn't sleep well.
\end{printdemosample}

身体部位疼痛,我们通常使用$\sim$ache的结尾词语表示:

\begin{tasks}[style=enumerate, label-offset=1em, label-align=right](4)
 \task headache  \task toothache  \task stomachache
\end{tasks}

\begin{printdemosample}
 The patient's stomachache stopped after he took the medicine.
\end{printdemosample}

病情康复,可用get well.

"感冒"的说法通常有下面两种:

\begin{tasks}[style=enumerate, label-offset=1em, label-align=right](2)
 \task catch a cold    \task have a cold
\end{tasks}

下面一段文字来自Let's Talk in English: \marginpar{2017.0313.mp3}

  \begin{quotation}
Taylor needs help. She is moving a desk. Rob can help her. Be careful! Some books are falling. A book drops on Rob's foot. Now he is hurt. His leg is hurt, too. Taylor can help. But Rob has to stop hopping around first! \marginpar{stop doing sth\\stop to do sth}

Linda's tooth hurts. How often does she go to the dentist? Never! Linda is scared of dentists. But she needs a dentist today.
  \end{quotation}

对应的对话场景如下:

\begin{multicols}{2}
 {\scriptsize{Taylor:}{}} Can you help me move this desk?

 {\scriptsize{Rob:}{}} Sure, Taylor. One, two, three...

 {\scriptsize{Taylor:}{}} Oh, be careful. The books are dropping on the floor.

 {\scriptsize{Rob:}{}} Ahh. A book hit my foot!

 {\scriptsize{Taylor:}{}} Oh no! Is your foot OK?

 {\scriptsize{Rob:}{}} No. And my leg is hurt, too.

 {\scriptsize{Taylor:}{}} Well, stop hopping and I can help you.
\end{multicols}


第二段对话,关于牙疼的话题, Conversation B:

\begin{multicols}{2}
 {\scriptsize{James:}{}} Hello, Linda. Is something wrong?

 {\scriptsize{Linda:}{}} Yes, James. I don't feel well. My tooth hurts.

 {\scriptsize{James:}{}} That isn't good. How often do you go to the dentist?

 {\scriptsize{Linda:}{}} I never go to the dentist. I'm too scared!

 {\scriptsize{James:}{}} Hmm. Dentists aren't scary. And you need a dentist.

 {\scriptsize{Linda:}{}} Maybe you're right. My tooth really hurts. I have to go today.
\end{multicols}


  \subsection{good和well}
一般而言,good是指事物(名词)好,well是指动作好。

\begin{printdemosample}
 I swim well.
\end{printdemosample}

\begin{printdemosample}
 She speaks English well.
\end{printdemosample}

\begin{printdemosample}
 You speak English quite well.
\end{printdemosample}

get well多表示病情转好、康复的意思

\begin{printdemosample}
 I guarantee that if you take this medicine, you'll soon get well.
\end{printdemosample}


\begin{question}
-  Which is \blank[width=1.6cm]{}, the sun, the moon or the earth?

- Of course the moon is.
   \begin{tasks}[counter-format=tsk[A].](4)
     \task small
     \task smaller
     \task smallest
     \task the smallest
    \end{tasks}
\end{question}

\begin{question}[tags={mathEng}]
 There are 8 bees near the hive. 3 bees fly away.

How many bees stayed?  \blank[width=2cm]{}
\end{question}

\begin{question}
 Amy is a \blank[width=1cm]{} girl. She cooks \blank[width=1cm]{}.

 \begin{tasks}[counter-format=tsk[A].](4)
     \task good; good
     \task well; well
     \task good; well
    \end{tasks}
\end{question}

  \subsection{简单的后缀}
英语单词词根和汉字的偏旁部首类似,这里只根据小学课程的安排,先记录非常简单的内容。

\begin{itemize}
 \item $\sim$ful - 表示"xxx的“, 比如\cite{bibEngJuniorOneB}中描述春天的场景中使用了两个这样的单词: \marginpar{类似词汇helpful, wonderful}
    \begin{tasks}[counter-format=tsk[1].](3)
      \task colourful  \task beautiful \task useful
    \end{tasks}

 \item $\sim$fly - 多表示能飞的东西或者动物。
   \begin{tasks}[counter-format=tsk[1].](3)
      \task dragonfly  \task butterfly  \task firefly
    \end{tasks}

 \item $\sim$erry - 这是一个和甜水果有关的后缀。通常小小,圆圆的小果子都是以这个erry作为结尾的。\marginpar{blueberry, lingonberry}

   \begin{tasks}[counter-format=tsk[1].](3)
      \task cherry  \task berry \task strawberry
   \end{tasks}

 \item $\sim$er - 表示"做xx工作的人", 或者"和xx相关的东西物品"等等,下面是一些单词示例:
    \begin{tasks}[counter-format=tsk[1].](3)
      \task singer  \task teacher \task dancer
   \end{tasks}

 \item $\sim$ache - 表示"疼",
    \begin{tasks}[style=enumerate, label-offset=1em, label-align=right](4)
 \task headache  \task toothache  \task stomachache
 \end{tasks}
\end{itemize}


\begin{printdemosample}
 Who am I? I have colorful feathers!
\end{printdemosample}

\begin{printdemosample}
 The bread and butter is thickly spread with strawberry jam.
\end{printdemosample}

\begin{printdemosample}
 Toothache is pain in one of your teeth.
\end{printdemosample}


\begin{question}[tags={mathEng}]
Maria has 2 cats.

Marta has 5 cats.

How many cats do they have in all?  \blank[width=2cm]{}
\end{question}

\begin{question}[tags={mathEng}]
Isi saw six cars.

Jamaal saw three cars.

How many cars did Isi and Jamaal see?  \blank[width=2cm]{}
\end{question}

\begin{question}[tags={mathEng}]
 3 people are on the bus. 3 more people get on.

 How many people are on the bus now? \blank[width=2cm]{}
\end{question}

  \subsection{few/little等词的区别} \marginpar{2017学而思三年级考点}
这里需要知道下面的几个“一些”、"一点点"的表述词汇:

 \begin{tasks}[counter-format=tsk[1].](4)
      \task few  \task a few \task little  \task a little
   \end{tasks}

虽然上面的词,在中文当中都有“很少、一点点”的意思,但使用场合是有区别的!


1. few, a few用来于可数名词

2. little, a little用于不可数名词

3. a little, a few是肯定含义,即“还有一些”

4. little, few是否定含义,即"没有多少了"

\begin{printdemosample}
 She has a few friends in school.

 She has few friends in school.
\end{printdemosample}

\begin{printdemosample}
 There is a little oil in the dish.

 There is little oil in the dish.
\end{printdemosample}

\begin{printdemosample}
 I'm still thirsty because there's little water left.
\end{printdemosample}

下面进行一些小测试:

\begin{question}
 - Mr Wang, would you please tell me the result of the test?

 - You've done a good job. You made \blank[width=1cm]{} mistakes.

 \begin{tasks}[counter-format=tsk[A].](4)
     \task a few
     \task few
     \task a little
     \task little
    \end{tasks}
\end{question}


\begin{question} \marginpar{2017学而思三年级}
 Let's buy some rice. We have \blank[width=1cm]{} rice at home.

 \begin{tasks}[counter-format=tsk[A].](4)
     \task few
     \task a few
     \task a little
     \task little
    \end{tasks}
\end{question}

\begin{question}
 Hurry up! We've got \blank[width=1cm]{} time before the train leaves.

 \begin{tasks}[counter-format=tsk[A].](3)
     \task few
     \task a little
     \task little
    \end{tasks}
\end{question}

%%注意people是集体名词,所以不能把它看成不可数名词!
\begin{question}
 There's still a little orange juice here, but \blank[width=1cm]{} people want to drink it.

  \begin{tasks}[counter-format=tsk[A].](4)
     \task few
     \task a few
     \task a little
     \task little
    \end{tasks}
\end{question}

 \begin{question}
 Don't worry. We have \blank[width=1cm]{} time left.

  \begin{tasks}[counter-format=tsk[A].](4)
     \task few
     \task a few
     \task a little
     \task little
    \end{tasks}
\end{question}

  \subsection{dessert和desert}
二者非常像,但意思完全不同! 注意形象地记忆这两个单词!比如把甜食吃掉一个s后,就变成沙漠了。

第一个音标为美国发音,第二个为英国发音:

%https://tex.stackexchange.com/questions/25249/how-do-i-use-a-particular-font-for-a-small-section-of-text-in-my-document
\begin{myWordList2}
  \begin{tasks}[counter-format=tsk[1].](1)
    \task dessert \qquad \tipaencoding{[dI"z3:rt]} \qquad \tipaencoding{[dI"z3:t]}{}
    \task desert \qquad \,  \tipaencoding{["dez@rt]} \qquad \tipaencoding{["dez@t]}{}
  \end{tasks}

\end{myWordList2}

\begin{printdemosample}
 She had homemade ice cream for dessert.
\end{printdemosample}


\begin{printdemosample}
 A desert is a large area of land, usually in a hot region, where there is almost no water, rain, trees, or plants.

 For example, the Sahara Desert - 撒哈拉沙漠
\end{printdemosample}

因为沙漠desert没有人烟,所以也有"无人居住、废弃"的意思:

\begin{printdemosample}
 The villages had been deserted.
\end{printdemosample}


  \subsection{鞋子的各种说法}

shoes只是鞋子的总称,我们要会下面的单词:

\begin{tasks}[style=enumerate](4)
 \task slippers  \task sneakers \task boots
 \task sandals
\end{tasks}

特别注意一双鞋(a pair of shoes)由于使用了a pair of这个说法,所以尽管鞋子shoes是复数,但是a pair of是一个整体,所以我们要用is而不用are:

\begin{printdemosample}
 This pair of shoes is not clean.

 Those two pairs of shoes are clean.
\end{printdemosample}

\begin{printdemosample} \marginpar{注意用they}
 William puts on his pants. They're too small.
\end{printdemosample}


\begin{question}
 This pair of gray trousers \blank[width=1cm]{} (is/are) a little short.
\end{question}



  \subsection{国家和国家人民的称呼}
国家和国家人民以及语言,都可以使用:

\begin{tasks}[style=enumerate, label-offset=1em, label-align=right](3)
 \task China      \task Japan       \task Germany
 \task Chinese    \task Japanese    \task German
\end{tasks}

继续:

\begin{tasks}[style=enumerate, label-offset=1em, label-align=right](4)
 \task England   \task America(USA)   \task France    \task Australia
 \task English   \task American         \task French    \task Australian
\end{tasks}

来自某个国家、地区(from):

\begin{printdemosample}
 - Where are you from?

 - I'm from China.
\end{printdemosample}

\begin{printdemosample}
 We won three games against Australia.
\end{printdemosample}

 \subsection{一般过去时} \label{pastGrammar} \marginpar{这个知识点需要长时间的练习!}
当我们论述故事的时候,多用一般过去时。

它强调的是过去发生的事情,故yesterday等,是最简单的判断依据。

\begin{printdemosample} \marginpar{get up \quad got up}
 I got up at 6:30 yesterday.
\end{printdemosample}

需要知道be对应的过去式: \marginpar{do \quad did}

\def\arraystretch{1.25}
\begin{tabular}{l | l  }
\hline
{\large{be}}      &         {\large{过去式}}        \\
\hline
{is}              & \multirow{2}{*}{was}     \\
{am}              &                            \\
\hline
{are}           &  {were}  \\
\hline
\end{tabular}

\begin{printdemosample}
 When I was a child, I usually played with my cousins.
\end{printdemosample}

\begin{printdemosample}
 My parents were in Beijing last year. They are here now.
\end{printdemosample}

\begin{printdemosample}
 Paul's aunt wasn't at home yesterday.
\end{printdemosample}

\begin{printdemosample}
 It was sunny and they were happy yesterday.
\end{printdemosample}


接下来,对动作(do), 需要对下面的句型有一定的熟悉度:

\bgroup
\def\arraystretch{1.45}
\begin{tabular}{|l|l|l|}
\hline
{\large{肯定式}} & {\large{疑问式}} & {\large{否定式}}\\
\hline
{I worked $\cdots$} & {Did I work $\cdots$ ?} & {I didn't work $\cdots$}  \\
{You worked $\cdots$} & {Did you work $\cdots$ ?} & {You didn't work $\cdots$} \\
{She worked $\cdots$} & {Did she work $\cdots$ ?} & {She didn't work $\cdots$} \\
{We worked $\cdots$} & {Did we work $\cdots$ ?} & {We didn't work $\cdots$} \\
\hline
\end{tabular}
\egroup

\begin{printdemosample}  \marginpar{say \quad said}
 - What did you say just now?

 - I said $\cdots$
\end{printdemosample}

\begin{printdemosample} \marginpar{grow \quad grew}
 - Where did you grow up?

 - I grew up in Nanjing.
\end{printdemosample}

\begin{printdemosample} \marginpar{go to \quad went to}
 They usually went to school at ten o'clock last year.
\end{printdemosample}

\begin{printdemosample} \marginpar{come \quad came}
 It was lucky you came along.
\end{printdemosample}

\begin{printdemosample}
 - Did you go to Beijing last week?

 - Yes, we did.
\end{printdemosample}

\begin{printdemosample}
 - Did you go to Beijing yesterday?

 - No, I didn't.
\end{printdemosample}

\begin{printdemosample}
 What did you do yesterday?
\end{printdemosample}

 \subsection{-ist结尾的单词}
英语当中,$\sim$ist结尾的单词,通常表示“从事xxx的人员, xx家”的意思,比如下面的几个:

\begin{tasks}[style=enumerate, label-offset=1em, label-align=right](4)
 \task art      \task physics         \task science      \task piano
 \task artist   \task physicist        \task scientist    \task pianist
\end{tasks}

\begin{printdemosample}
 Each poster is signed by the artist.
\end{printdemosample}


继续:

\begin{tasks}[style=enumerate, label-offset=1em, label-align=right](4)
 \task chemistry  \task dentistry    \task psychology     \task violin
 \task chemist    \task dentist      \task psychologist   \task violinist
\end{tasks}


\begin{printdemosample}
 Linda's tooth hurts. How often does she go to the dentist? Never! Linda is scared of dentists.
\end{printdemosample}

\begin{printdemosample}
 Scientists have collected more data than expected.
\end{printdemosample}

 \subsection{复杂的长句子(从句)} \label{sec.ComplicatedLong}
看长句子的一个示例 (他就是那个偷了我钱包的小偷):

\begin{printdemosample}
 He is the thief.

 The thief stole my purse!
\end{printdemosample}

因为是人,所以合并的时候用who来连接:

\begin{printdemosample}
 He is the thief who stole my purse!
\end{printdemosample}

如果表示的是其它类型,则对应的连接词是:

\bgroup
\def\arraystretch{1.15}
\begin{tabular}{l|l}
\hline
{地点} & {where} \\
\hline
{时间} & {when}  \\
\hline
{原因} & {why}  \\
\hline
{物品} & {which/that}  \\
\hline
{谁的} & {whose} \\
\hline
\end{tabular}
\egroup

\begin{printdemosample}
 I like the place where I was born.
\end{printdemosample}

\begin{printdemosample}
 Is that the reason why you didn't come to my birthday party?
\end{printdemosample}

\begin{question} \marginpar{2018学而思二年级}
 The city \blank[width=1cm]{} she lives in is very far away.
 \begin{tasks}[counter-format=tsk[A].](3)
     \task when
     \task who
     \task which
    \end{tasks}

\end{question}

\begin{question}
 The letter is from my sister, \blank[width=1cm]{} is working in Beijing.

 \begin{tasks}[counter-format=tsk[A].](4)
     \task which
     \task that
     \task whom
     \task who
    \end{tasks}
\end{question}

  \subsection{反身代词}
反身代词是指“我自己”、“他自己”等等:

\begin{tasks}[style=enumerate, label-offset=1em, label-align=right](4)
 \task myself   \task himself   \task herself   \task ourselves
 \task themselves \task itself  \task yourself  \task yourselves
\end{tasks}

\begin{printdemosample}
 Take good care of yourself.
\end{printdemosample}

\begin{printdemosample}
 The child can dress himself.
\end{printdemosample}

\begin{printdemosample} \marginpar{creep 爬行\\trough 水槽}
 Wilbur checked himself and crept slowly to his trough.
\end{printdemosample}


\begin{question}
Ann is alone. She is playing by \blank[width=1cm]{}.

  \begin{tasks}[counter-format=tsk[A].](4)
        \task she  \task her  \task herself  \task me
  \end{tasks}
\end{question}

\begin{question}
The boy always likes to talk to\blank[width=1cm]{}.

  \begin{tasks}[counter-format=tsk[A].](4)
        \task himself  \task myself  \task herself  \task yourself
  \end{tasks}
\end{question}

  \subsection{睡和醒的几个不同说法}
注意下面几个不同的含义:

\begin{itemize}
 \item be asleep
 \item be sleepy
 \item be awake
 \item wake up
\end{itemize}

be sleepy可以理解为“打瞌睡,想要睡觉”(但是还没睡着):

\begin{printdemosample}
 You must be sleepy, aren't you?
\end{printdemosample}

而fall asleep, be asleep表示"已经睡着了"

\begin{printdemosample}
 My father was so tired that he fall asleep quickly.
\end{printdemosample}

be awake是“这个人是醒过来的,不是睡着的”, wake up一般表示“把某人叫醒”。

\begin{printdemosample}
 The kid wakes me up at 7:00 am.
\end{printdemosample}


 \subsection{常见节假日}
节假日的单词书写及对应的风俗文化,基本是小升初考试中较常见的内容。

比如中国的一些传统节日:

\begin{tasks}[style=enumerate, label-offset=1em, label-align=right](2)
 \task Dragon Boat Festival  \task Mid-Autumn Festival
 \task Lantern Festival
\end{tasks}


西方的节日比较多,通常是"xxx Day"的样子。 目前需要记得下面几个常见的:

\begin{tasks}[style=enumerate, label-offset=1em, label-align=right](2)
 \task Easter \task Christmas \task Christmas Eve  \task Halloween
 \task Thanksgiving
\end{tasks}

最隆重的是圣诞节和圣诞节前的晚上(Christmas Eve)

\begin{printdemosample}
 The day after Christmas is generally a busy one for retailers.
\end{printdemosample}

\begin{printdemosample}
 Merry Christmas, Mom.
\end{printdemosample}

Halloween Day(万圣节)是每年的10月最后一天(31st October),小孩穿上戏服,一家一户地要糖果吃。一般开门后说的话是:

\begin{printdemosample}
 "Trick or treat?"
\end{printdemosample}

当中的trick是做恶、欺骗的意思。假设一些题目有陷阱让人不小心就上当,我们可以称这些题目比较tricky. \marginpar{tricky}

treat则是"款待、招待"的意思

感恩节上,火鸡(turkey)和南瓜饼是必备的。它是在万圣节后的11月份,美国等国家规定是十一月份(November)的第4个星期四。

\begin{printdemosample}
 It's going to be the most magnificent Thanksgiving dinner we ever had.
\end{printdemosample}

\begin{printdemosample}
 On Thanksgiving Day, a teacher asked her class of first-graders to draw a picture of something they were thankful for.
\end{printdemosample}


复活节(Easter Day)的具体日期,是按照月亮和地球的转动来计算的,所以每年的日期都不定,一般分布在3月底到4月之间。

复活节的传统就是Easter Eggs,把鸡蛋染成彩色。兔子也是复活节的一个象征。

\begin{printdemosample}
 He only sees her at Christmas and Easter.
\end{printdemosample}

\begin{printdemosample}
 "Happy Easter," he yelled.
\end{printdemosample}



  \subsection{一年12个月份}

\begin{tasks}[style=enumerate, label-offset=1em, label-align=right](3)
 \task January \task February  \task March
 \task April   \task May    \task June
 \task July   \task August  \task September
 \task October \task November  \task December
\end{tasks}

在平时使用中,上面的月份多使用前三个字母作为缩写。

\begin{printdemosample}
 Her baby was born on 4th December.
\end{printdemosample}

\begin{printdemosample}
 Her son, Jerome, was born in September.
\end{printdemosample}

从上面两个示例要看出,具体到某一天,用on,否则用in。这个可以记住于日历来进行形象记忆。

  \subsection{单词kind} \index{kind}
kind的单词主要有两个不同含义:

\begin{tasks}[style=enumerate, label-offset=1em, label-align=right](2)
 \task 种类、类型 \task 友善的、好心的
\end{tasks}

\begin{printdemosample}
 This kind of question often appears in the exam.
\end{printdemosample}

\begin{printdemosample}
- Would you please do me a favor?

- Sure. How may I help you?

- Thank you. It's kind of you.
\end{printdemosample}

\begin{printdemosample}
 Which kind of music do you prefer?
\end{printdemosample}

\begin{printdemosample}
 -  I'm a writer.

 - What kind of book are you writing?
\end{printdemosample}


  \subsection{关于时间(time)}

两种问时间的方式: \marginpar{需要注意两者的区别!}

\begin{itemize}
 \item What's the time?
 \item What time is it?
\end{itemize}

\begin{printdemosample}
 - What's the time?

 - It's time to go to school!
\end{printdemosample}

我们说“按时,准时”使用on time这个惯用词:

\begin{printdemosample}
 You must be there on time.
\end{printdemosample}

另一个短语是in time, 多表示"迟早、过了一段时间,或者及时"的意思。

on time一般用在句子最后,表示正好和指定的时间一样,没有偏差。而in time表示及时,也只是说明比指定时间提前了。

注意这两个的区别。

\begin{printdemosample}
 They moved on to a larger farm and in time made it over to Francis.
\end{printdemosample}

几点半的说法,我们要会两种不同的说法:

\begin{printdemosample} \marginpar{7:30}
 - What time is it?

 - It's seven thirty.
\end{printdemosample}

另一种是用“7点过了一半”的方式:

\begin{printdemosample}\marginpar{7:30}
  - What time is it?

 - It's half past seven.
\end{printdemosample}

除了上面的表达外,我们还需要学会下面的说法:

\begin{tikzpicture}
   \draw (0,0) circle (1);

   \draw (90:1) --+(-90:0.3); \draw (-90:1) --+(90:0.3);  \draw (0:1) --+(-180:0.3); \draw (180:1) --+(0:0.3);
   \draw (60:1) --+(-120:0.18);  \draw (30:1) --+(-150:0.18);
   \draw (-60:1) --+(120:0.18); \draw (-30:1) --+(150:0.18);
   \draw (120:1) --+(-60:0.18); \draw (150:1) --+(-30:0.18);
   \draw (-120:1) --+(60:0.18);\draw (-150:1) --+(30:0.18);

   \node at (3.5, 0) [] {minute past hour};
   \node at (-3.5, 0) [] {minute to hour};

   \draw [->, >=stealth'] (-3.5, 0.3) to [bend left=35] (-0.4, 0.1);
   \draw [->, >=stealth'] (3.5, 0.3) to [bend right=35] (0.4, 0.1);

      \draw [dashed] (90:1) -- (-90:1);
\end{tikzpicture}

下面学习几个时间:

\begin{tasks}[style=enumerate, label-offset=1em, label-align=right](2)
 \task \begin{tikzpicture}
   \draw (0,0) circle (1);

   \draw (90:1) --+(-90:0.3); \draw (-90:1) --+(90:0.3);  \draw (0:1) --+(-180:0.3); \draw (180:1) --+(0:0.3);
   \draw (60:1) --+(-120:0.18);  \draw (30:1) --+(-150:0.18);
   \draw (-60:1) --+(120:0.18); \draw (-30:1) --+(150:0.18);
   \draw (120:1) --+(-60:0.18); \draw (150:1) --+(-30:0.18);
   \draw (-120:1) --+(60:0.18);\draw (-150:1) --+(30:0.18);
   %% tick
   \draw [->, >=stealth'] (0:0) --+ (-10: 0.42);
   \draw [->, >=stealth'] (0:0) --+ (300: 0.65);

  \end{tikzpicture}
 \task \begin{tikzpicture}
   \draw (0,0) circle (1);

   \draw (90:1) --+(-90:0.3); \draw (-90:1) --+(90:0.3);  \draw (0:1) --+(-180:0.3); \draw (180:1) --+(0:0.3);
   \draw (60:1) --+(-120:0.18);  \draw (30:1) --+(-150:0.18);
   \draw (-60:1) --+(120:0.18); \draw (-30:1) --+(150:0.18);
   \draw (120:1) --+(-60:0.18); \draw (150:1) --+(-30:0.18);
   \draw (-120:1) --+(60:0.18);\draw (-150:1) --+(30:0.18);
   %% tick
   \draw [->, >=stealth'] (0:0) --+ (-20: 0.42);
   \draw [->, >=stealth'] (0:0) --+ (120: 0.65);


  \end{tikzpicture}
\end{tasks}

对应答案:

\begin{tasks}[style=enumerate, label-offset=1em, label-align=right](2)
 \task twenty-five past three  \task five to four
\end{tasks}

其实在中文当中也是有类似的叫法,故我们中文中的“一刻钟”,在英语中则是有quarter来表达的,比如:

\begin{printdemosample}
A quarter past three.

A quarter to four.
\end{printdemosample}

对应的时间点分别是:

\begin{tasks}[style=enumerate, label-offset=1em, label-align=right](2)
 \task \begin{tikzpicture}
   \draw (0,0) circle (1);

   \draw (90:1) --+(-90:0.3); \draw (-90:1) --+(90:0.3);  \draw (0:1) --+(-180:0.3); \draw (180:1) --+(0:0.3);
   \draw (60:1) --+(-120:0.18);  \draw (30:1) --+(-150:0.18);
   \draw (-60:1) --+(120:0.18); \draw (-30:1) --+(150:0.18);
   \draw (120:1) --+(-60:0.18); \draw (150:1) --+(-30:0.18);
   \draw (-120:1) --+(60:0.18);\draw (-150:1) --+(30:0.18);
   %% tick
   \draw [->, >=stealth'] (0:0) --+ (-10: 0.42);
   \draw [->, >=stealth'] (0:0) --+ (0: 0.65);

  \end{tikzpicture}
 \task \begin{tikzpicture}
   \draw (0,0) circle (1);

   \draw (90:1) --+(-90:0.3); \draw (-90:1) --+(90:0.3);  \draw (0:1) --+(-180:0.3); \draw (180:1) --+(0:0.3);
   \draw (60:1) --+(-120:0.18);  \draw (30:1) --+(-150:0.18);
   \draw (-60:1) --+(120:0.18); \draw (-30:1) --+(150:0.18);
   \draw (120:1) --+(-60:0.18); \draw (150:1) --+(-30:0.18);
   \draw (-120:1) --+(60:0.18);\draw (-150:1) --+(30:0.18);
   %% tick
   \draw [->, >=stealth'] (0:0) --+ (-15: 0.42);
   \draw [->, >=stealth'] (0:0) --+ (180: 0.65);

  \end{tikzpicture}
\end{tasks}

上午和下午的两个讲法在英语中对应的:

\begin{tasks}[style=enumerate, label-offset=1em, label-align=right](2)
 \task a.m. \task p.m.
\end{tasks}

具体到某一天,使用on, 其它则用in

 \begin{printdemosample}
 She was born in 1950.

 She was born in America.
  \end{printdemosample}

具体的某一天的时间,多使用on

  \begin{printdemosample}
 More than eight years later, on Sept. 11, 2001, terrorists struck again, this time with more chilling results.

 My little daughter was born on a cold morning of December.
  \end{printdemosample}

在星期几,也是用on

\begin{printdemosample}
 Zhang Lei and Li Ming want to draw some pictures on Wednesday.
\end{printdemosample}


  \subsection{ph/dr发音}
英语中ph通常发 \begin{IPA}[f]\end{IPA} 的音:

\begin{tasks}[style=enumerate, label-offset=1em, label-align=right](2)
 \task phone  \task photo  \task physics
\end{tasks}

体育课P.E.实际是physical education的缩写。

dr的\begin{IPA}[dr]\end{IPA}:

\begin{tasks}[style=enumerate](3)
 \task drum   \task drink   \task drunk
 \task dress  \task drill   \task drop
\end{tasks}

drink不但可用于“喝”这个动作,也可以作名词,表示“喝的东西(即饮料)"

比如soft drink表示不含酒精的饮料。

\begin{printdemosample}
 Can I have a drink?
\end{printdemosample}

\begin{printdemosample}
 He doesn't drink.
\end{printdemosample}

短语dress up表示”打扮“:

\begin{printdemosample}
 You do not need to dress up for dinner.
\end{printdemosample}

  \subsection{be made of/from} \index{plastic} \index{metal} \index{swim}
这两个说法都用来表示"由什么材料制成的"


1. be made of 表示制成成品后,仍可看出原材料是什么,保留原材料的质和形状,制作过程仅发生物理变化.如:

\begin{printdemosample}
 The kite is made of paper. \quad \quad 风筝是用纸做的.
\end{printdemosample}


2.be made from 表示制成的东西完全失去了原材料的外形或特征,或原材料在制作过程中发生化学变化,在成品中已无法辨认.如:

\begin{printdemosample}
 The paper is made from wood. \quad \quad \quad 纸是木头做的.
\end{printdemosample}

\begin{printdemosample}
 Butter is made from milk.\quad \quad \quad 黄油是从牛奶中提炼出来的.
\end{printdemosample}

  \subsection{without/until}

without doing sth

without sth

\begin{printdemosample}
 He entered my room without asking me.

 He entered my room without my permission.
\end{printdemosample}

这里的两个词,经常通过否定(not)方式来表达:

not $\cdots$ without $\cdots$

not $\cdots$ until $\cdots$

对应的中文意思不应该生硬地翻译,而需要按照中文的习惯来表达!

\begin{printdemosample}
 I waited until 6 o'clock and then I went home.
\end{printdemosample}

\begin{printdemosample}
 Please wait until I finish my homework.
\end{printdemosample}

当使用not $\cdots$ without $\cdots$的时候,我们最好按照中文的思维,翻译成"直到$\cdots$才$\cdots$"

\begin{printdemosample}
 You must not leave the room until I let you go.
\end{printdemosample}

\begin{printdemosample}
 We won't be seeing each other until Christmas.
\end{printdemosample}

\begin{printdemosample}
 He finished five essays without difficulty.
\end{printdemosample}

\begin{printdemosample}
 We can't live without air and water.
\end{printdemosample}

另外一个类似的短语就是at all, 以及any more/any longer等。

\begin{printdemosample}
 This is not difficult at all!
\end{printdemosample}

上面的at all表明一点都不难,相当于说非常简单,用来强调语气。

\begin{printdemosample}
 It wasn't fast at all.
\end{printdemosample}


\begin{printdemosample}
 She liked pandas when she was a child. Now, she doesn't like them any more.
\end{printdemosample}

这里再添加一对容易混淆的词组,形容人"金发碧眼的": \marginpar{男性、女性}

\begin{tasks}[style=enumerate](2)
 \task blond  \task blonde
\end{tasks}

当中blond形容男性,blonde则用于女性.

  \subsection{人称代词的几种形式}
熟练地记住下面不同形式:

\def\arraystretch{1.25}
\begin{tabular}{| l | l  | l | l |}
\hline
{主格}      &         {宾格}       & {物主代词}  &   {所有格}\\
\hline
{I}              & {me}   &    {my}    &  {mine}\\
\hline
{you}              &  {you} & {your}        & {yours}                  \\
\hline
{he}           &  {him}   & {his} & {his}\\
\hline
{she}           &  {her}   & {her} & {hers}\\
\hline
{it}           &  {it}   & {its} & {its}\\
\hline
{we}           &  {us}   & {our} & {ours}\\
\hline
{they}           &  {them}   & {their} & {theirs}\\
\hline
\end{tabular}

注意这当中的his/her等比较特殊,几个不同形式会是一样的词。

\begin{question} \marginpar{博亚杯试题}
找出下面的不同项:

 \begin{tasks}[counter-format=tsk[A].](4)
 \task you \task her  \task them  \task his
\end{tasks}

\end{question}


  \subsection{between和among}
一般而言,between是只在两个$\cdots$中间,而among是只在一堆$\cdots$中间(超过两个)

\begin{printdemosample}
 They walked among the crowds in Red Square.
\end{printdemosample}

\begin{printdemosample} \marginpar{sit $\rightarrow$ sat}
 I sat down between Joe and Diana.
\end{printdemosample}

这里再补充记录一下老掉牙的:

\begin{printdemosample}
 It's time \ul{to do sth}.
\end{printdemosample}

\begin{printdemosample}
 It's time \ul{for sth}.
\end{printdemosample}


  \subsection{方向}
标准的东南西北:

\begin{tikzpicture}

  \draw [<->, >=stealth'] (0,0) -- (2, 0);
  \draw [<->, >=stealth'] (1, 1) -- (1, -1);

  \node at (2.5, 0) {east};
  \node at (-0.5, 0) {west};

  \node at (1, 1.5) {north};
  \node at (1, -1.5) {south};

  \draw [->, >=stealth'] (6, 0) -- (5, 0) node [above, pos=.5] {left};

  \draw [->, >=stealth'] (6.5, 0) -- (7.5, 0) node [above, pos=.5] {right};
\end{tikzpicture}

\begin{printdemosample}
 In the north the ground becomes very cold as the winter snow and ice covers the ground.
\end{printdemosample}

注意左、右方向的单词在英语中是有多个含义的。

left除了表示左边,还表示"剩下"

\begin{printdemosample}
 Is there any gin left?
\end{printdemosample}

right除了表示右边,还表示“对的"

\begin{printdemosample}
 Yes, you are right!
\end{printdemosample}

除此之外,right还可以用于加强语气的作用,比如:

\begin{printdemosample}
 It's right there!
\end{printdemosample}

\begin{printdemosample}
 If you have a problem with that, I want you to tell me right now!
\end{printdemosample}



  \subsection{带ly后缀的副词}
带ly后缀的,通常是副词,修饰动作(类似中文的"$\cdots$地$\cdots$")

不带ly的,则是形容词,修饰名词(类似中文的"$\cdots$的$\cdots$")

\begin{tasks}[style=enumerate](4)
 \task slowly   \task quickly    \task quietly  \task carefully
 \task slow     \task quick      \task quiet    \task careful
\end{tasks}


\begin{printdemosample}
 The waiter walked away quickly.
\end{printdemosample}

\begin{printdemosample}
 He spoke slowly.
\end{printdemosample}

\begin{printdemosample}
 We carefully watched every detail of his action.
\end{printdemosample}

\begin{printdemosample}
 She picks them up happily.

 She picked them up happily yesterday.
\end{printdemosample}


  \subsection{动作及运动(sports)}
运动相关的单词在初级阶段,属于考察较多的词汇,这里稍微列举一下。

球类运动多用$\sim$ball的方式,但也有特殊的(如\framebox{羽毛球}):

\begin{tasks}[style=enumerate](4)
 \task basketball   \task baseball    \task football  \task volleyball
 \task badminton    \task hockey      \task tennis    \task table tennis
\end{tasks}

由于文化差异 \marginpar{注意这种表达差异!},足球football在美国英语中并不是足球,而是橄榄球。美国英语中使用\framebox{soccer}来表示足球!

\begin{printdemosample}
 NBA - National Basketball Association
\end{printdemosample}

注意使用play的时候,弹奏乐器是在乐器前面加the,而体育运动前面一般不加the.

\begin{printdemosample}
 In their family, one plays the piano, another plays the violin, and the third is a singer; in short, they are all musicians.
\end{printdemosample}

\begin{printdemosample}
 Several boys were still playing football on the ground.
\end{printdemosample}

小练习:

\begin{question}
 I can play football very \blank[width=1cm]{}, but my brother isn't \blank[width=1cm]{} at it.

  \begin{tasks}[counter-format=tsk[A].](4)
        \task good; good \task well; well  \task good; well  \task well; good
  \end{tasks}

\end{question}

 \subsection{suit}
suit通常表示人的"一套衣服",而旅行时候我们通常会把自己的衣服放到手提箱中,所以suitcase就是手提箱。

同时,衣服穿在身上是要合适的,所以suitable就表示"合适的"的意思:

\begin{tasks}[style=enumerate](3)
 \task suitcase   \task suitable    \task suit
\end{tasks}

\begin{printdemosample}
 Books are suitable for children.
\end{printdemosample}

\begin{printdemosample}
 This programme is not suitable for children.
\end{printdemosample}

\begin{printdemosample}
 What is the most suitable word to replace the \ul{good} word here?
\end{printdemosample}

  \subsection{other的几种讲法}
对于总数是两个的情况,我们是用"一个$\cdots$, 另外一个$\cdots$"

\begin{printdemosample}
 They have two toys. One is a doll, the other is a ball.
\end{printdemosample}

对于总数$> 2$的情况,another或者others,具体的差别看下图:

\begin{tikzpicture}

  \draw[] (0,0) circle (0.35);

  \draw[pattern = north west lines] (1,0) circle (0.35);
  \node at (1.2, 1) [] {one $\cdots$, the other $\cdots$};
  %%%%%%%%%%%%%%%%%%%%%%%%%%%%%%%%%%%%%%%%%%%%%

  \draw[] (3,0) circle (0.35);
  \draw[pattern = north west lines] (4.2,0) circle (0.35);
  \draw[] (4.9,0) circle (0.35);

  \draw [dashed] (3.7,-0.5) --++ (1.8, 0) --++(0, 1) --++ (-1.8, 0) --cycle;

  \node at (4.1, -1) {one $\cdots$, another $\cdots$};

  %%%%%%%%%%%%%%%%%%%%%%%%%%%%%%%%%%%%%%%%%%%%%
  \draw[] (6.5,0) circle (0.35);
  \draw[pattern = north west lines] (7.7,0) circle (0.35);
  \draw[pattern = north west lines] (8.4,0) circle (0.35);

   \draw [dashed] (7.2,-0.5) --++ (1.8, 0) --++(0, 1) --++ (-1.8, 0) --cycle;

   \node at (7.8, 1) {one $\cdots$, others $\cdots$};

\end{tikzpicture}

\begin{printdemosample}
 Some students like English and \ul{other students} like physics.
\end{printdemosample}

\begin{printdemosample}
 Some students like English and \ul{others} like physics.
\end{printdemosample}

\begin{printdemosample}
 I have two pens. One is blue, the other (pen) is black.
\end{printdemosample}


  \subsection{look forward to}
look forward to是一个非常特殊的使用方法,它作为一个整体使用,所以后面跟的方式有下面两种:

\begin{tasks}[style=enumerate](2)
  \task look forward to \ul{doing} sth.  \task look forward to sth.
\end{tasks}

这个和我们多数情况下使用to do sth的场合是不一样的!

\begin{printdemosample} \marginpar{新概念英语青少版2A第2单元句型}
 REPORTER :  Are you looking forward to Sunday?

 PAUL : Yes, I am.

 REPORTER : That's the spirit, Paul! Good luck on Sunday!
\end{printdemosample}

\begin{printdemosample} \marginpar{look forward to}
 He was looking forward to working with the new Prime Minister.
\end{printdemosample}

\begin{printdemosample}
 Well, we shall look forward to seeing him tomorrow.
\end{printdemosample}

\begin{printdemosample}
 We all look forward to the day when the scientists can discover more secrets of the universe.
\end{printdemosample}

  \subsection{too/either及neither}
"也"这个单词在肯定句中用too,否定句中是either。

注意either有两种读法:

\begin{tasks}[style=enumerate](2)
  \task \tipaencoding{[aID@]}
  \task \tipaencoding{[ID@]}
\end{tasks}

\begin{printdemosample}
 I don't know either.
\end{printdemosample}

从字面上看, neither和either一样,都表示否定的意思,表示'都不'的意思, 而且这个多表示(两个)都不的场景。

\begin{printdemosample} \marginpar{我们两都没有忘记它}
 Neither of us forgot about it.
\end{printdemosample}


\begin{printdemosample}
 - I don't like this dress.

 - Neither do I.
\end{printdemosample}

\begin{printdemosample}
 I never learned to swim and neither did they.
\end{printdemosample}


下面的几个用法需要熟记,不管对于考试和平常阅读都有非常大的帮助。

\begin{printdemosample}\marginpar{neither $\cdots$ nor $\cdots$}
 Professor Hisamatsu spoke neither English nor German.
\end{printdemosample}

\begin{printdemosample}
 I have neither time nor money.
\end{printdemosample}

\begin{printdemosample}
 Neither Anna nor I are interested in high finance.
\end{printdemosample}

\begin{printdemosample} \marginpar{either $\cdots$ or $\cdots$}
 He must be either mad or drunk.
\end{printdemosample}

\begin{printdemosample}
 The clock is available with either Roman or Arabic numerals.
\end{printdemosample}

下面的例句演示就近原则:

\begin{printdemosample} \marginpar{近视眼!}
 Either you or he is wrong.
\end{printdemosample}

\begin{printdemosample}
 Neither he nor I am well-educated.
\end{printdemosample}

总结 - neither $\cdots$ nor $\cdots$和either $\cdots$ or $\cdots$这两种表达方式,一定要会!


  \subsection{再论过去式}
前面的\ref{pastGrammar}节已经初步学习了一般过去时。当中需要记得下面两个重要点(需要笑笑自己能够说出来!)

\begin{tasks}[style=enumerate](2)
  \task is/am变was, are变were
  \task 动作词(如do)变did
\end{tasks}

这节将继续对动作词的过去式(do $\rightarrow$ did)做进一步的学习!

学习之前,需要理解英语当中,表达"过去曾经做过$\cdots$"的话,我们只需要把动作的单词做以下变化即可(就像名词后面加个s来表示多个东西类似)。这个变化的基本准则如下:

\begin{enumerate}
 \item 正常情况下,是在动词后面加上ed, 若以e结尾,则只加d

 \begin{tasks}[style=itemize](3)
  \task jump $\rightarrow$ jumped
  \task finish $\rightarrow$ finished
  \task walk $\rightarrow$ walked
 \end{tasks}

 \item 辅音+y结尾的,把y变i, 再加ed

 \begin{tasks}[style=itemize](3)
  \task cry $\rightarrow$ cried
  \task reply $\rightarrow$ replied
  \task study $\rightarrow$ studied
 \end{tasks}


 \item 辅音+元音+辅音(辅元辅), 则双拼结尾的字母

 \begin{tasks}[style=itemize](3)
  \task nod $\rightarrow$ nodded
  \task stop $\rightarrow$ stopped
  \task drop $\rightarrow$ dropped
 \end{tasks}

 \item 不符合上面规则的情况(动词的特殊过去式)

 \begin{tasks}[style=itemize](3)
  \task do $\rightarrow$ did
  \task go $\rightarrow$ went
  \task see $\rightarrow$ saw
 \end{tasks}

\end{enumerate}

而不规则的动词过去式几乎是必须要掌握的,所以下面列举了小学阶段应该熟练使用的动词不规则过去式,希望笑笑在平时学习中不断反复地练习:

几乎没变化的: \marginpar{注意read的过去式读音有变化!}

\begin{tasks}[style=enumerate](2)
  \task cut $\rightarrow$ cut
  \task hit  $\rightarrow$ hit
  \task put  $\rightarrow$ put
  \task cost  $\rightarrow$ cost
  \task hurt  $\rightarrow$ hurt
  \task read  $\rightarrow$ read
\end{tasks}

\begin{printdemosample}
 She hit him on the head with her umbrella yesterday. But now, she doesn't hit him any more.
\end{printdemosample}


\begin{printdemosample}\marginpar{注意读音}
 He read the poem aloud two hours ago.

 He reads the poem everyday.
\end{printdemosample}

不按常理出牌的:

\begin{tasks}[style=enumerate, label-offset=1em, label-align=right](2)
  \task feel $\rightarrow$ felt
  \task smell $\rightarrow$ smelt
  \task sell $\rightarrow$ sold
  \task tell $\rightarrow$ told
  \task meet $\rightarrow$ met
  \task find $\rightarrow$ found
  \task get $\rightarrow$ got
  \task have/has $\rightarrow$ had
  \task hold $\rightarrow$ held
  \task leave $\rightarrow$ left
  \task make $\rightarrow$ made
  \task come $\rightarrow$ came
  \task become $\rightarrow$ became
  \task go $\rightarrow$ went
  \task run $\rightarrow$ ran
  \task learn $\rightarrow$ learnt/learned
  \task hear $\rightarrow$ heard
  \task win $\rightarrow$ won
  \task catch $\rightarrow$ caught
  \task teach $\rightarrow$ taught
  \task bring $\rightarrow$ brought
  \task buy $\rightarrow$ bought
  \task think $\rightarrow$ thought
  \task fly $\rightarrow$ flew
  \task throw $\rightarrow$ threw
  \task know $\rightarrow$  knew
  \task eat $\rightarrow$ ate
  \task fall $\rightarrow$ fell
  \task give $\rightarrow$ gave
  \task take $\rightarrow$ took
  \task see $\rightarrow$ saw
  \task write $\rightarrow$ wrote
  \task ride $\rightarrow$ rode
  \task build $\rightarrow$ built
  \task lend $\rightarrow$ lent
  \task send $\rightarrow$ sent
  \task spend $\rightarrow$ spent
  \task lose $\rightarrow$ lost
  \task pay $\rightarrow$  paid
  \task say $\rightarrow$ said
  \task draw $\rightarrow$ drew
  \task speak $\rightarrow$ spoke
  \task steal $\rightarrow$ stole
  \task wake $\rightarrow$ woke
  \task choose $\rightarrow$ chose
  \task forget $\rightarrow$ forgot
  \task sleep $\rightarrow$ slept
  \task keep $\rightarrow$ kept
  \task sweep $\rightarrow$ swept
  \task stand $\rightarrow$ stood
  \task understand $\rightarrow$ understood
  \task begin $\rightarrow$ began
  \task ring $\rightarrow$ rang
  \task sing $\rightarrow$ sang
  \task swim $\rightarrow$ swam
  \task drink $\rightarrow$ drank
  \task sit $\rightarrow$ sat
\end{tasks}

  \subsection{过去式牛刀小试}
学习了前面一节内容,下面开始进行练习:

\begin{question}
 There \blank[width=1.5cm]{}(be) no one here a moment ago.
\end{question}
\begin{solution}
 was
\end{solution}

\begin{question}
 Helen and Nancy \blank[width=1.5cm]{}(be) good friends, but now they are not anymore.
\end{question}
\begin{solution}
 were
\end{solution}

\begin{question}
 I listened but \blank[width=1.5cm]{}(hear) nothing.
\end{question}
\begin{solution}
 heard
\end{solution}

\begin{question}
\noindent 找出下面句子的错误并修改正确:

  \begin{tasks}[style=enumerate, label-offset=1em, label-align=right](1)
    \task How is Jane yesterday? \qquad \quad \blank[width=2cm]{}
    \task He go to school by bus last week. \quad \blank[width=2cm]{}
    \task I can fly kites seven years ago. \quad \blank[width=2cm]{}
    \task Did you saw him just now? \quad \blank[width=2cm]{}
    \task Tom wasn't watch TV last night. \quad \blank[width=2cm]{}
    \task I did not my homework yesterday. \quad \blank[width=2cm]{}
  \end{tasks}
\end{question}
\begin{solution}
 1.
 2.
 3. can -> could
\end{solution}

\begin{question}
\noindent 分别写出下面动词的第三人称单数形式、现在分词形式、过去式形式:

  \begin{tasks}[style=enumerate, label-offset=1em, label-align=right](2)
    \task go \quad \blank[width=1cm]{} \quad \blank[width=1cm]{} \quad \blank[width=1cm]{}
    \task enjoy \quad \blank[width=1cm]{} \quad \blank[width=1cm]{} \quad \blank[width=1cm]{}
    \task buy \quad \blank[width=1cm]{} \quad \blank[width=1cm]{} \quad \blank[width=1cm]{}
    \task like \quad \blank[width=1cm]{} \quad \blank[width=1cm]{} \quad \blank[width=1cm]{}
    \task get \quad \blank[width=1cm]{} \quad \blank[width=1cm]{} \quad \blank[width=1cm]{}
    \task walk \quad \blank[width=1cm]{} \quad \blank[width=1cm]{} \quad \blank[width=1cm]{}
    \task take \quad \blank[width=1cm]{} \quad \blank[width=1cm]{} \quad \blank[width=1cm]{}
    \task dance \quad \blank[width=1cm]{} \quad \blank[width=1cm]{} \quad \blank[width=1cm]{}
    \task write \quad \blank[width=1cm]{} \quad \blank[width=1cm]{} \quad \blank[width=1cm]{}
    \task run \quad \blank[width=1cm]{} \quad \blank[width=1cm]{} \quad \blank[width=1cm]{}
    \task swim \quad \blank[width=1cm]{} \quad \blank[width=1cm]{} \quad \blank[width=1cm]{}
    \task find \quad \blank[width=1cm]{} \quad \blank[width=1cm]{} \quad \blank[width=1cm]{}
    \task begin \quad \blank[width=1cm]{} \quad \blank[width=1cm]{} \quad \blank[width=1cm]{}
    \task eat \quad \blank[width=1cm]{} \quad \blank[width=1cm]{} \quad \blank[width=1cm]{}
    \task play \quad \blank[width=1cm]{} \quad \blank[width=1cm]{} \quad \blank[width=1cm]{}
    \task study \quad \blank[width=1cm]{} \quad \blank[width=1cm]{} \quad \blank[width=1cm]{}
  \end{tasks}
\end{question}
\begin{solution}
 1.
 2.
 3. can -> could
\end{solution}

\begin{question}
 I \blank[width=1cm]{} my room last Sunday.

   \begin{tasks}[counter-format=tsk[A].](4)
              \task cleaned
              \task clean
              \task am cleaning
              \task cleans
   \end{tasks}

\end{question}
\begin{solution}
TODO
\end{solution}

\begin{question}
What  \blank[width=1cm]{} Mike do last weekend?

   \begin{tasks}[counter-format=tsk[A].](4)
              \task do
              \task does
              \task is
              \task did
   \end{tasks}

\end{question}
\begin{solution}
TODO
\end{solution}

\begin{question}
 \blank*[width=1cm]{} you \blank[width=1cm]{} TV last night?

   \begin{tasks}[counter-format=tsk[A].](4)
              \task Do, watch
              \task Did, watch
              \task Did, watched
              \task Does, watches
   \end{tasks}

\end{question}
\begin{solution}
TODO
\end{solution}

\begin{question}
 \blank*[width=1cm]{} he \blank[width=1cm]{} football two days ago?

   \begin{tasks}[counter-format=tsk[A].](4)
              \task Does, play
              \task Did, played
              \task Did, play
              \task Does, playing
   \end{tasks}

\end{question}
\begin{solution}
TODO
\end{solution}

\begin{question}
 They \blank[width=1cm]{} on a trip in February, 2007.

   \begin{tasks}[counter-format=tsk[A].](4)
              \task are going
              \task going
              \task went
              \task goes
   \end{tasks}

\end{question}
\begin{solution}
TODO
\end{solution}

\begin{question}
 She watered the flowers \blank[width=1cm]{}.

   \begin{tasks}[counter-format=tsk[A].](2)
              \task tomorrow
              \task sometimes
              \task yesterday morning
              \task every morning
   \end{tasks}

\end{question}
\begin{solution}
TODO
\end{solution}

\begin{question}
 Mike \blank[width=1cm]{} a big tiger in the nature park last year.

   \begin{tasks}[counter-format=tsk[A].](4)
              \task saw
              \task see
              \task seeing
              \task seen
   \end{tasks}

\end{question}
\begin{solution}
TODO
\end{solution}

\begin{question}
- Good afternoon, Miss Lee. How does Mike feel?

- He's tired. He \blank[width=1cm]{} a lot of work \blank[width=1cm]{}.

   \begin{tasks}[counter-format=tsk[A].](2)
              \task does, this morning
              \task do, this morning
              \task did, this morning
              \task have, this morning
   \end{tasks}

\end{question}
\begin{solution}
TODO
\end{solution}

\begin{question}
 尝试将绘本中的现在时改成过去时。

 或者将绘本中的过去时的动词给找出来。
\end{question}
\begin{solution}
TODO
\end{solution}

  \subsection{both和all}
both一般是指两个,其反义词是neither

all用于两个以上,其反义词是none

\begin{printdemosample}
I have three brothers. All of them are college students.
\end{printdemosample}

\begin{printdemosample}
He has two sisters. Both of them major in English.
\end{printdemosample}

\begin{printdemosample}
 - Which one would you like? The cake, or the orange?

 - Both.
\end{printdemosample}

both $\cdots$ and $\cdots$的短语也是平常阅读中经常碰到的.

\begin{printdemosample} \marginpar{夏洛的网}
 He was strong and brave, but the truth is, both the goose and the gander were worried about Templeton.
\end{printdemosample}

\begin{printdemosample}
 Both you and I are good students.
\end{printdemosample}

\begin{printdemosample}
Either you or I am good students.

Either I or you are good students.
\end{printdemosample}

  \subsection{though/although}\label{sec.Although}
though/although的意思类似,表示"虽然、尽管"的意思。

这里需要特别注意的是,中文里面经常有"虽然$\cdots$,但是$\cdots$"的讲法,在英文中,千万不要因为这种思维,而给英语加上"though/although $\cdots$, but $\cdots$"的说法。

英语当中,though/although和but,二者只能使用一个,不能同时使用。

\begin{printdemosample}
 Although he was tired, he went on working.

 Though he was tired, he still went on working.
\end{printdemosample}

\begin{printdemosample}
 He often helps me with my English (al)though he is quite busy.
\end{printdemosample}

\begin{printdemosample}
 He said he would come, he didn't, though.
\end{printdemosample}

类似的不能同时使用的还有"因为$\cdots$,所以$\cdots$", 这个在英语当中不可同时用"because $\cdots$, so $\cdots$"

\begin{printdemosample}
 He can't come because he is ill.
\end{printdemosample}

\begin{printdemosample}
 It was still painful so I went to see a doctor.
\end{printdemosample}


because of sth.

\begin{printdemosample}
 He can't come because of his illness.
\end{printdemosample}

\begin{printdemosample}
 He walked slowly because of his bad leg.
\end{printdemosample}

\begin{printdemosample}
 Many families break up because of a lack of money.
\end{printdemosample}

和although/though类似的一个词是however, 一般都用于句首,表示"然而,尽管,但是"等意思:

\begin{printdemosample}
 This question is very difficult. However, Qucy solves it in a few minutes.
\end{printdemosample}

  \subsection{状语从句}
这个多用于"当$\cdots$时候"

\begin{printdemosample}
 When the teacher came in, I stopped laughing.
\end{printdemosample}

\begin{printdemosample}
 Wash your hands before you have meals.
\end{printdemosample}

\begin{printdemosample}
 After you use the toilet, flush it.
\end{printdemosample}

\begin{printdemosample}
 The detective found the truth as soon as he saw the window.
\end{printdemosample}

另一个需要记住的表达方式:

\begin{tasks}[style=enumerate, label-offset=1em, label-align=right](1)
  \task so $\cdots$ that $\cdots$ - 某某某是如此的$\cdots$,以至于$\cdots$
  \task so that - 因此,以便
\end{tasks}

\begin{printdemosample}
 Bring it closer so that I may see it better.
\end{printdemosample}

\begin{printdemosample}
 I am so tired that I want to sleep right away!
\end{printdemosample}

\begin{printdemosample}
 The boy ran so fast that I couldn't catch him.
\end{printdemosample}

如果我们表达的是事物(名词),而不是形容词, 那么要用such $\cdots$ that $\cdots$, 而不用so $\cdots$ that $\cdots$:

\begin{printdemosample}
 This is such an interesting book that we all enjoy reading it!
\end{printdemosample}

\begin{printdemosample}
 This book is so interesting that we all enjoy reading it!
\end{printdemosample}

\begin{printdemosample}
 It was such a fine day that we went out for a walk.
\end{printdemosample}

单词for, 也可以表示"因为"的意思,只不过没有because那么强烈,相当于简单地补充原因的意味:

\begin{printdemosample}
 He seldom goes out now, for he is very lazy.
\end{printdemosample}


下面进行一些小测试:

\begin{question}
 The girl is crying \blank[width=1cm]{} she felt very sad.

   \begin{tasks}[counter-format=tsk[A].](4)
              \task though
              \task before
              \task because
              \task so
   \end{tasks}

\end{question}
\begin{solution}
C
\end{solution}

\begin{question}
 It was \blank[width=1cm]{} hot last night \blank[width=1cm]{} we wouldn't sleep well.

   \begin{tasks}[counter-format=tsk[A].](4)
              \task too, to
              \task so, that
              \task /, when
              \task /, if
   \end{tasks}

\end{question}
\begin{solution}
B
\end{solution}

\begin{question}
 It is \blank[width=1cm]{} a heavy box \blank[width=1cm]{} nobody can move it away.

   \begin{tasks}[counter-format=tsk[A].](4)
              \task as, as
              \task so, that
              \task such, that
              \task so, as
   \end{tasks}

\end{question}
\begin{solution}
C
\end{solution}

\begin{question}
 He doesn't see anything \blank[width=1cm]{} he is blind.

   \begin{tasks}[counter-format=tsk[A].](4)
              \task though
              \task for
              \task but
              \task so
   \end{tasks}

\end{question}
\begin{solution}
B for 也可以表示因为的意思,只不过没有because那么强烈。
\end{solution}

\begin{question}
 He didn't go home \blank[width=1cm]{} he finished his homework.

   \begin{tasks}[counter-format=tsk[A].](4)
              \task if
              \task because
              \task until
              \task when
   \end{tasks}

\end{question}
\begin{solution}
C
\end{solution}

\begin{question}
 Bring it nearer \blank[width=1cm]{} I may see it better.

   \begin{tasks}[counter-format=tsk[A].](4)
              \task although
              \task even though
              \task so that
              \task since
   \end{tasks}

\end{question}
\begin{solution}
?
\end{solution}

\begin{question}
 Though he is a great writer,  \blank[width=2cm]{}.

   \begin{tasks}[counter-format=tsk[A].](1)
              \task his books is not widely read
              \task but his books are not widely read
              \task however his books is not widely read
              \task his books are not widely read
   \end{tasks}

\end{question}
\begin{solution}
D
\end{solution}

\begin{question}
 \blank*[width=1cm]{} he does, nobody believes him again.

   \begin{tasks}[counter-format=tsk[A].](4)
              \task Whenever
              \task However
              \task Whatever
              \task What
   \end{tasks}

\end{question}
\begin{solution}
C
\end{solution}


  \subsection{ever结尾的词}
以$\sim$ever结尾的词比较多,需要记得它们的用法:

\begin{tasks}[style=enumerate, label-offset=1em, label-align=right](3)
  \task never  \task however
  \task whatever \task whenever
  \task wherever \task whoever
\end{tasks}

however在\ref{sec.Although}节中讲过有"可是,但是"等和though/although类似的意思,除此之外,还有"无论如何、不管怎样"这个和'how'有关的含义。

\begin{printdemosample}
 However carefully I explained, she still didn't understand.
\end{printdemosample}

\begin{printdemosample}
 She has the window open, however cold it is outside.
\end{printdemosample}

\begin{printdemosample}
 However you look at it, it's going to cost a lot.
\end{printdemosample}

whatever表示'不管什么,任何事物':

\begin{printdemosample}
 Do whatever you like.
\end{printdemosample}

whenever表示‘无论何时’(任何时候):

\begin{printdemosample}
 You can ask for help whenever you need it.
\end{printdemosample}

wherever表示'无论何地':

\begin{printdemosample}
 Sit wherever you like.
\end{printdemosample}

whoever表示'无论何人':

\begin{printdemosample}
 Whoever did this will sooner or later be caught!
\end{printdemosample}

   \subsection{not only/but also的表达}
not only$\cdots$, but (also)$\cdots$在中文当中,是"不但$\cdots$, 而且$\cdots$"的表达意思。

\begin{printdemosample}
 Not only Jim but also his wife saw those elephants.
\end{printdemosample}

\begin{printdemosample}
 The Americans and the British not only speak the same language but also share a large number of social customs.
\end{printdemosample}

注意下面的句子中,指的是light color和bright color两种(分别是淡色/浅色和鲜艳色):

\begin{printdemosample}
 Light and bright colors make people not only happier but more active.
\end{printdemosample}

当not only/but also连接2个句子时,第一个句子可用倒装的方式:

\begin{printdemosample}
 Not only has he a first-class brain but also he is a tremendously hard worker.
\end{printdemosample}

  \subsection{whether和weather}
这两个词发音一样,但意思完全不同。 \marginpar{自己要知道怎么巧妙记忆这两个的区别}

whether相当于if,表示"是不是,是否"


\begin{printdemosample}
 Nobody knows whether there are aliens.
\end{printdemosample}

\begin{printdemosample}
 Tell me if you are just joking.
\end{printdemosample}

有时候也会采用 whether $\cdots$ or not 的表达方式:

\begin{printdemosample}
 Let me know whether he will come or not.
\end{printdemosample}

\begin{printdemosample}
 To this day, it's unclear whether he shot himself or was murdered.
\end{printdemosample}

而weather是天气(详见\ref{sec.Weather}节内容)。下面再额外学习两个表示天气的单词:

\begin{tasks}[style=enumerate, label-offset=1em, label-align=right](2)
  \task storm  \task thunder
\end{tasks}

\begin{printdemosample}
 A few minutes later the storm began.
\end{printdemosample}

\begin{printdemosample}
 There was frequent thunder and lightning, and torrential rain.
\end{printdemosample}


  \subsection{宾语从句}
宾语从句比较简单,是一种稍微长的句子,按照大体单词意思基本都能够理解其表达的含义:

比如William说"我饿了。",那么我们就可以这么描述:

\begin{printdemosample}
 William says that he is hungry.
\end{printdemosample}

\begin{printdemosample}
 William says that he wants to have dinner.
\end{printdemosample}

\begin{printdemosample}
 I want to know why you are late.
\end{printdemosample}


如果说的是一个地点、时间、人物,那么将分别用where,when,who/whom等词。\marginpar{\ref{sec.ComplicatedLong}节初步讨论过}

\begin{printdemosample}
 Could you tell me who knows the answer, please?
\end{printdemosample}

而若是"谁的", 则用whose:

\begin{printdemosample}
 He asked whose handwriting was the best in our class.
\end{printdemosample}

\begin{printdemosample}
 The teacher asked us how many people there were in the room.
\end{printdemosample}

\begin{printdemosample}
 Could you tell me what I should do with the money?
\end{printdemosample}

\begin{printdemosample}
 Do you know which class he is in?
\end{printdemosample}

\begin{printdemosample}
 She asked me if I knew whose pen it was.
\end{printdemosample}

下面来几个练习:

\begin{question}
Could you telll me \blank[width=1cm]{} the nearest hospital is?

   \begin{tasks}[counter-format=tsk[A].](4)
              \task what
              \task how
              \task whether
              \task where
   \end{tasks}
\end{question}
\begin{solution}
D
\end{solution}

\begin{question}
Do you know where \blank[width=1cm]{} now?

   \begin{tasks}[counter-format=tsk[A].](4)
              \task he lives
              \task does he live
              \task he lived
              \task did he live
   \end{tasks}
\end{question}
\begin{solution}
A
\end{solution}

\begin{question}
I can't understand \blank[width=3cm]{}.

   \begin{tasks}[counter-format=tsk[A].](2)
              \task what does Christmas mean
              \task what Christmas does mean
              \task what mean Christmas does
              \task what Christmas means
   \end{tasks}
\end{question}
\begin{solution}
D
\end{solution}

\begin{question}
I don't know \blank[width=2cm]{}. Can you tell me, please?

   \begin{tasks}[counter-format=tsk[A].](2)
              \task how the two players are old
              \task how old are the two players
              \task the two players are how old
              \task how old the two players are
   \end{tasks}
\end{question}
\begin{solution}
D
\end{solution}

\begin{question}
I want to know \blank[width=3cm]{}.

   \begin{tasks}[counter-format=tsk[A].](2)
              \task whom is she looking after
              \task whom she is looking
              \task whom is she looking
              \task whom she is looking after
   \end{tasks}
\end{question}
\begin{solution}
D
\end{solution}

  \subsection{instead}
instead这个词,字面上是‘作为替代’的意思,在翻译中需要注意体会其含义。

\begin{printdemosample}
 Lily isn't here. Ask Lucy instead.
\end{printdemosample}

\begin{printdemosample}
 She didn't answer me, instead, she asked me another question.
\end{printdemosample}

有时候还会使用instead of的方式:

\begin{printdemosample} \marginpar{注意究竟是问的哪一位}
 We'll ask Sucy instead of Mary.
\end{printdemosample}

\begin{printdemosample}\marginpar{代替}
 I will go instead of you.
\end{printdemosample}

\begin{printdemosample}
 She had to spend nearly four months away from him that summer, instead of the usual two.
\end{printdemosample}

  \subsection{many和much}
这两个词都表示"很多的"意思:

\begin{tasks}[style=enumerate, label-offset=1em, label-align=right](2)
  \task many 用于可数名词
  \task much 用于不可数名词
\end{tasks}

\begin{printdemosample} \marginpar{people是集体名词}
 There are many flowers.

 There are many people.
\end{printdemosample}

\begin{printdemosample}
 I have much money.
\end{printdemosample}

much除了表示许多数量外,还可以修饰动作类的词:

\begin{printdemosample}
 I like it very much!

 We like them so much!
\end{printdemosample}

或者修饰比较级:

\begin{printdemosample}
 I feel much better.
\end{printdemosample}

\begin{printdemosample}
 He's much taller than her.
\end{printdemosample}

  \subsection{enjoy}
enjoy本身表示乐于其中的意思。

\begin{printdemosample}
 They all enjoyed themselves at the party.
\end{printdemosample}

\begin{printdemosample}
 I must say I am really enjoying myself at the moment.
\end{printdemosample}

喜欢做某事也可以使用enjoy doing sth来表达:\marginpar{\ref{sec.like}节也类似表达方式}

\begin{printdemosample} \marginpar{板球}
 I enjoyed playing cricket.
\end{printdemosample}

\begin{printdemosample}
 Everyone in my family enjoys listening to music.
\end{printdemosample}

上述例句当中,需要明白everyone, someone, anyone这个表示每个人,某个人,任何人,其实际指代人是一个,所以都是第三人称。在现在时当中,动作要使用第三人称单数形式。

\begin{printdemosample}
 There is someone at the door.
\end{printdemosample}

\begin{printdemosample}
 Is anyone here?
\end{printdemosample}

\begin{printdemosample}
 Everyone in the street was shocked when they heard the news.
\end{printdemosample}

  \subsection{各种表达说话的单词}
英语当中常见的表示说话的单词有下面几个:

\begin{tasks}[style=enumerate, label-offset=1em, label-align=right](4)
  \task say
  \task talk
  \task speak
  \task tell
\end{tasks}

这当中,talk强调的是几个人一起交谈(每个人都在说);speak则比较庄重,前调一个人在说,而其他人则是听众。

\begin{printdemosample}
 Let's talk in English!
\end{printdemosample}

\begin{printdemosample}
 Miss Wu is going to speak at our class meeting.
\end{printdemosample}

因为speak的这个1人说,多人听的特点,演讲单词有就是speech(也是speak的名词形式):

\begin{printdemosample}
 Several people made speeches at the wedding.
\end{printdemosample}

另外, speak也可表示说的某国语言(因为这也是1个说,其他人听的场景)

\begin{printdemosample}
 Please speak in English!
\end{printdemosample}

talk经常和with, about, to等连着用

\begin{printdemosample}
 Jim is going to talk about English names tomorrow.
\end{printdemosample}

\begin{printdemosample}
 Look, Mrs Chen is talking with Lucy's parents.
\end{printdemosample}

\begin{printdemosample}
 What are you talking about?
\end{printdemosample}


say则主要指说话的内容:

\begin{printdemosample}
 I don't know what she said.
\end{printdemosample}

或者也可以表示纸条上写的话:

\begin{printdemosample}
 Kate saw a card on her table, it said: "Happy Birthday!"
\end{printdemosample}

tell是把一件事情、故事讲述给别人听,从而让别人知道事情、故事的结果。

\begin{printdemosample}
 He will tell this good news to everyone.
\end{printdemosample}

\begin{printdemosample}
 Buzz off, Bill! I will tell Miss Hill!
\end{printdemosample}

除了上面的传统表示说话的单词外,我们还需要知道更多的一些形容说话的词语,这些词语在英语的文章、小说当中经常出现\marginpar{cry并不只表示哭泣!}:

\begin{tasks}[style=enumerate, label-offset=1em, label-align=right](5)
  \task cry
  \task scream
  \task shout
  \task yell
  \task whisper
\end{tasks}

这些词语,相当于我们中文中如 - 窃窃私语、柔声细语、粗声粗气等说法

下面是夏洛的网中一段描述:

\begin{printdemosample}
  ''Run downhill!'' suggested the cows.

 "Run toward me!'' yelled the gander.

 "Run uphill!'' cried the sheep.

 "Turn and twist!'' honked the goose.

 "Jump and dance!'' said the rooster.

 "Look out for Lurvy!'' called the cows.

 "Look out for Zuckerman!'' yelled the gander.

 "Watch out for the dog!'' cried the sheep.

 "Listen to me, listen to me!'' screamed the goose.
\end{printdemosample}

  \subsection{as well}
这个单词表示"也"的意思,和too类似。所以也不会用于否定意思的句子当中。

\begin{printdemosample}
 You need to go shopping and I need to go shopping as well.
\end{printdemosample}

另一个表示"也"的是also

\begin{printdemosample}
 His father, also a top-ranking officer, had perished during the war.
\end{printdemosample}

\begin{printdemosample}
 Santa Claus comes from Finland. The country is near the North Pole. Santa lives in a small village there.
\end{printdemosample}


  \subsection{no matter}
no matter这个表示“不论,无论”的意思

\begin{printdemosample}
 No matter what you may say, he will not believe you.
\end{printdemosample}

\begin{printdemosample}
 No matter how well you know Paris, it is easy to get lost!
\end{printdemosample}


\begin{printdemosample}
  "All right," said Wilbur. "But don't fail to let me know if there's
anything I can do to help, no matter how slight."
\end{printdemosample}

  \subsection{一些名人的英文名}
一些中文经常看到的人名,往往并不清楚对应的英文单词。多数情况下,可以根据读音大致猜出人名,但这里还是把这些词语简单地罗列一下。

\begin{tasks}[style=enumerate, label-offset=1em, label-align=right](3)
  \task Einstein      \task [+] Germany German     \task[+] 爱因斯坦
  \task Washington    \task [+] America American    \task[+] 华盛顿
  \task Lincoln      \task [+] America American  \task[+] 林肯
  \task Newton       \task [+] England English \task[+] 牛顿
  \task Gauss        \task [+] Germany German  \task[+] 高斯
  \task Shakespeare  \task [+] England English \task[+] 莎士比亚
  \task Bach         \task [+] Germany German \task[+] 巴赫
  \task Chopin         \task [+] Poland Polish \task[+] 肖邦
  \task Beethoven     \task [+] Germany German \task[+] 贝多芬
  \task Michelangelo  \task [+] Italy Italian \task[+] 米开朗基罗
\end{tasks}

The mathematician Joseph-Louis Lagrange said that Newton was the greatest genius who ever lived, and once added that Newton was also "the most fortunate, for we cannot find more than once a system of the world to establish."

English poet Alexander Pope wrote the famous epitaph:

\begin{quote}
 Nature and nature's laws lay hid in night;

 God said "Let Newton be" and all was light.
\end{quote}

  \subsection{too ... to ...}
too $\cdots$ to $\cdots$的中文表达应该是"太$\cdots$了,以至于不能$\cdots$"

\begin{printdemosample}
 The boy is too young to go to school.
\end{printdemosample}

\begin{printdemosample}
 The hat is too large to wear.
\end{printdemosample}

\begin{printdemosample}
 The question is too difficult for me to answer.
\end{printdemosample}

\begin{question}
 \blank*[width=1cm]{} difficult questions they are! I can't answer them.
     \begin{tasks}[counter-format=tsk[A].](4)
              \task How
              \task How an
              \task What
              \task What an
    \end{tasks}
\end{question}

\begin{question}
 The girl is too short to get on the elephant. (改成同义句)

 1. The girl \blank[width=2cm]{} that she \blank[width=2cm]{} get on the elephant.

 2. The elephant \blank[width=2cm]{} that the girl \blank[width=2cm]{} get on it.
\end{question}

  \subsection{-able的后缀}
这个后缀来源于拉丁文,主要是用于形容词:

\begin{tasks}[style=enumerate](5)
 \task valuable   \task adjustable \task respectable \task usable  \task reasonable
 \task[+] value   \task[+] adjust  \task[+] respect  \task[+] use \task[+] reason
\end{tasks}

\begin{printdemosample}
Trainers are so comfortable to wear.
\end{printdemosample}

\begin{printdemosample}
She shifted into a more comfortable position on the chair.
\end{printdemosample}

\begin{printdemosample}
He made us a reasonable offer for the car.
\end{printdemosample}

\begin{question} \marginpar{学而思四年级}
She offered \blank[width=1cm]{} valuable advice that \blank[width=1cm]{} people disagreed.

  \begin{tasks}[counter-format=tsk[A].](4)
   \task such; a few
   \task such; few
   \task so; a few
   \task so; few
   \end{tasks}
\end{question}

我们在前面(\ref{sec.HowOften}节)学习问频率的时候知道了sometimes,实际上有好多个长得像的单词:\marginpar{有名的四胞胎}

\begin{tasks}[style=enumerate](4)
  \task sometimes
  \task some times
  \task sometime
  \task some time
\end{tasks}

记忆这些词,首先记得一个准则就是带time为时间,带times为次数。

所以:

\begin{itemize}
 \item sometimes是"有些次数",在中文里我们就使用"有的时候"来表达。 some times则是字面上的“一些次数”
 \item sometime表示"某个时间"(某个点), some time则是"一些时间"(某个段)
\end{itemize}

\begin{printdemosample}
- When can you finish the work?

- Sometime next month.
\end{printdemosample}

尝试填入合适的词语:

% some time
\begin{question}
They haven't seen each other for \blank[width=2cm]{}.
\end{question}

% sometime
\begin{question}
We'll take our holiday \blank[width=2cm]{} in August.
\end{question}

\begin{question}
\blank*[width=2cm]{} he goes to the cinema on Sunday.
\end{question}

\begin{question}
He has been to Beijing for \blank[width=2cm]{} before.
\end{question}

  \subsection{How about/What about}
提某种建议的时候,可以使用How about或者What about

\begin{printdemosample}
 How about you, Yang Ling?
\end{printdemosample}

注意如果是动作,那么需要使用doing sth.

\begin{printdemosample} \marginpar{\footnotesize{英国:mum\\美国:mom}}
 How about leaving a message to your mom?
\end{printdemosample}

\begin{printdemosample}
 He texted everyone and said, "How about coming to my house?"
\end{printdemosample}

\begin{printdemosample}
 It costs about \$10.
\end{printdemosample}


  \subsection{关于女子的相关词汇}

 首先引入ress这个后缀,通常指女性。比如tress指:(女子的)一绺长发; (女子的)披肩长发;

 接下来就可以记忆下面几个词汇了:

 \begin{tasks}[style=enumerate](2)
    \task[+] waiter  \task[+] waitress
    \task[+] actor   \task[+] actress
 \end{tasks}


 \begin{printdemosample} \marginpar{\footnotesize{a female actor}}
  An actress is a woman whose job is acting in plays or films.
 \end{printdemosample}

 \begin{printdemosample}
  A waitress is a woman who works in a restaurant, serving people with food and drink.
 \end{printdemosample}


下面是几个积累知识点:

\begin{printdemosample}
 I'm on my way.  \hspace{1cm} 我就来了(或者马上就到)

 I'm on it.      \hspace{2cm} 对于某人交代的事情,表示我正在做
\end{printdemosample}

\begin{printdemosample}
  My dog was barking on night. Woof woof woof, all night!
\end{printdemosample}

注意这里的on night,说话者是想具体指狗叫的那个晚上,所以采用on night的说法,注意在平时使用中揣摩。


\begin{question}
 What about \blank[width=1cm]{} (have) a picnic on the beach?
\end{question}


 \subsection{动词形式变化的认知}
这里面,需要练习能从第三人称单数到过去式, 以及现在分词,都能反过来知道其动词原形是什么。


 \begin{tasks}[style=enumerate](4)
 \task[+] tries   \task[+] tried   \task[+] trying      \task try
 \task[+] replies \task[+] replied \task[+] replying    \task reply
 \task[+] cries   \task[+] cried   \task[+] crying      \task cry
 \task[+] goes    \task[+] went    \task[+] going       \task go
 \task[+] stops   \task[+] stopped \task[+] stopping    \task stop
 \task[+] shops   \task[+] shopped \task[+] shopping    \task shop
 \task[+] says    \task[+] said    \task[+] saying      \task say
 \task[+] flies   \task[+] flew    \task[+] flying      \task fly
 \task[+] has     \task[+] had     \task[+] having      \task have
 \task[+] dances  \task[+] danced  \task[+] dancing     \task dance
 \task[+] runs    \task[+] ran     \task[+] running     \task run
\end{tasks}

%%%%%%%%%%%%%%%%%%%%%%%%%%%%%%%%%%%%%%%%%%%%%%%%%%%%%%%%%%%%%%%%%%%%%%%%%%%%%%%
 %%% 2019-05-25 上述小学英语第一阶段内容的终稿
 %%% 不再新添内容。后期的新知识放在下面的第二阶段!
%%%%%%%%%%%%%%%%%%%%%%%%%%%%%%%%%%%%%%%%%%%%%%%%%%%%%%%%%%%%%%%%%%%%%%%%%%%%%%%


\section{小学英语-II}

  \subsection{现在完成时}
首先建议观看完学而思上完成时的视频。同时知道现在完成时的表达方式是:

\hspace{1.2cm} $\cdots$ + have/has + 过去分词 + $\cdots$

\begin{printdemosample}
 He has gone shopping.

 I have already cleaned the toilet.
\end{printdemosample}

\begin{printdemosample}
 I have lost my watch.
\end{printdemosample}

\begin{printdemosample}
 I haven't seen him for a long time.
\end{printdemosample}

\begin{printdemosample}
 He has lived in Beijing for 30 years.

 She has lived in Beijing since 2010.
\end{printdemosample}


这当中需要注意的过去分词,一般情况下,和过去式一样,也是加ed,但对于不规则的过去式,它的过去分词通常也是不规则的。

 \begin{tasks}[style=enumerate](3)
 \task[+] is/am   \task[+] was   \task[+] been
 \task[+] are     \task[+] were  \task[+] been
 \task[+] do/does \task[+] did   \task[+] done
 \task[+] begin  \task[+] began   \task[+] begun
 \task[+] go  \task[+] went   \task[+] gone
 \task[+] break  \task[+] broke  \task[+] broken
\end{tasks}

暂时先记住上面这几个简单的变化。记住大部分情况下,过去分词和过去式的形式是类似的,只不过上面的是几个比较特殊的情况。

接下来学习现在完成时的否定讲法和疑问讲法. 如果知道了以前带be和带动作的否定与疑问的讲法, 那么就可以猜出来围绕这个have/has进行就行了.

不过和之前不同的是,这里面的动词不要再还原回去:

\begin{printdemosample}
 You have seen it.

 Have you seen it?

 You haven't seen it.
\end{printdemosample}

\begin{printdemosample}
 She has seen it.

 Has she seen it?

 She hasn't seen it.
\end{printdemosample}

下面有几个超级考点,需要知道:

\begin{itemize}
 \item have been to  $\quad$去过哪儿
 \item have gone to  $\quad$去$\cdots$了
 \item have been in  $\quad$在哪儿待了多长时间
\end{itemize}

\begin{printdemosample}
 - Have you ever been to India?

 - Yes, I have. $quad$ No, I haven't.
\end{printdemosample}

\begin{printdemosample}
 She has been at this school since 2016.

 Miss Zhu have taught English for 10 years.
\end{printdemosample}

下面进行一些简单练习:

\begin{question}
 - \blank[width=1cm]{} have you been an actor?

 - For three years.

  \begin{tasks}[counter-format=tsk[A].](4)
   \task When
   \task How
   \task How long
   \task Why
   \end{tasks}
\end{question}
%C

\begin{question}
 They have \blank[width=1cm]{} gone to Tibet.


  \begin{tasks}[counter-format=tsk[A].](4)
   \task yet
   \task just now
   \task already
   \task still
   \end{tasks}
\end{question}
%C

\begin{question}
 You don't need to describe her. I \blank[width=1cm]{} her serveral times.

  \begin{tasks}[counter-format=tsk[A].](4)
   \task had met
   \task have met
   \task met
   \task meet
   \end{tasks}
\end{question}
%B

\begin{question}
 Mr. Li isn't here. He \blank[width=1cm]{} to England.

  \begin{tasks}[counter-format=tsk[A].](4)
   \task has been
   \task have been
   \task has gone
   \task have gone
   \end{tasks}
\end{question}
%C

\begin{question}
 Mrs. Wang has lived in Nanjing \blank[width=1cm]{} 2010.

  \begin{tasks}[counter-format=tsk[A].](4)
   \task since
   \task from
   \task for
   \task in
   \end{tasks}
\end{question}
%A
\begin{question}
- Mum, may I go out and play basketball?

- \blank[width=1cm]{} you \blank[width=1cm]{} your homework yet?

  \begin{tasks}[counter-format=tsk[A].](4)
   \task Do; finish
   \task Are; finishing
   \task Did; finish
   \task Have; finished
   \end{tasks}
\end{question}

有的时候,注意完成时表示一种持续性的动作,所以通常有些非持续性动作的情况,我们不会使用完成时来表述的。

下面是一个示例:

\begin{question}
Tom \blank[width=1cm]{} the CD player for two weeks.

  \begin{tasks}[counter-format=tsk[A].](4)
   \task has lent
   \task has borrowed
   \task has bought
   \task has had
   \end{tasks}
\end{question}
% D

接下来,我们需要继续对不规则的过去分词进行整理:

 \begin{tasks}[style=enumerate](3)
 \task[+] cost   \task[+] cost   \task[+] cost
 \task[+] meet   \task[+] met  \task[+] met
 \task[+] sit    \task[+] sat   \task[+] sat
 \task[+] cut    \task[+] cut   \task[+] cut
 \task[+] put    \task[+] put   \task[+] put
 \task[+] read   \task[+] read  \task[+] read
 \task[+] hurt    \task[+] hurt   \task[+] hurt
 \task[+] burn    \task[+] burnt   \task[+] burnt
 \task[+] build    \task[+] built   \task[+] built
 \task[+] smell    \task[+] smelt   \task[+] smelt
\end{tasks}



  \subsection{used to的用法}
入门级别的used to do sth, 表示"过去经常做某事", 言下之意就是现在已经不做了:

\begin{printdemosample}
 Palmer says he used to spend four weeks talking about TV cooking shows in his class.
\end{printdemosample}

 上面句型中还包括了下面的2个用法

 \begin{itemize}
  \item spend time/money (in) doing sth
  \item spend time/money on/for sth
 \end{itemize}

\begin{printdemosample}
 I spent two hours on this maths problem.
\end{printdemosample}

\begin{printdemosample}
 They spent two years (in) building this bridge.
\end{printdemosample}

\begin{printdemosample}
  This kind of food programming is not as popular as it used to be.
\end{printdemosample}

 而另一个用到的则是doing sth的方式, 表示习惯做某事:

 \begin{itemize}
  \item be used to doing sth/sth
  \item get used to doing sth/sth
 \end{itemize}

\begin{printdemosample}
 I am used to cleaning the room.

 She is not used to eating Chinese food.
\end{printdemosample}

再比如夏洛的网中的一段话:

\begin{printdemosample}
 "Don't worry, you'll get used to it," said Templeton. He sat up $\cdots$
\end{printdemosample}

而正常的be used to do sth, 则是字面意思上的'用来做$\cdots$'

\begin{printdemosample}
 Wood is often used to make desks and chairs.
\end{printdemosample}


 \subsection{kind of与a kind of}
 kind of是一种口语化的表达方式,表示"有一点$\cdots$,稍微$\cdots$",所以有点a little的意味,通常后面带形容词。

 a kind of就是字面的"一种$\cdots$类型"意思

\begin{printdemosample}
 It is a kind of animal.

  It is kind of interesting
\end{printdemosample}

 kind表达种类的方式还有下面几种:

  \begin{tasks}[style=enumerate](2)
 \task[+] all kinds of $\cdots$   \task[+] 各种$\cdots$
 \task[+] a kind of $\cdots$      \task[+] 一种$\cdots$
 \task[+] different kinds of $\cdots$     \task[+] 不同类型的$\cdots$
 \task[+] many kinds of $\cdots$  \task[+] 许多种$\cdots$
\end{tasks}

\begin{printdemosample}
 There are many kinds of birds in the zoo.
\end{printdemosample}


\begin{question}
- Do you know that there are different \blank[width=1cm]{} animals in the zoo?

- Yes, I do. And I also know that some of them are \blank[width=1cm]{} scary.
  \begin{tasks}[counter-format=tsk[A].](2)
   \task kinds of; kind of
   \task kinds of; kinds of
   \task kind of; kinds of
   \task kind of; kind of
   \end{tasks}

\end{question}

 \subsection{三升四暑假知识整理}
表达'某个人的blabla'的时候,英语一般用's来表示, 但需要知道下面的一些区别:

\begin{printdemosample} \marginpar{两张桌子和一张桌子}
 This is Yuki's ruler.

 These are students' rullers.

 Lily's and Lucy's desk.

 Lily and Lucy's desk.
\end{printdemosample}

当中注意who's和whose千万别搞错!

\begin{printdemosample}
 - Whose book is this?

 - It's mine.
\end{printdemosample}

一个单词既当名词用,也当动词用的示例:

\begin{lstlisting}
 I saw a saw saw a saw.

 I comb my hair with my comb. (make your hair tidy)

 The business suit suits you well.
\end{lstlisting}

词汇学习:

\begin{tasks}[style=enumerate](3)
 \task[+] alarm clock   \task[+] Brazil   \task[+] whole
 \task[+] swan   \task[+]] bettle  \task[+] jellyfish
\end{tasks}

到达某地的说法:

\begin{itemize}
 \item arrive at schoool/station $\qquad$到达小地点
 \item arrive in Nanjing/Beijing $\qquad$到达大地点
 \item arrive home
\end{itemize}

看书、读报的说法,需要明白watch与read:

\begin{itemize}
 \item read - 通常看的东西是静态的(看里面的内容,不是look)
 \item watch - 通常看的东西是动态的
\end{itemize}

\begin{printdemosample}
 read a newspaper, read a magazine
\end{printdemosample}

频率副词的放置位置:

\begin{itemize}
 \item be的后面
 \item 动作do的前面
 \item 对于一般现在时,是不影响三单的
\end{itemize}

\begin{printdemosample}
 I'm never late.

 He always/sometimes goes to school at 7:00.
\end{printdemosample}

  \subsection{再论完成时}
这里面学习下面的两个不同场景:

\begin{tasks}[style=enumerate](2)
 \task has/have been to \task has/have gone to
\end{tasks}

记住have/has gone to表示已经去某地,但目前还没有回来。

而have/has been to表示已经去过某地,现在可能已经回来了。

\begin{printdemosample}
 She has been to Hangzhou.

 Oh, Qucy is not here, she has gone to Hangzhou.
\end{printdemosample}

另一个是一直呆在某地,has/have been in

\begin{printdemosample}
 He has been in Beijing for three months.
\end{printdemosample}

\begin{printdemosample}
 They have lived in Shanghai for 3 years.

 They have lived in Beijing since 2010.
\end{printdemosample}

注意上面的since一词,表示自从2010年开始,就一直居住在北京。

下面是一些过去分词和过去式不相同的动词整理:

\begin{tasks}[style=enumerate](3)
 \task[+] eat   \task[+] ate   \task[+] eaten
 \task[+] give   \task[+] gave   \task[+] given
 \task[+] know   \task[+] knew   \task[+] known
 \task[+] grow   \task[+] grew   \task[+] grown
 \task[+] run   \task[+] ran   \task[+] run
 \task[+] come   \task[+] came   \task[+] come
 \task[+] fly   \task[+] flew   \task[+] flown
 \task[+] ring   \task[+] rang   \task[+] rung
 \task[+] swim   \task[+] swam   \task[+] swum
 \task[+] see   \task[+] saw   \task[+] seen
 \task[+] write   \task[+] wrote   \task[+] written
 \task[+] wake   \task[+] woke   \task[+] waken
 \task[+] break   \task[+] broke   \task[+] broken
 \task[+] choose   \task[+] chose   \task[+] chosen
 \task[+] forget   \task[+] forgot   \task[+] forgotten
\end{tasks}


  \subsection{分数的数学表达方式}
这部分可以考虑结合Smart Math这本书一起练习

简单来讲,英语使用"数词+序数词"的方式来表达分数,当分子是大于1的数字的时候,后面的序数词要用复数


\begin{tasks}[style=enumerate](2)
 \task[*] $\displaystyle \frac{1}{3}$   \task[+] one third, 或者 a third
 \task[*] $\displaystyle \frac{2}{3}$      \task[+] two thirds
 \task[*] $\displaystyle \frac{5}{6}$     \task[+] five sixths
 \task[*] $\displaystyle \frac{4}{9}$  \task[+] four ninths
\end{tasks}

 除此之外,一刻钟的quarter也用来表示四分之一的含义,比如:

\begin{printdemosample}
 Three quarters of the students have passed the exam.
\end{printdemosample}

\begin{tasks}[style=enumerate](2)
 \task[*] $\displaystyle \frac{1}{4}$   \task[+] one quarter, 或者 a quarter, one fourth
\end{tasks}

  \subsection{动词的过去式过去分词整理}
1. 完全不变类型

\begin{tasks}[style=enumerate](3)
  \task[+] cut   \task[+] cut   \task[+] cut
  \task[+] put   \task[+] put   \task[+] put
  \task[+] let   \task[+] let   \task[+] let
  \task[+] hit   \task[+] hit  \task[+] hit
  \task[+] hurt   \task[+] hurt  \task[+] hurt
  \task[+] cost   \task[+] cost   \task[+] cost
  \task[+] read   \task[+] read   \task[+] read
 \end{tasks}

2. AAB类型
\begin{tasks}[style=enumerate](3)
  \task[+] beat   \task[+] beat   \task[+] beaten
\end{tasks}

3. ABA类型,兜了一圈又回去了
\begin{tasks}[style=enumerate](3)
  \task[+] come    \task[+] came   \task[+] come
  \task[+] become   \task[+] became   \task[+] become
  \task[+] run   \task[+] ran   \task[+] run
\end{tasks}

4. ABC类型,都不相同

\begin{tasks}[style=enumerate](3)
  \task[+] begin  \task[+] began  \task[+] begun
  \task[+] take    \task[+] took    \task[+] taken
  \task[+] drink    \task[+] drank   \task[+] drunk
  \task[+] mistake   \task[+] mistook   \task[+] mistaken
  \task[+] ring   \task[+] rang  \task[+] rung
  \task[+] ride   \task[+] rode  \task[+] ridden
  \task[+] sing   \task[+] sang  \task[+] sung
  \task[+] do   \task[+] did \task[+] done
  \task[+] swim   \task[+] swam \task[+] swum
  \task[+] write   \task[+] wrote \task[+] written
  \task[+] blow   \task[+] blew \task[+] blown
  \task[+] go   \task[+] went \task[+] gone
  \task[+] draw   \task[+] drew \task[+] drawn
  \task[+] lie   \task[+] lay \task[+] lain
  \task[+] fly   \task[+] flew \task[+] flown
  \task[+] see   \task[+] saw \task[+] seen
  \task[+] grow   \task[+] grew \task[+] grown
  \task[+] wear   \task[+] wore \task[+] worn
  \task[+] know   \task[+] knew \task[+] known
  \task[+] throw   \task[+] threw \task[+] thrown
  \task[+] show   \task[+] showed \task[+] shown
  \task[+] break   \task[+] broke \task[+] broken
  \task[+] choose   \task[+] chose \task[+] chosen
  \task[+] forget   \task[+] forgot \task[+] forgotten
  \task[+] bear   \task[+] bore \task[+] born/borne
  \task[+] speak   \task[+] spoke \task[+] spoken
  \task[+] draw   \task[+] drew \task[+] drawn
  \task[+] wake   \task[+] woke \task[+] woken
  \task[+] drive   \task[+] drove \task[+] driven
  \task[+] hide   \task[+] hid \task[+] hidden
  \task[+] eat   \task[+] ate \task[+] eaten
  \task[+] lay   \task[+] laid \task[+] laid
  \task[+] fall   \task[+] fell \task[+] fallen
  \task[+] give   \task[+] gave \task[+] given
  \task[+] rise   \task[+] rose \task[+] risen
  \task[+] shake   \task[+] shook \task[+] shaken
  \task[+] steal   \task[+] stole \task[+] stolen
 \end{tasks}

5. ABB类型,多数的动词都是这个摸样

\begin{tasks}[style=enumerate](3)
  \task[+] build  \task[+] built   \task[+] built
  \task[+] get    \task[+] got     \task[+] got/gotten
  \task[+] catch    \task[+] caught     \task[+]caught
  \task[+] hang(吊死)    \task[+] hanged     \task[+]hanged
  \task[+] hang(悬挂)    \task[+] hung     \task[+]hung
  \task[+] deal    \task[+] dealt    \task[+]dealt
  \task[+] feed    \task[+] fed     \task[+]fed
  \task[+] hold    \task[+] held     \task[+]held
  \task[+] find    \task[+] found     \task[+]found
  \task[+] sit    \task[+] sat     \task[+] sat
  \task[+] pay    \task[+] paid     \task[+] paid
  \task[+] win    \task[+] won    \task[+] won
  \task[+] send    \task[+] sent    \task[+] sent
  \task[+] meet    \task[+] met   \task[+] met
  \task[+] shoot    \task[+] shot   \task[+] shot
  \task[+] keep    \task[+] kept   \task[+] kept
  \task[+] tell    \task[+] told   \task[+] told
  \task[+] sleep    \task[+] slept   \task[+] slept
  \task[+] win    \task[+] won  \task[+] won
  \task[+] sweep    \task[+] swept  \task[+] swept
  \task[+] feel    \task[+] felt  \task[+] felt
  \task[+] smell    \task[+] smelt  \task[+] smelt
  \task[+] leave    \task[+] left  \task[+] left
  \task[+] lend    \task[+] lent  \task[+] lent
  \task[+] send    \task[+] sent  \task[+] sent
  \task[+] spend    \task[+] spent  \task[+] spent
  \task[+] lose    \task[+] lost  \task[+] lost
  \task[+] burn    \task[+] burnt  \task[+] burnt
  \task[+] learn    \task[+] learnt  \task[+] learnt
  \task[+] mean    \task[+] meant  \task[+] meant
  \task[+] teach    \task[+] taught  \task[+] taught
  \task[+] bring    \task[+] brought  \task[+] brought
  \task[+] fight    \task[+] fought  \task[+] fought
  \task[+] buy    \task[+] bought  \task[+] bought
  \task[+] think    \task[+] thought  \task[+] thought
  \task[+] hear   \task[+] heard  \task[+] heard
  \task[+] sell    \task[+] sold  \task[+] sold
  \task[+] tell    \task[+] told  \task[+] told
  \task[+] say    \task[+] said  \task[+] said
  \task[+] make    \task[+] made  \task[+] made
  \task[+] stand    \task[+] stood  \task[+] stood
  \task[+] understand    \task[+] understood  \task[+] understood

 \end{tasks}

  \subsection{数学英语初步}

这里主要介绍几何图形(geometry)在英语中的描述方法。

首先为平面图形:

\begin{tikzpicture}

 \draw (0,0) --(2,0) -- (1, 1.8) -- (0, 0);
 \node at (2.4, 1.6) [font=\scriptsize](IDslide) {side};
 \node at (2.4, 0.6) [font=\scriptsize](IDcorner) {corner};

 \draw [->, >=stealth'] (IDslide.south) to [bend left=30] ($(IDslide.south) + (-0.6, -0.4)$);
 \draw [->, >=stealth'] (IDcorner.south) to [bend right=30] ($(IDcorner.south) + (0.5, -0.4)$);;

 \draw (3.0, 0) rectangle (5.36, 1.3);

 \draw (6.0, 0) rectangle (7.30, 1.3);

 \draw (9.5, 0.75)  circle (0.7); \fill (9.5, 0.75) circle (1.5pt);
 \draw [->, >=stealth'] (8,1.1) to [bend left=30] (9.3, 0.82);
 \node at (8.2, 1.3)[font=\scriptsize]{center};

 \node at (1, -0.8) [] {triangle};

 \node at (4, -0.8) [] {rectangle};

 \node at (6.8, -0.8) [] {square};

 \node at (9.5, -0.8) [] {circle};
\end{tikzpicture}


\begin{question}
 How much is eighty-five minus fifty-eight plus sixty-nine?  \blank[width=2cm]{}
\end{question}


\begin{question}[tags={mathEng}]
 Write how many.

 \begin{tikzpicture}
  \draw (0, 0) rectangle (1 , 1) ;
 \end{tikzpicture}\hspace{1cm} \blank[width=1cm]{}sides. \blank[width=1cm]{}corners.

\end{question}
\begin{solution}
%placeholder
\end{solution}
\begin{question}[tags={mathEng}]
 How is a square different from a rectangle? \blank[width=3cm]{}
\end{question}
\begin{solution}
%placeholder
\end{solution}
\begin{question}[tags={mathEng}]
 I have 4 corners. My sides are not the same length.

What shape am I? \blank[width=3cm]{}
\end{question}
\begin{solution}
%placeholder
\end{solution}
\begin{question}[tags={mathEng}]
My sides are all the same length. I have 3 corners.

What shape am I? \blank[width=3cm]{}
\end{question}
\begin{solution}
%placeholder
\end{solution}

  \subsection{数学英语进阶}

  接下来为立体图形:

\begin{tikzpicture}
\fill[ball color=black!10] (0,0) circle (1.0 cm);
\newcommand\latitude[1]{%
  \draw (#1:1.0) arc (0:-180:{1.0*cos(#1)} and {0.2*cos(#1)});
  \draw[dashed] (#1:1.0) arc (0:180:{1.0*cos(#1)} and {0.2*cos(#1)});
}
\latitude{30};
\latitude{0};
\latitude{-30};

    \draw (-6,-1) arc (180:360:1cm and 0.5cm) -- (-5,1) -- cycle;
    \draw[dashed] (-6,-1) arc (180:0:1cm and 0.5cm);
    \shade[left color=gray,right color=gray,opacity=0.2] (-6,-1) arc (180:360:1cm and 0.5cm) -- (-5,1) -- cycle;


    \node at (4, 0)[cylinder, shape border rotate=180, draw, shape aspect=0.35,minimum height=3cm,minimum width=0.99cm,  cylinder uses custom fill, cylinder end fill=gray!20, cylinder body fill=gray!10] {cylinder};

    \node at (0, -1.8) [] {sphere};
    \node at (4.0, -1.8) [] {cylinder};
    \node at (-4.9, -1.8) [] {cone};
\end{tikzpicture}

这里需要整理一下"球",在中文中,地球、球状物虽然都是带"球"字,但英文中的描述却是完全不一样的,所以对应的单词也完全不同:

\begin{tasks}[style=enumerate](4)
 \task ball  \task sphere  \task earth  \task globe
\end{tasks}



%%%%%%%%%%%%%%%%%%%%%%%%%%%%%%%%%%%%%%%%%%%%%%%%%%%%%%%%%%%%%%%%%%%%%%%%%%%%%%%%%%%%%%%%%%%%%%%%%%%%%%%%%%%%%%%%%%%%%%%%%%%%%
%%%%%%%%%%%%%%%%%%%%%%%%%%%%%%%%%%%%%%%%%%%%%%%%%%%%%%%%%%%%%%%%%%%%%%%%%%%%%%%%%%%%%%%%%%%%%%%%%%%%%%%%%%%%%%%%%%%%%%%%%%%%%



   \subsection{谚语金句}

\begin{printdemosample} \marginpar{意喻苦尽甘来}
 April showers bring May flowers.
\end{printdemosample}

\begin{printdemosample}
 Yesterday is a history, tomorrow is a mystery, but today is a gift, that is why it is called Present.
\end{printdemosample}

\section{背诵默写}
这里整理需要能够背诵出来,并且能够默写出来的各种句型。有点类似英语写作素材和范文的功能。

Don't be afraid to use it!

  \subsection{学而思三年级下的背诵}

\begin{printdemosample}
 Would you like to buy my watch? It's a good watch. It was a birthday present that I got when I was seven years old.
\end{printdemosample}

\begin{printdemosample}
 My bedroom is my favorite place. It's purple which is my favorite color.
\end{printdemosample}

  \subsection{四年级寒假英语书写内容}
关于时态的练习:

她去学校的不同表达方式:

\begin{printdemosample}
She goes to school everyday.

She went to school yesterday/last Thursday.

She has gone to school.

She had gone to school when her mother got up yesterday morning.
\end{printdemosample}

将上面的时态分别写出否定、疑问句。

学习书写下面几个长句子:

\begin{itemize}
  \item 一年有四季
  \item 简介自己(名字,年龄,爱好)
  \item 介绍自己的玩具,好朋友
  \item 他们真的很有趣
  \item 我们完得很开心
\end{itemize}


  \subsection{Breakfast or lunch?} \marginpar{\scriptsize{新概念英语}}
\lettrine[lines=2, lraise=0.1]{I}t was Sunday. I never get up early on Sundays. I sometimes stay in bed until lunch time.

Last Sunday I got up very late. I looked out of the window. It was dark outside.   “What a day!” I thought. “It's raining again.” Just then, the telephone rang. It was my aunt Lucy. “I've just arrived by train,” she said. “I'm coming to see you.” “But I'm still having breakfast,” I said. “What are you doing ?” she asked. “I'm having breakfast,” I repeated. “Dear me,” she said. “Do you always get up so late ? It's one o'clock!”

\noindent \rule{1.0\textwidth}{0.5mm}

\noindent 1. I didn't finish my homework \blank[width=1cm]{} my mother came back home.
\begin{tasks}(4)
  \task since \task because \task until \task after

\end{tasks}

\noindent 2. The door bell \blank[width=1cm]{} when I was leaving.
\begin{tasks}(4)
  \task rang \task ringing \task rings \task rung
\end{tasks}

  \subsection{A Private Conversation} \marginpar{\scriptsize{新概念英语}}
\lettrine[lines=2, lraise=0.1]{L}{ast} week I went to the theatre. I had a very good seat. The play was very interesting. I did not enjoy it. A young man and a young woman were sitting behind me. They were talking loudly. I got very angry. I could not hear the actors.

I turned round. I looked at the man and the woman angrily. They did not pay any attention. In the end, I could not bear it. I turned round again. "I can't hear a word!" I said angrily.  "It's none of your business," the young man said rudely. "This is a private conversation!".

  \subsection{家庭成员(family)}
介绍人员的方式:

\begin{printdemosample}
 There are three people in my family. They are my father, my mother and me.
\end{printdemosample}

\begin{printdemosample}
 Hello everyone, my name is Yang YuXin. I'm eight years old.
\end{printdemosample}

 \subsection{熟练词语运用}

\begin{printdemosample}
 That's really interesting!
\end{printdemosample}


\begin{printdemosample}
 Fantastic! How beautiful they are!
\end{printdemosample}


\newpage

\section{英语进阶}
这里更多的是把各个知识点进行整理归纳,便于应对考试和课外阅读能力的提高。

  \subsection{特殊单复数整理}
1. 单词以s, o, x, ch, sh结尾的,多数是加es:

\begin{tasks}[counter-format=tsk[1].,label-offset=1em, label-align=right](4)
 \task bus  \task buses
 \task box  \task boxes
 \task glass    \task glasses
 \task potato  \task potatoes
 \task tomato  \task tomatoes
 \task peach    \task peaches
 \task sandwich  \task sandwiches
\end{tasks}

\framebox{注意} 有些o结尾的单词又不需要es:

\begin{tasks}[counter-format=tsk[1].,label-offset=1em, label-align=right](2)
 \task photo  \task photos
 \task piano  \task pianos
 \task stomach  \task stomachs
\end{tasks}

2. 辅音加y结尾的, y变成i, 再加es

\begin{tasks}[counter-format=tsk[1].,label-offset=1em, label-align=right](4)
 \task cherry  \task cherries
 \task strawberry  \task strawberries
\end{tasks}

\framebox{非辅音的不要加} \qquad boy

3. 以f, fe结尾的,变v, 再加es
\begin{tasks}[counter-format=tsk[1].,label-offset=1em, label-align=right](4)
 \task life  \task leaf  \task leaves
 \task scarf  \task scarves
\end{tasks}

4. oo变ee:

\begin{tasks}[style=itemize](2)
 \task foot   \task feet
 \task tooth  \task teeth
\end{tasks}

5. 其它特殊的:

\begin{tasks}[style=itemize](2)
 \task man  \task men       \task woman  \task women
 \task child  \task children \task mouse \task mice
\end{tasks}

6. 不可数名词:

\begin{tasks}[counter-format=tsk[1].,label-offset=1em, label-align=right](4)
  \task milk \task people \task sheep  \task water
  \task fish \task meat   \task juice  \task food
  \task fruit \task cheese
\end{tasks}

  \subsection{动词时态整理}\label{secEngProgress.VerbShitai}

现在时中,不发音的e:

\begin{tasks}[style=itemize](4)
 \task come  \task coming  \task make    \task making
 \task have  \task having  \task write   \task writing
 \task take  \task taking  \task become  \task becoming
\end{tasks}

最后字母双写再加ing的:

\begin{tasks}[style=itemize](4)
 \task stop  \task stopping   \task run   \task running
 \task hop   \task hopping    \task chop  \task chopping
 \task shop  \task shopping
\end{tasks}


  \subsection{太阳系的八大行星}

首先,在古代各个国家里都是能够观察得到行星的。但由于各个国家文化上的差异,对于这些行星的命名,都各自按照自己国家的习惯进行,但基本上多以本国神话人物名字命名。

目前英语上的八大行星的称谓,大多数源自罗马神话人物,而中国采用的命名方式则是古老的阴阳五行理论。



%%TODO - 下面的东西要重新整理。。。。  (https://www.zhihu.com/search?type=content&q=%E5%85%AB%E5%A4%A7%E8%A1%8C%E6%98%9F%E5%91%BD%E5%90%8D)
太阳系有八大行星,从内到外分别为水星Mercury、金星Venus、地球Earth、火星Mars、木星Jupiter、土星Saturn、天王星
Uranus和海王星Neptune。其中,水木金火土都比较亮,用肉眼就能观察到,所以人类很早就开始认识它们了。中文之所以称为水星、木星、金星、火
星、土星,因为古人观察到,金星色白、木星色青、水星色灰、火星色赤、土星色黄,对应我国古老的阴阳五行理论,分别给它们取名为金星、木星、水星、火星和
土星。

希腊人则用神话中的神名来命名这些行星,罗马人将其翻译为罗马神话中对应的神名,水星为信使之神Mercury、金星为爱与美之女神Venus、火
星为战神Mars、木星为主神Jupiter、土星为农神Saturn。天王星直到1781年才被人们用天文望远镜发现的,学者们延续古代命名行星的习
惯,用天空之神Uranus命名了这颗行星,中文译为天王星。海王星被发现的更晚,到1846年才用天文望远镜捕捉到,学者们用海神Neptune命名了
这个星,中文译为海王星。


  \subsection{英文的词根}\label{secCiGeng}
英语单词当中很多都是派生词,也就是所谓的前缀+词根+后缀,常用的词根有400个,前缀和后缀分别为30个,所以只要记住这些,就能组合成3000多个单词


er后缀的单词整理已经放在\ref{secErSuffix}节中介绍了!




dis

tele

这个前缀的单词多和通信, 无线电相关.

tele + vision = television

tele + phone = telephone

un

uniform, ...

%  \subsection{bi-}
bi - 表示“两,双”

binary

bilingual

biweekly

binomial - 二项式, 多项式叫做polynomial

\begin{tikzpicture}
  \node at (0, -1) [] {$(x+y)^n$};
  \node at (0, -1) [] {$a_0 + a_1x + a_2x^2 + \cdots + a_nx^n$};
\end{tikzpicture}



 % \subsection{dis-}
dis - 具有否定的含义

%% ALWAYS call this in each section, in order to
%% reset the num to 1
\setcounter{mydemonum}{0}

disconnect

discover discovery

dislike

disorder


%disappear
\begin{tikzpicture}
  \node [rectangle, anchor = east, text width=2cm] (ID1) {\hfill appear};
  \node [rectangle, anchor = east, below of = ID1, yshift=0.2cm, text width=2cm] (ID2) {\hfill \colorbox{mygreen}{dis}appear};

  \node [rectangle, anchor = east, right of = ID1, xshift=3cm, text width=3cm] {出现};
  \node [rectangle, anchor = east, right of = ID2, xshift=3cm, text width=3cm] {消失不见了};
\end{tikzpicture}

\begin{mydemosample}
The black car drove away from them and disappeared.
\end{mydemosample}


dis-前缀的另一个含义是分离、分开或者散开

比如老师宣布下课通常都这么讲:

%dismiss
\begin{mydemosample} \marginpar{\scriptsize{老师宣布下课通常都这么讲}}
You are dismissed.
\end{mydemosample}

discard

  \subsection{in-}
这是一个表示否定的前缀.

%% insane VS sane
\begin{tikzpicture}
  \node [rectangle, anchor = east, text width=2cm] (ID1) {\hfill sane};
  \node [rectangle, anchor = east, below of = ID1, yshift=0.2cm, text width=2cm] (ID2) {\hfill \colorbox{mygreen}{in}sane};

  \node [rectangle, anchor = east, right of = ID1, xshift=3cm, text width=3cm] {神智正常};
  \node [rectangle, anchor = east, right of = ID2, xshift=3cm, text width=3cm] {神智不正常, 疯了};
\end{tikzpicture}

\setcounter{mydemonum}{0}
  \begin{mydemosample}
  Are you insane?
  \end{mydemosample}

  \subsection{poly-}
poly-表示"多"的意思,经常出现在科学(数学、化学等)中的用语当中。

poly + gon = polygon 多边形

\begin{tikzpicture}
\foreach \a in {3,...,7}{
\draw[red, dashed] (\a*2,0) circle(0.5cm);
\node[regular polygon, regular polygon sides=\a, draw,
inner sep=0.3535cm] at (\a*2,0) {};
}
\end{tikzpicture}

poly + nomial = polymomial 多项式

  \subsection{tele-}


  %%%%%%%%%%%%%%%%%%%%%%%%%%%%%%%%%%%%%%%%%%%%%%%%%%%%%%%%%%
  \subsection{vi}
图像

visual vision video

visible


\section{易错用的例子整理}

  \subsection{note}
Please note that xxxx

相当于提醒,注意某某事情.

而不要用Please be noted that xxx

这个被动的说法教少见.


%%%%%%%%%%%%%%%%%%%%%%%%%%%%%%%%%%%%%%%%%%%%%%%%%%%%%%%%%%%%%%%%%%%%%%%%%%

%\input letustalkenglish.tex

%%%%%%%%%%%%%%%%%%%%%%%%%%%%%%%%%%%%%%%%%%%%%%%%%%%%%%%%%%%%%%%%%%%%%%%%%%

\section{空中美语初级}

因为这个节目的语速较快,内容也比较高, 所以这里不再按照时间顺序, 而是按照学习的难度逐个往后排序。


  \subsection{Be Nice to Your Sister}

音频文件2017-0612, 正文内容时间点大约是00:56开始。

场景发生于黄先生和黄太太(Mr and Mrs Wong)的三个小孩之间。这三个小孩分别为Page, Jimmy, Timmy, 他们在一起因为看电视的缘故而发生了争吵。

The three kids are sitting on the sofa. Page is watching a show, but Jimmy taks the remote.

\begin{multicols}{2}
 {\scriptsize{Page:}{}} Hey, I was watching my favorite TV show!

 {\scriptsize{Timmy:}{}} Your show is boring, we want to watch cartoons.

 {\scriptsize{Page:}{}} Why aren't you doing your homework?

 {\scriptsize{Timmy:}{}} You sound like mum.
...

{\scriptsize{Page:}{}} How dare you!

...

Why are you all shouting?


\end{multicols}

Page:

...


They didn't ...

...


You two need...

..

Yes, ...


...

...

%%%%%%%%%%%%%%%%%%%%%%%%%%%%%%%%%%%%%%%%%%%%%%%%%%%%%%%%%%%%%%%%%%%%%%%%%%%%%%

该话题中需要学习的单词如下:

\begin{myWordList}
  \begin{tasks}[counter-format=tsk[1].](2)
    \task cartoon
    \task sound
    \task naughty
    \task manners
  \end{tasks}

\end{myWordList}



\begin{printdemosample}
 Let's watch the cartoon about cute rabbits on T.V. now.

 Eating breakfast food for dinner sounds like a great idea.

 The teacher asks the naughty student to leave the classroom.

 It's not good manners to talk with your mouth..
\end{printdemosample}

  \subsection{Pineapples}
2017-0623

  \subsection{Study}
 2017-03002




  \subsection{书写-handwriting}
 2017-03003

 Could/Would you ... ºÍCan you ...Ö®¼äµÄһЩÇø±ð¡£

 Can I/May I ...

 9:45 listing and write:

 * duck * * pond *


 * personality  *

  12:08 handwriting  personality map

 * emotion *

15:45 or so, learning how to say `like it'

 * generally *


%%%%%%%%%%%%%%%%%%%%%%%%%%%%%%%%%%%%%%%%%%%%%%%%%%%%%%%%%%%%%%%%%%%%%%%%%%%%%
%%%%%%%%%%%%%%%%TODO - SHOULD include this in the future?%%%%%%%%%%%%%%%%%%%%
%%\input adv_read.tex
%%%%%%%%%%%%%%%%%%%%%%%%%%%%%%%%%%%%%%%%%%%%%%%%%%%%%%%%%%%%%%%%%%%%%%%%%%%%%
%%%%%%%%%%%%%%%%%%%%%%%%%%%%%%%%%%%%%%%%%%%%%%%%%%%%%%%%%%%%%%%%%%%%%%%%%%%%%

%\input new_concept.tex


\section{试题整理}

%%使用引用方式,防止示例的代码内容被脚本抽取到作为练习内容!
%\input use_exsheets.tex

  \subsection{基础练习题目}

这里计划超前选择一些书面的试题,用于练习、掌握基本的知识。

\begin{question}[type=exercise]
 There is \blank[width=8mm]{} map of the world on \blank[width=8mm]{} wall. \blank[width=8mm]{} map is mine.
 \begin{tasks}[counter-format=tsk[A])](4)
 \task a, a, A
 \task a, the, The
 \task the, the, The
 \task the, the, A
\end{tasks}
\end{question}

\begin{question}[type=exercise]
 I bought \blank[width=8mm]{} shoes yesterday. \blank[width=8mm]{} shoes are very beautiful.
 \begin{tasks}[counter-format=tsk[A])](4)
 \task a, The
 \task a pair of, The
 \task the, The
 \task a pair, The pair
\end{tasks}
\end{question}

下面的问题为根据描述写出对应的事物:

\begin{question}[type=exercise] \marginpar{2010学而思杯小学二年级试题}
 This is a bird that is very clever and can talk. \hspace{0.5cm} \blank[width=2cm]{}
\end{question}

\begin{question}[type=exercise]
 This is the place where you put things that you don't use. \hspace{0.5cm} \blank[width=2cm]{}
\end{question}

  \subsection{新课标的准备}

下面是为了准备2018年的新课标1级做准备的练习题目:

\newpage

%\input xinkebiao.tex

  \subsection{小升初专题}

这里收集小升初的一些难读较大的选拔试题,争取能够在小学4年级前达到对应水平!

%\input raise.tex

  \subsection{中考专题}
这里虽然是初中水平,还是希望笑笑在小学高年级能够达到这个水准。

%\input zhongkao.tex
%\input scrapy.eng.1.tex

  \subsection{旧高考专题}
%\input oldgaokao.tex


  \subsection{新高考专题}
TODO

  \subsection{错题整理}

% \input gzeng.tex

  Can you see this long words of text?


\begin{question}[type=exam,name=abc]
What's this?

\end{question}

\begin{question}[type=exam]
What's \blank[width=1cm]{} this?
\vspace{-0.3cm} % This is bad, but it works
\begin{tasks}(3)
 \task hello
 \task world
 \task good
\end{tasks}

\end{question}

\begin{question}[type=exam]\label{q:abc}
What's \blank[width=1cm] this?
\vspace{-0.3cm} % This is bad, but it works
\begin{tasks}(3)
 \task hello
 \task world
 \task good
\end{tasks}

\end{question}

参照Ex-~\ref{q:abc}

 \subsection{练习题的分类设计}

从2017年开始,发现借助于\lstinline{exsheets}等包,可以制作比较合适的练习题内容。

为了能够更加合理地整理和运用电子文档操作练习题的功能,这里把各个学科的练习题目总类和分法都提前设计好,这样后面随着长时间的收集与练习,可以非常方便的抽取出自己想要的练习主题出来。

\begin{table}
\centering
      \def\arraystretch{1.15}
        \begin{tabular}{l | l |  l }
        \hline
        {学科} & {练习分类}& {分类说明}  \\
            \hline
            \multirow{6}{*} {语文} & \multirow{3}{*} {topic}      & \multicolumn{1}{l}{pinyin : 拼音,这是一个很大的主题,class则会细化} \\\cline{3-3}
                                  &                              & \multicolumn{1}{l}{cy : 成语} \\\cline{3-3}
                                  &                              & \multicolumn{1}{l}{poem : 古诗词} \\\cline{2-3}
                                  & \multirow{3}{*}{class}       & \multicolumn{1}{l}{wrongone: 一年级错题}     \\\cline{3-3}
                                  &                              & \multicolumn{1}{l}{pyone: 一年级拼音练习} \\\cline{3-3}
                                   &                              & \multicolumn{1}{l}{numpy.pi} \\\hline

        \end{tabular}

    \caption{语文练习分类的设计}
\end{table}

通常情况下,可以通过class/topic两种方式来指定练习的种类,这样可以自己根据class/topic的设置值来自由抽取特定类型的题目。

比如经过简短的考虑和设计,决定笑笑学校期间的错题也作为一个class类别放在所有练习题库中,原因是:

\begin{itemize}
 \item 错题的总体数量不会太多,对现有的收集题库影响不大
 \item 把错题class再加上对应的topic,可以总结出容易出错的知识模块
 \item 无论是单独只抽取错题,还是把它们混合到对应的主题topic模块,都能有效地、有针对性的进行温习
\end{itemize}

但是,class/topic目前只能指定一个数值。所以当我们需要作一些交叉类型题目的抽取的时候,这种方式就比较局限了。

举个例子,我想方便地把各年的希望杯试题都抽取出来,同时,这当中有的题目还要作为各个年级易错题目可抽取,且试题总数、以及各年的试题、或者各个年级的试题,都需要能够自由抽取。

为了处理这种场景,就需要使用tags的选项。使用该方式的最大好处就是可以按照任意的组合方式来抽取试题。\marginpar{\TeX{}-Live 2016支持该选项}

\begin{table}
\centering
      \def\arraystretch{1.15}
        \begin{tabular}{l | l |  l }
        \hline
        {学科} & {练习分类}& {分类说明}  \\
        \hline
        \multirow{16}{*} {数学} & \multirow{2}{*} {topic}      & \multicolumn{1}{l}{multi: 乘法问题} \\\cline{3-3}
                                &                              & \multicolumn{1}{l}{complex : 复数} \\\cline{2-3}
                                & \multirow{3}{*}{class}       & \multicolumn{1}{l}{wrongone: 一年级错题}     \\\cline{3-3}
                                &                              & \multicolumn{1}{l}{wrong/two/...依次二,三年级错题...} \\\cline{3-3}
                                 &                             & \multicolumn{1}{l}{numpy.pi} \\\cline{2-3}
                                 & \multirow{2}{*} {希望杯}      & \multicolumn{1}{l}{hope: 希望杯试题} \\\cline{3-3}
                                  &                              & \multicolumn{1}{l}{hopejf :希望杯小学四年级(Junior Four)} \\\cline{2-3}
                                  & \multirow{3}{*} {华杯赛}      & \multicolumn{1}{l}{huashu: 华杯赛试题} \\\cline{3-3}
                                  &                              & \multicolumn{1}{l}{huashujf :华杯赛小中(Junior Four)} \\\cline{3-3}
                                  &                              & \multicolumn{1}{l}{huashujs :华杯赛小高(Junior Six)} \\\cline{2-3}
                                  & \multirow{3}{*} {走美杯}      & \multicolumn{1}{l}{zoumei3: 走美三年级试题} \\\cline{3-3}
                                  &                              & \multicolumn{1}{l}{huashujf :华杯赛小中(Junior Four)} \\\cline{3-3}
                                  &                              & \multicolumn{1}{l}{huashujs :华杯赛小高(Junior Six)} \\\cline{2-3}
                                  & \multirow{3}{*} {学而思}      & \multicolumn{1}{l}{xes: 学而思培训内容} \\\cline{3-3}
                                  &                              & \multicolumn{1}{l}{xesjf :小中阶段内容} \\\cline{3-3}
                                  &                              & \multicolumn{1}{l}{xesjs :小高阶段内容} \\\hline

        \end{tabular}

    \caption{数学练习分类的设计}
\end{table}

数学的topic分类有:

tuse - 涂色问题(多见于小学数学场景)

TODO -


理论上,表格中列举的topic/class数值都可以移到tags当中去,这样可以非常方便地处理任意组合的试题抽取工作。

\begin{table}
\centering
      \def\arraystretch{1.15}
        \begin{tabular}{l | l |  l }
        \hline
        {学科} & {练习分类}& {分类说明}  \\
        \hline
        \multirow{5}{*} {英语} & \multirow{2}{*} {topic}      & \multicolumn{1}{l}{phonics: 自然拼读练习} \\\cline{3-3}
                                  &                              & \multicolumn{1}{l}{basicwrite : 基础书写} \\\cline{2-3}
                                  & \multirow{3}{*}{tags}       & \multicolumn{1}{l}{wrongone: 一年级错题,这里用作新课标1级}     \\\cline{3-3}
                                  &                              & \multicolumn{1}{l}{listenwrBasic: 基本听写或者默写} \\\cline{3-3}
                                   &                              & \multicolumn{1}{l}{gram1:一级语法,一次类推} \\\hline

        \end{tabular}

    \caption{英语练习分类的设计}
\end{table}

  \subsection{音标的输出}

目前在\LaTeX{]下可以使用\lstinline{tipa}包来输出英文的音标, 也可以在一定条件下和\XeTeX{}使用.

这个条件就是包的命名放在fontspec之前。

\begin{lstlisting}
%% MUST defined before fontspce
\usepackage{tipa}

\usepackage{fontspec,xunicode,xltxtra,makeidx,xecolor}
\end{lstlisting}

音标的输出符号基本和字母类似,无需死记这些,只要能在适当时候查询得到即可:

\bgroup
\def\arraystretch{1.15}
\begin{tabular}{|r|l|r|l|r|l|}
\hline
{音标} & {字母} & {音标} & {字母} & {音标} & {字母}\\
\hline
{\begin{IPA}[@]\end{IPA}} & {\verb+@+} & {\begin{IPA}[E]\end{IPA}} & {E}  & {\begin{IPA}[S]\end{IPA}} & {S}\\
{\begin{IPA}[O]\end{IPA}} & {O} & {\begin{IPA}[\ae]\end{IPA}}  & {\verb+\ae+} & {\begin{IPA}[N]\end{IPA}} & {N}\\
{\begin{IPA}[2]\end{IPA}} & {2} & {\begin{IPA}[Z]\end{IPA}}  & {Z} & {\begin{IPA}[D]\end{IPA}} & {D}\\
{\begin{IPA}[T]\end{IPA}} & {T} & {\begin{IPA}[U]\end{IPA}}  & {U} & {\begin{IPA}[B]\end{IPA}} & {B}\\
{\begin{IPA}[3]\end{IPA}} & {3} & {}  & {} & {} & {}\\
\hline
\end{tabular}
\egroup

\section{小学错题}

\input xiaoxue.tex

\begin{thebibliography}{99}
\bibitem{bibNewConceptEngYoungVer} 外语教学与研究出版社:{\em 新概念英语青少版入门级}, 2010年.
\bibitem{bibEngJuniorOneA}译林出版社:{\em 小学英语(一上)}, 2012年.
\bibitem{bibEngJuniorOneB}译林出版社:{\em 小学英语(一下)}, 2012年.
\bibitem{bibPhonicsStageTwo} 外语教学与研究出版社:{\em 丽声拼读故事会 第二级}, 2011年.
\bibitem{bibNationalGeoBasic} 外语教学与研究出版社:{\em 国家地理儿童百科入门级}, 2010年.
\bibitem{bibNewDongFangReadBasic} 西安交通大学出版社:{\em 新东方泡泡剑桥儿童英语故事阅读}, 2012年.
\bibitem{bibCaliforniaMath}McGraw-Hill Press:{\em 加州小学数学}, 2012年.
\end{thebibliography}


%%%%Index%%%%%
\printglossaries

\printindex

\end{document}
